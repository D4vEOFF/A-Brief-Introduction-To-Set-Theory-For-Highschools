\section{Relace podrobněji}\label{sec:relace_podrobneji}
U~relací ještě chvíli zůstaneme, neboť ty budou pro nás v~dalším textu nejpodstatnější. Mezi nimi lze najít mnoho zvláštních typů, které jsou svojí strukturou zajímavější než ty, které jsme si ukazovali doteď. Pokud se čtenář do této chvíle stihl ztratit v~záplavě nových pojmů a znalostí, doporučuji se vrátit k~sekcím \ref{sec:zavedeni_relace} o~relacích a \ref{sec:zobrazeni} o~zobrazeních.

\subsection{Druhy relací}\label{subsec:druhy_relaci}
Zatím jsme si uvedli pouze jeden zvláštní typ relace, který nám (nejspíše) byl již trochu povědomý, a to sice zobrazení. Lze se však setkat i s~relacemi, které jsou svojí povahou zcela odlišné. V~této sekci si zavedeme dva takové typy. Nejdříve se však podíváme na nejdůležitější 4 druhy relací.\par
Ačkoliv jsme si zaváděli relaci obecně mezi dvěma množinami $X$ a $Y$, v~dalším textu se omezíme již pouze na relace na množině.
\begin{definition}[Důležité druhy relací]\label{def:dulezite_druhy_relaci}
    Nechť $R$ je relace na množině $X$. Pak $R$ je
    \begin{enumerate}[label=(\roman*)]
        \item \emph{reflexivní}, jestliže $\forall x\in X: xRx$.
        \item \emph{symetrická}, jestliže $\forall x,y\in X: xRy \implies yRx$.
        \item \emph{tranzitivní}, jestliže $\forall x,y,z\in X: xRy \land yRz \implies xRz$.
        \item \emph{antisymetrická}, jestliže $\forall x,y\in X: xRy \land yRx \implies x=y$.
    \end{enumerate}
\end{definition}
Reflexivní relace je jednoduše taková relace, kdy jsou všechny prvky v~relaci samy se sebou (viz \ref{fig:priklady_reflexivnich_relaci}).
\begin{figure}[H]
    \centering
    \begin{subfigure}{6cm}
        \centering
        \includegraphics[scale=\normalipe]{ch02_reflexivni_relace.pdf}
        \caption{Reflexivní relace na čtyřech prvcích.}
        \label{subfig:reflexivni_relace}
    \end{subfigure}
    \quad
    \begin{subfigure}{6cm}
        \centering
        \includegraphics[scale=\normalipe]{ch02_nejmensi_reflexivni_relace.pdf}
        \caption{Reflexivní relace na třech prvcích.}
        \label{subfig:nejmensi_reflexivni_relace}
    \end{subfigure}
    \caption{Příklady reflexivních relací.}
    \label{fig:priklady_reflexivnich_relaci}
\end{figure}
Speciálně obrázek \ref{subfig:nejmensi_reflexivni_relace} je příkladem nejmenší možné reflexivní relace. Z~definice \ref{def:identita} můžeme vidět, že se jedná o~\emph{identitu}.\par
Podobně si můžeme znázornit i symetrii na obrázku \ref{fig:priklad_symetricke_relace}.
\begin{figure}[H]
    \centering
    \includegraphics[scale=\normalipe]{ch02_symetricka_relace.pdf}
    \caption{Symetrická relace na čtyřech prvcích.}
    \label{fig:priklad_symetricke_relace}
\end{figure}
Z~obrázku \ref{fig:priklad_symetricke_relace} můžeme vidět, že mezi dvojicí prvků, které jsou v~relaci, vedou šipky oběma směry.\par
U~tranzitivity musí platit, že pokud máme šipku mezi $x$ a $y$ a zároveň mezi $y$ a $z$, pak musí být i šipka mezi $x$ a $z$ (viz \ref{fig:priklad_tranzitivni_relace}).
\begin{figure}[H]
    \centering
    \includegraphics[scale=\normalipe]{ch02_tranzitivni_relace.pdf}
    \caption{Tranzitivní relace na čtyřech prvních.}
    \label{fig:priklad_tranzitivni_relace}
\end{figure}
Antisymetrie je pravděpodobně nejtěžší z~těchto druhů relací, co se týče definice. Zatímco u~symetrie platí, že relace musí být "vzájemná", u~antisymetrie naopak říkáme, že pokud jsou prvky ve "vzájemné" relaci, pak se jedná o~tentýž prvek. Z~toho si však můžeme uvědomit, že u~antisymetrické relace nemůže tedy nastat, že by mezi dvěma prvky vedly šipky oběma směry, jak lze vidět na obrázku \ref{fig:priklad_antisymetricke_relace}.
\begin{figure}[H]
    \centering
    \includegraphics[scale=\normalipe]{ch02_antisymetricka_relace.pdf}
    \caption{Antisymetrická relace na třech prvcích.}
    \label{fig:priklad_antisymetricke_relace}
\end{figure} 
Nyní si zadefinujeme další důležitý pojem v~definici \ref{def:inverzni_relace}.
\begin{definition}[Inverzní relace]\label{def:inverzni_relace}
    Nechť $R$ je relace na množině $X$. \emph{Inverzní relací} k~relaci $R$ nazýváme relaci
    \begin{equation*}
        R^{-1}=\set{(y,x) \admid xRy}.
    \end{equation*}
\end{definition}
Proč právě inverzní? Mějme libovolnou relaci $R\subseteq X\times Y$ a k~ní inverzní relaci $R^{-1}$ (ta je naopak podmnožinou "obráceného" kartézského součinu $Y\times X$). Zkusme relace $R$ a $R^{-1}$ složit (pro připomenutí viz definice \ref{def:skladani_relaci}).
\begin{equation*}
    R\circ R^{-1}=\set{(x,z) \admid \exists y\in Y: xRy \land yR^{-1}z}
\end{equation*}
Ovšem víme, že když $xRy$, pak $yR^{-1}x$, což znamená, že $x(R\circ R^{-1})x$. Tedy složením získáme identitu:
\begin{equation*}
    R\circ R^{-1}=1_X.
\end{equation*}
Stejně je tomu i u~zobrazení. Čtenář pravděpodobně již slyšel termín \emph{inverzní funkce}. Zde se nám tato znalost krásně propojuje se středoškolským učivem.
\begin{remark}\label{rem:existence_inv_zobrazeni}
    Inverzní zobrazení $f^{-1}$ k~$f$ existuje právě tehdy, když $f$ je \textbf{prosté}.
\end{remark}
Některé příklady jsou níže.
\createcnt{funcex_cnt}
\begin{enumerate}[label=(\roman*)]
\item Funkce $\map{f_{\printnstepcnt{funcex_cnt}}}{\R}{\R^+}$, kde $f_{\printcnt{funcex_cnt}}(x)=e^x$; inverzní funkce $\map{f_{\printcnt{funcex_cnt}}^{-1}}{\R^+}{\R}$, kde $f_{\printcnt{funcex_cnt}}^{-1}(x)=\ln{x}$.
\item \sloppy Funkce $\map{f_{\printnstepcnt{funcex_cnt}}}{\left\langle-\frac{\pi}{2},\frac{\pi}{2}\right\rangle}{\langle-1,1\rangle}$, kde $f_{\printcnt{funcex_cnt}}(x)=\sin{x}$; inverzní funkce ${\map{f_{\printcnt{funcex_cnt}}^{-1}}{\langle-1,1\rangle}{\left\langle-\frac{\pi}{2},\frac{\pi}{2}\right\rangle}}$, kde $f_{\printcnt{funcex_cnt}}^{-1}=\arcsin{x}$,
\item Funkce $\map{f_{\printnstepcnt{funcex_cnt}}}{\R}{\R}$, kde $f_{\printcnt{funcex_cnt}}(x)=x^3-1$; inverzní funkce $\map{f_{\printcnt{funcex_cnt}}^{-1}}{\R}{\R}$, kde $f_{\printcnt{funcex_cnt}}^{-1}(x)=\sqrt[3]{x+1}$.
\end{enumerate}
Je vidět, že složením libovolné $f_i$ s~$f_i^{-1}$ dostaneme identitu $\bigl(f_i\circ f_i^{-1}\bigr)(x)=x$.

\subsection{Relace uspořádání}\label{subsec:relace_usporadani}
Relace lze dále klasifikovat podle jejich typů. Každý z~nich se vyznačuje právě podle definovaných druhů v~\ref{def:dulezite_druhy_relaci}, do nichž spadají. Pro nás bude velmi podstatné tzv. \emph{uspořádání}. (Vedle tohoto typu ještě existuje tzv. \emph{relace ekvivalence}, o~které si lze přečíst v~příloze \ref{chap:dodatky_k_porovnavani_nekonecnych_mn}.)
\begin{definition}[Uspořádání]\label{def:usporadani}
    Nechť $R$ je relace na množině $X$. Řekneme, že $R$ je relací \emph{uspořádání} (též jen \emph{uspořádání}), pokud je \emph{reflexivní}, \emph{antisymetrická} a \emph{tranzitivní}.
\end{definition}
Začněme opět příklady, ať víme, jak si takový typ relace vlastně představit. Mějme relace $R_1$ a $R_2$ na obrázcích \ref{fig:relace_usporadani_1} a \ref{fig:relace_usporadani_2}. (Zkuste si sami rozmyslet, zda $R_1,R_2$ splňují podmínky uspořádání.)
\begin{figure}[H]
    \centering
    \includegraphics[scale=\normalipe]{ch02_relace_usporadani_1.pdf}
    \caption{Relace uspořádání $R_1$ na šesti prvcích.}
    \label{fig:relace_usporadani_1}
\end{figure}
\begin{figure}[H]
    \centering
    \includegraphics[scale=\normalipe]{ch02_relace_usporadani_2.pdf}
    \caption{Relace uspořádání $R_2$ na devíti prvcích.}
    \label{fig:relace_usporadani_2}
\end{figure}
Naopak ukázky modifikací relací $R_1,R_2$ na obrázcích \ref{fig:relace_neusporadani_1} a \ref{fig:relace_neusporadani_2} \textbf{nejsou} uspořádáními.
\begin{figure}[H]
    \centering
    \includegraphics[scale=\normalipe]{ch02_relace_neusporadani_1.pdf}
    \caption{Relace $R_1\cup (x_4,x_6)$, která není uspořádáním.}
    \label{fig:relace_neusporadani_1}
\end{figure}
\begin{figure}[H]
    \centering
    \includegraphics[scale=\normalipe]{ch02_relace_neusporadani_2.pdf}
    \caption{Relace $R_2\setminus (x_6,x_9)$, která není uspořádáním.}
    \label{fig:relace_neusporadani_2}
\end{figure}
\begin{example}
    Další příklady uspořádání:
    \begin{enumerate}[label=(\roman*)]
        \item Relace "$\leq$" na množině $\N$ je uspořádání.
        \item Relace "$\geq$" na množině $\R$ je uspořádání.
        \item Relace "$\mid$" ("býti dělitelem") na množině $\N$ je uspořádání.
        \item Relace "$\subseteq$" na množině $\powset{X}$ je uspořádání, kde $X$ je konečná množina.
    \end{enumerate}
\end{example}
Zde si můžeme všimnout jisté redundance při zakreslování. Konkrétně dvojice vyplývající z~tranzitivity a reflexivity bychom v~těchto případech mohli klidně vynechat a považovat za samozřejmé. Zároveň jsme cíleně zakreslili relace $R_1$ a $R_2$ tak, aby šipky vedly pouze nahoru, neboť to činí obrázek přehlednějším. (Toto funguje díky tomu, že v~relaci uspořádání se nemohou vyskytnout cykly kvůli podmínce antisymetrie a tranzitivity, nepočítáme-li cykly z~reflexivity.) Zkuste se schválně podívat na znázornění relace $R_1$ na obrázku \ref{fig:relace_usporadani_1_jinak} pro srovnání.
\begin{figure}[H]
    \centering
    \includegraphics[scale=\normalipe]{ch02_relace_usporadani_1_jinak.pdf}
    \caption{Relace $R_1$ zakreslená podle pořadí indexů prvků.}
    \label{fig:relace_usporadani_1_jinak}
\end{figure}
Tím se dostáváme k~tzv. \emph{Hasseovým diagramům}, které se nám poskytují velmi komfortní reprezentaci uspořádání. Dosavadní způsoby reprezentace relací v~sobě obsahovaly jistou míru libovůle, co do umístění prvků v~diagramech, neboť důležité byly pouze šipky mezi nimi. Zde toto již neplatí, neboť umístěním prvků budeme určovat, které jsou společné v~relaci. Přijmeme konvenci, že šipky mezi prvky povedou pouze směrem nahoru. Tzn. pokud $xRy$, pak $x$ bude nakresleno níž oproti $y$. Na obrázku \ref{fig:relace_usporadani_zakresleni} máme zakreslené uspořádání na množině $\set{y_1,y_2,y_3,y_4}$.
\begin{figure}[H]
    \centering
    \begin{subfigure}{6cm}
        \centering
        \includegraphics[scale=\normalipe]{ch02_relace_usporadani_standard.pdf}
        \caption{Standardní zakreslení $S$}
        \label{subfig:relace_usporadani_standard}
    \end{subfigure}
    \qquad
    \begin{subfigure}{6cm}
        \centering
        \includegraphics[scale=\normalipe]{ch02_relace_usporadani_hasseuv_diagram.pdf}
        \caption{Hasseův diagram $S$}
        \label{subfig:relace_usporadani_hasseuv_diagram}
    \end{subfigure}
    \caption{Relace $S$ zakreslená standardně a pomocí Hasseova diagramu.}
    \label{fig:relace_usporadani_zakresleni}
\end{figure}
Podobně bychom můžeme zakreslit i relaci $R_1$ z~obrázku \ref{fig:relace_usporadani_1} (viz obrázek \ref{fig:relace_usporadani_1_hasseuv_diagram}).
\begin{figure}[H]
    \centering
    \includegraphics[scale=\normalipe]{ch02_relace_usporadani_1_hasseuv_diagram.pdf}
    \caption{Hasseův diagram relace $R_1$.}
    \label{fig:relace_usporadani_1_hasseuv_diagram}
\end{figure}
(Sekce inspirována \cite{MatousekNesetril2009}, str. 44--48 a str. 55)