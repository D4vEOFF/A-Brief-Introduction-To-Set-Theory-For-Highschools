\section{Axiomy 4 až 6}\label{sec:axiomy_4_az_6}
První trojice axiomů se zdá být dobrým základem, avšak stále je stále hodně typů množin, jejichž existence z nich neplyne. Trochu "podvodným" způsobem jeden takový typ použili v diskuzi axiomu extenzionality, konkrétně množinu $\set{a,b,c}$. Při zamyšlení zjistíme, že čistě z axiomů \ref{item:axiom_extenzionality}, \ref{item:axiom_existence} a \ref{item:axiom_dvojice} nelze takovou množinu "sestrojit". Pomocí axiomu dvojice plyne pro množiny $a,b,c$ existence množin
\begin{equation*}
    \set{a,b}\;\text{a tudíž i}\;\set{\set{a,b},c},
\end{equation*}
což jak víme, není to samé co $\set{a,b,c}$. Její existenci a existenci mnoha dalších množin nám zaručí (společně se \ref{item:axiom_extenzionality}, \ref{item:axiom_existence} a \ref{item:axiom_dvojice}) axiomy \ref{item:schema_axiomu_vydeleni}, \ref{item:axiom_potence} a \ref{item:axiom_sumy}.

\subsubsection{Schéma axiomů vydělení}
\begin{equation}\label{eq:schema_axiomu_vydeleni}
    \forall a\,\exists y\,\forall x\,(x\in y \iff x\in a \land \varphi(x)),
\end{equation}
kde $\varphi(x)$ je formule neobsahující proměnnou $y$.\\
Často potřebujeme z určité množiny prvků vybrat množinu prvků takových, že všechny sdílejí jistou vlastnost. Např.
\begin{itemize}
    \item všechna sudá čísla z množiny $\Z$,
    \item všechna nezáporná čísla z množiny $\R$,
    \item všechna prvočísla z množiny $\N$, apod.
\end{itemize}
Schéma axiomů vydělení\footnote{Slovo "schéma" přidáváme z důvodu, že pro každou volbu formule $\varphi$ dostáváme jeden konkrétní axiom teorie -- axiom vydělení. Tedy schéma axiomů vydělení představuje nekonečně mnoho různých axiomů, které vzniknou tím, že $\varphi$ proběhne všechny možné formule s proměnnou $x$.} nám obecně říká, že pro každou množinu $a$ existuje množina $y$ taková, že každý její prvek $x$ je zároveň prvek $a$ splňuje určitou formuli $\varphi(x)$ (ta reprezentuje danou vlastnost). Podle axiomu extenzionality \ref{item:axiom_extenzionality} je množina v \eqref{eq:schema_axiomu_vydeleni} jednoznačně určena. Čtenář je nejspíše zvyklý množiny, jejichž prvky sdílejí určitou vlastnost, zapisovat např. jako
\begin{equation*}
    \set{x\in\R \admid x\geq 0}.
\end{equation*}
Obecněji množinu z \eqref{eq:schema_axiomu_vydeleni} zapisujeme výrazem
\begin{equation*}
    \set{x \admid x\in a \land \varphi(x)}\;\text{nebo též}\;\set{x\in a\admid \varphi(x)}.
\end{equation*}
Pokud se nyní vrátíme k množinám, se kterými jsme doteď pracovali, můžeme pro množiny $a$ a $b$ definovat množinu
\begin{equation*}
    \set{x\in a\admid x\notin b},
\end{equation*}
kde formule $\varphi(x)$ je $x\notin b$. Toto schéma axiomů má následující důsledek.
\begin{corollary}
    Existuje množina, která nemá žádné prvky.
\end{corollary}
\begin{proof}
    Důkaz je jednoduchý. Máme-li libovolnou množinu $a$, pak podle schématu axiomů vydělení lze sestrojit množinu, která nemá žádné prvky. Toho docílíme volbou formule $\varphi(x)$ jako $x\neq x$, tzn.
    \begin{equation*}
        \set{x\in a \admid x\neq x}
    \end{equation*}
    je také množina.
\end{proof}
Takovou množinu pak nazýváme \emph{prázdná množina} a typicky ji označujeme znakem $\emptyset$ nebo též někdy prázdnými složenými závorkami $\set{}$. Toto tvrzení se v jiných variantách ZF považuje za axiom a nahrazuje axiom existence. Všimněte si, že důkaz výše (a prakticky důkazy všech zatím zformulovaných tvrzení) závisely (mimo jiné) právě na axiomu existence. Jinak bychom množinu $a$ vůbec nemohli uvažovat. Naopak pokud bychom přijali existenci prázdné množiny jako axiom, pak to automaticky implikuje existenci množiny obecně, kterou nám zaručuje axiom existence.\par
Pomocí schématu axiomů vydělení můžeme definovat některé základní operace s množinami.