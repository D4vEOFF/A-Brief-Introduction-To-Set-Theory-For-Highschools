\chapter{Dodatky k~logice}\label{chap:dodatky_k_logice}
V kapitole o výrokovém a predikátovém počtu jsme pracovali s výrokovými a později predikátovými formulemi, kde jsme si pouze neformálně vysvětlili, co pod těmito termíny rozumíme. Formálněji můžeme k~těmto záležitostem přistoupit pomocí následující metadefinice.
\begin{definition}[Výroková a atomická formule]\label{def:vyrokova_a_atomicka_formule}
    
    \begin{enumerate}[label=(\roman*)]
        \item\label{item:formule_1} Každá výroková proměnná je \emph{výroková formule} (tzv. \emph{atomická formule}).
        \item\label{item:formule_2} Jsou-li $\varphi$ a $\psi$ výrokové formule, pak $\neg (\varphi)$, $(\varphi) \land (\psi)$, $(\varphi) \lor (\psi)$, $(\varphi) \implies (\psi)$ a $(\varphi) \iff (\psi)$ jsou také výrokové formule.
        \item\label{item:formule_3} Výraz, který nelze získat pomocí pravidel \ref{item:formule_1} a \ref{item:formule_2} není výrokovou formulí.
    \end{enumerate}
\end{definition}
(Převzato z \cite{Fuchs2003}, str. 14 a \cite{BalcarStepanek1986}, str. 30).\par
Definice výrokové formule \ref{def:vyrokova_a_atomicka_formule} nám v~podstatě říká, jakým způsobem můžeme sestrojit potenciálně všechny možné formule. Mějme výrokové proměnné $A$, $B$ a $C$. Ty jsou podle \ref{item:formule_1} v definici \ref{def:vyrokova_a_atomicka_formule} výrokovými formulemi. Podle \ref{item:formule_2} jsou pak formulemi i výrazy
\begin{equation}\label{eq:vyrokove_formule_z_definice}
    (A) \land (B)\;,\;(A) \lor (C)\;\text{a}\;\neg(B).
\end{equation}
Nyní můžeme opakovaně použít \ref{item:formule_2} k sestrojení dalších složitějších formulí. Tedy užitím formulí \eqref{eq:vyrokove_formule_z_definice} můžeme dále postupně sestrojit např. výrazy
\begin{equation*}
    \bigl((A) \land (B)\bigr) \implies \bigl((A) \lor (C)\bigr)\quad\text{a}\quad\Bigl((A) \land \bigl(\neg(B)\bigr)\Bigr) \iff \Bigl(\neg\bigl((A) \lor (B)\bigr)\Bigr),
\end{equation*}
které jsou opět podle \ref{item:formule_2} výrokovými formulemi. Opětovným užitím \ref{item:formule_2} pak je dále výrokovou formulí např.
\begin{equation*}
    \Bigl(\bigl((A) \land (B)\bigr) \implies \bigl((A) \lor (C)\bigr)\Bigr) \implies \bigg(\Bigl((A) \land \bigl(\neg(B)\bigr)\Bigr) \iff \Bigl(\neg\bigl((A) \lor (B)\bigr)\Bigr)\bigg).
\end{equation*}
Takto můžeme postupovat dál a opakovanou aplikací pravidla \ref{item:formule_2} vytvořit ještě složitější výrokové formule.\par
Naopak pokud bychom uvážili nějaký výraz, můžeme obdobně zjistit, jestli se jedná o výrokovou formuli či nikoliv.
\begin{example}\label{ex:overeni_formule}
    Mějme výraz
    \begin{equation*}
        \varphi\sim\bigl((A) \land (C)\bigr) \iff \Bigl(\bigl((A) \lor (B)\bigr) \implies \neg(C)\Bigr).
    \end{equation*}
    Ověřte, zda $\varphi$ je výroková formule.\par
    \begin{solution}
        Aby $\varphi$ byla formule, musí být
        \begin{equation*}
            \varphi_1\sim(A) \land (C)\quad\text{a}\quad\varphi_2\sim\bigl((A) \lor (B)\bigr) \implies \neg(C)
        \end{equation*}
        též formulemi. Výraz $\varphi_1$ zřejmě je formulí, neboť $A$ a $C$ jsou atomické formule. Ovšem u $\varphi_2$ lze již vidět, že výraz nesplňuje definici výrokové formule, neboť u $\neg(C)$ chybí vnější závorky. Z bodu \ref{item:formule_3} definice \ref{def:vyrokova_a_atomicka_formule} tedy plyne, že výraz $\varphi$ \textbf{není výrokovou formulí}, neboť jej nelze získat pomocí pravidel \ref{item:formule_1} a \ref{item:formule_2}.
    \end{solution}
\end{example}

Čtenáře možná již napadlo, že formule, které jsme sestrojili z definice \ref{def:vyrokova_a_atomicka_formule}, jsou zapsány poměrně komplikovaně, především co do nadměrného používání závorek. Např. výraz
\begin{equation}\label{eq:poradi_operaci}
    A \land \neg C,
\end{equation}
není podle \ref{def:vyrokova_a_atomicka_formule} výrokovou formulí. I přesto je však nejspíše zřejmé, že tímto zápisem vyjadřujeme výrok "Platí $A$ a zároveň neplatí $C$.". Nebo vrátíme-li se k~příkladu \ref{ex:overeni_formule}, i při vypuštění uzávorkování u výrazu $\varphi_2$ by dávalo smysl interpretovat výraz
\begin{equation*}\label{eq:vypusteni_uzavorkovani}
    A \lor B \implies \neg C
\end{equation*}
jako výrok "Jestliže platí $A$ a zároveň $B$, pak neplatí $C$.".