\section{Speciálně o uspořádáních a uspořádaných množinách}
Uspořádání v matematice lze dále klasifikovat. Už jsme měli možnost vidět různé příklady tohoto typu relace, kdy naším primárním cílem bylo si ilustrovat jeho vlastnosti, jimiž je definována a ukázat si pro nás jejich výhodný jejich způsob zakreslování -- \emph{Hasseovými diagramy}. V této části se více zaměříme na vlastnosti uspořádání, zavedeme si další typy a především se podíváme na tzv. \emph{dobře uspořádané množiny}.\par
V ohledu, v jakém se budeme dále uspořádáním věnovat, se nám bude lépe pracovat s následujícím termínem.
\begin{definition}[Uspořádaná množina]
    Nechť $R$ je uspořádání na množině $X$. Pak uspořádanou dvojici $(X,R)$ nazýváme \emph{uspořádaná množina}.
\end{definition}

\subsection{Lineární a částečná uspořádání}
Uveďme si nejdříve definici těchto pojmů a poté si vysvětleme jejich význam.
\begin{definition}[Lineární uspořádání]
    Nechť $(X,R)$ je uspořádaná množina. Pak $R$, resp. $(X,R)$ nazýváme \emph{lineárním uspořádáním}, resp. \emph{lineárně uspořádanou množinou}, jestliže $\forall x,y\in X: xRy \lor yRx$.
\end{definition}
Pojem \emph{částečné uspořádání}, resp. \emph{částečně uspořádaná množina} jsou používány naopak pro uspořádání, která nemusí být nutně lineární. Jedná se tedy pouze o obšírnější termín. Obecně platí, že každá lineárně uspořádaná množina je částečně uspořádanou, avšak ne každá částečně uspořádaná množina je lineárně uspořádanou.
\begin{example}\label{ex:cast_a_lin_usporadani}
    Klasifikace některých známých uspořádání a jejich Hasseovy diagramy.
    \begin{enumerate}[label=(\roman*)]
        \item\label{item:usporadani_i} $(\N,\leq)$ je lineárně uspořádaná množina.
        \begin{figure}[H]
            \centering
            \includegraphics{ch02_priklad_usporadani_i.pdf}
            \caption{Diagram uspořádané množiny $(\N,\leq)$.}
            \label{fig:priklad_usporadani_i}
        \end{figure}
        \item\label{item:usporadani_ii} $(S,\mid)$, kde $S=\set{1, 2, 3, 4, 5, 6, 10, 12, 15, 20, 30, 60}$ (množina všech dělitelů čísla 60), je částečně (nikoliv však lineárně) uspořádaná množina.
        \begin{figure}[H]
            \centering
            \includegraphics{ch02_priklad_usporadani_ii.pdf}
            \caption{Diagram uspořádané množiny $(S,\mid)$.}
            \label{fig:priklad_usporadani_ii}
        \end{figure}
        \item\label{item:usporadani_iii} $(\powset{X}, \subseteq)$, kde $X=\set{a,b,c,d}$, je částečně (nikoliv však lineárně) uspořádaná množina.
        \begin{figure}[H]
            \centering
            \includegraphics{ch02_priklad_usporadani_iii.pdf}
            \caption{Diagram uspořádané množiny $(\powset{X}, \subseteq)$.}
            \label{fig:priklad_usporadani_iii}
        \end{figure}
    \end{enumerate}
\end{example}
Lze si všimnout, že díky antisymetrii tvoří prvky jistou "hierarchii". Např. v \ref{item:usporadani_i} v předešlém příkladu \ref{ex:cast_a_lin_usporadani} můžeme vidět z obrázku \ref{fig:priklad_usporadani_i}, že každému prvku "předchází" jiný prvek až na číslo 1. Podobně je tomu tak i u \ref{item:usporadani_ii} a \ref{item:usporadani_iii}. S tím se pojí následující terminologie.
\begin{definition}[Bezprostřední předchůdce]\label{def:predchudce}
    Nechť $(X,R)$ je uspořádaná množina. Řekneme, že $x\in X$ je \emph{bezprostředním předchůdcem} prvku $y\in X$, pokud platí podmínky:
    \begin{enumerate}[label=(\roman*)]
        \item $xRy$,
        \item $\forall t\in X: \neg(xRt \land tRy)$.
    \end{enumerate}
\end{definition}
Podmínka \textit{(ii)} definice \ref{def:predchudce} výše říká, že neexistuje prvek $t$ takový, že by se nacházel "mezi" $x$ a $y$ (viz obrázek \ref{fig:predchudce_upresneni}).
\begin{figure}[H]
    \centering
    \begin{subfigure}{6cm}
        \centering
        \includegraphics{ch02_predchudce_upresneni_1.pdf}
    \end{subfigure}
    \qquad
    \begin{subfigure}{6cm}
        \centering
        \includegraphics{ch02_predchudce_upresneni_2.pdf}
    \end{subfigure}
    \caption{Hasseovy diagramy k upřesnění definice \ref{def:predchudce}.}
    \label{fig:predchudce_upresneni}
\end{figure}
