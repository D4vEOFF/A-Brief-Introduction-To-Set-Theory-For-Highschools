\chapter{Dodatky k~porovnávání nekonečných množin}\label{chap:dodatky_k_porovnavani_nekonecnych_mn}
Dodatečná ukázka Cantorova diagonálního argumentu při důkazu Cantorovy věty pro spočetné množiny.
\begin{theorem}\label{thm:cantorova_veta_spocetne}
    Pro libovolnou spočetnou množinu $X$ platí
    \begin{equation*}
        X\prec\powset{X}.
    \end{equation*}
\end{theorem}
\begin{proof}
    Ukážeme, že $X\napprox\powset{X}$ (případ $X\preccurlyeq\powset{X}$ již známe). K tomu lze přistoupit sporem. Pro spor nechť $X\approx\powset{X}$, tzn. existuje bijektivní zobrazení $\map{f}{X}{\powset{X}}$. Obdobně jako v~důkazu věty \ref{thm:N_a_R}, i~zde ukážeme, že $f$ není surjektivní. Obrazy prvků $x_i\in X$ tak můžeme "uspořádat" do "seznamu" $f(x_1),f(x_2),f(x_3),\dots$. Podmnožinu $A$ můžeme z množiny sestrojit $X$ tak, že pro každý z prvků množiny $X$ určíme, zda náleží $A$ či nikoliv. Označíme-li si případ $x\in A$ písmenem A a~případ $x\notin A$ jako N, pak zmíněný "seznam" bychom mohli znázornit podobně jako na obrázku \ref{fig:seznam_podmnoziny}.
    \begin{figure}[H]
        \centering
        \includegraphics[scale=\normalipe]{ch06_seznam_podmnoziny.pdf}
        \caption{Podmnožiny (obrazy) množiny $X$ určené náležením každého z prvků.}
        \label{fig:seznam_podmnoziny}
    \end{figure}
    Opět se zaměříme na diagonálu tohoto seznamu.
    \begin{figure}[H]
        \centering
        \includegraphics[scale=\normalipe]{ch06_diagonala_podmnoziny.pdf}
        \caption{Diagonála seznamu podmnožin množiny $X$.}
        \label{fig:diagonala_podmnoziny}
    \end{figure}
    Zkonstruujeme množinu $S$ tak, že každý prvek $x$ na diagonále jí náleží právě tehdy, když nenáleží podmnožině (tedy obrazu $f(x)$) v~příslušném řádku. Evidentně množina $S$ je podmnožinou $X$. Zároveň se však od každé podmnožiny na seznamu liší minimálně v~prvku na diagonále. To znamená, že $S$ nemůže být na seznamu, čímž dostáváme spor.
\end{proof}
\section{Relace ekvivalence}\label{sec:relace_ekvivalence}
Zmíněné druhy relací nám dovolují definovat dva jejich nejdůležitější typy, jednomu z nichž se budeme dále přednostně věnovat. Začneme prvním z nich.
\begin{definition}[Relace ekvivalence]\label{def:relace_ekvivalence}
    Nechť $R$ je relace na množině $X$. Řekneme, že $R$ je \emph{relací ekvivalence na $X$} (nebo jen \emph{ekvivalencí na $X$}), pokud je \emph{reflexivní}, \emph{symetrická} a~\emph{tranzitivní}.
\end{definition}
Ač se to nemusí zdát, tento typ relace má velmi příjemné vlastnosti. Jak si ji představit? Příkladem může být třeba relace $R$ na množině $X=\set{x_1,\dots,x_7}$ znázorněná na obrázku \ref{fig:priklad_relace_ekvivalence} níže.
\begin{figure}[H]
    \centering
    \includegraphics[scale=\normalipe]{ch02_relace_ekvivalence.pdf}
    \caption{Relace ekvivalence $R$ na $X$.}
    \label{fig:priklad_relace_ekvivalence}
\end{figure}
Pokud by však byl např. prvek $x_3$ v~relaci prvkem $x_4$, pak by již $R$ nebyla ekvivalencí, jak lze naopak vidět z obrázku \ref{fig:priklad_relace_neekvivalence}.
\begin{figure}[H]
    \centering
    \includegraphics[scale=\normalipe]{ch02_relace_neekvivalence.pdf}
    \caption{Relace $R \cup (x_3,x_4)$ na $X$.}
    \label{fig:priklad_relace_neekvivalence}
\end{figure}
Všimněte si, že na obrázku \ref{fig:priklad_relace_ekvivalence} jsou prvky rozděleny na "ostrůvky", kde v~rámci každého z nich jsou spolu všechny prvky v~relaci\footnote{U relace ekvivalence též říkáme, že prvky jsou spolu \emph{ekvivalentní}.}. (Zkuste si z definice rozmyslet, že to tak vždy musí být.) Zakreslovat relaci ekvivalence dosavadním se tak stává již celkem nevýhodným, neboť pro větší množství ekvivalentních prvků již zakreslovat všechny vztahy šipkami je v~tomto případě poměrně zdlouhavý proces (např. pro 5 ekvivalentních prvků bychom museli kreslit 25 šipek). Úspornější a~taktéž názornější pro nás bude si pouze schématicky rozdělit prvky do skupinek (relace mezi nimi z definice ekvivalence považujeme za samozřejmé), např. jako na obrázku \ref{fig:relace_ekvivalence_tridy}.
\begin{figure}[H]
    \centering
    \includegraphics[scale=\normalipe]{ch02_relace_ekvivalence_tridy.pdf}
    \caption{Schématické znázornění ekvivalence $R$ na $X$.}
    \label{fig:relace_ekvivalence_tridy}
\end{figure}
Definujme si nyní tyto "ostrůvky" trochu formálněji.
\begin{definition}[Třída ekvivalence]
    Nechť $R$ je relace ekvivalence na množině $X$ a~nechť $x\in X$. Pak definujeme množinu
    \begin{equation*}
        [x]_R=\set{y \admid xRy},
    \end{equation*}
    kterou nazýváme \emph{třída ekvivalence $R$ určená prvkem $x$}.
\end{definition}
Třída ekvivalence $[x]_R$ jistého prvku $x$ tak obsahuje všechny prvky, které jsou s $x$ ekvivalentní. Z příkladu výše je vidět že platí:
\begin{itemize}
    \item $[x_1]_R=[x_2]_R=[x_3]_R=\set{x_1,x_2,x_3}$,
    \item $[x_4]_R=[x_5]_R=\set{x_4,x_5}$,
    \item $[x_6]_R=[x_7]_R=[x_8]_R=\set{x_6,x_7,x_8}$.
\end{itemize}
\begin{example}
    Další příklady relací ekvivalence a~jejich tříd.
    \begin{enumerate}[label=(\roman*)]
        \item $(\N,=)$ (rovnost přirozených čísel) je relace ekvivalence, kde každý prvek tvoří samostatnou třídu ekvivalence.
        \begin{align*}
            [1]_R&=\set{1}\\
            [2]_R&=\set{2}\\
            &\vdots
        \end{align*}
        \item Relace \emph{"mít stejnou paritu"}\footnote{Tzn. obě čísla jsou sudá nebo lichá.} na množině $\Z$ je relace ekvivalence o~dvou třídách (sudá a~lichá čísla).
        \begin{align*}
            [1]_R&=\set{-1,1,-3,3,-5,5,\dots}\\
            [2]_R&=\set{0,-2,2,-4,4\dots}\\
            &\vdots
        \end{align*}
        \item Relace "mít stejný zbytek po celočíselném dělení 5" na množině $\Z$ je relace ekvivalence o~pěti třídách (čísla 0--4 jako zbytek po celočíselném dělení).
        \begin{align*}
            [0]_R&=\set{0,5,10,\dots}\\
            [1]_R&=\set{1,6,11,\dots}\\
            [2]_R&=\set{2,7,12,\dots}\\
            [3]_R&=\set{3,8,13,\dots}\\
            [4]_R&=\set{4,9,14,\dots}
        \end{align*}
        \item Relace "mít stejnou absolutní hodnotu" na množině $\R$ je relace ekvivalence, kde každá třída obsahuje prvek $x$ a~$-x$ pro všechna $x\in\R$.
        \begin{align*}
            [0]_R&=\set{0}\\
            [1]_R&=\set{-1,1}\\
            [\sqrt{2}]_R&=\set{\sqrt{2},-\sqrt{2}}\\
            [\sqrt[4]{30}]_R&=\set{\sqrt[4]{30},-\sqrt[4]{30}}\\
            &\vdots
        \end{align*}
    \end{enumerate}
\end{example}

\section{Mohutnost množiny}\label{sec:mohutnost_mnoziny}
V sekci jsme v~definici subvalence a~ekvipotence \ref{def:subvalence_a_ekvipotence} zmínili pojem "mohutnost" (ostatně zmínili jsme jej ue v~distorickém úvodu). Již víme, co je míněno pod tvrzením, že množina "má větší/stejnou mohutnost" jako jiná množina. To nám však pouze dává představu, jak mohutnosti porovnávat. Jak tuto vlastnost množiny explicitně vyjádřit?\par
V případě konečných množin rozumíme pod "mohutností" množiny jednoduše její velikost (tj. počet prvků), kterou reprezentuje nějaké přirozené číslo. U nekonečných množin je to však složitější. Nelze říci, že mohutnost je $\infty$. Podle této logiky by pak muselo platit např. $\sizeof{\N}=\sizeof{\R}=\infty$. To by však nebylo konzistentní s definicí, že dvě množiny mají stejnou mohutnost, když mezi nimi existuje bijekce, protože podle věty \ref{thm:N_a_R} víme, že $\R\preccurlyeq\N$. Na mohutnost množiny lze nahlížet i~trochu abstraktněji.
\medskip

Pro lehčí pochopení se na chvíli přesuňme ke geometrii. Uvážíme-li relaci "být rovnoběžný s", tj. "$\parallel$" na množině všech přímek v~rovině, jaké vlastnosti splňuje?
\begin{itemize}
    \item \textbf{Reflexivita}. Každá přímka je rovnoběžná sama se sebou, tj. pro přímku $p$ platí $p\parallel p$.
    \item \textbf{Symetrie}. Platí-li $p\parallel q$, pak platí i~$q\parallel p$.
    \item \textbf{Tranzitivita}. Je-li přímka $p$ rovnoběžná s přímkou $q$ a~zároveň $q$ je rovnoběžná s přímkou $r$, pak určitě $p\parallel r$.
\end{itemize}
Tedy "$\parallel$" je relací ekvivalence. Na základě tohoto poznatku, máme-li libovolnou přímku $p$, jaké přímky obsahuje třída ekvivalence $[p]_\parallel$? Budou to právě takové přímky, které jsou rovnoběžné s přímkou $p$. Jak bychom definovali \emph{směr} přímky? Zkuste se zamyslet, než budete pokračovat.\par
Zkusme se ale zpětně zaměřit na třídy ekvivalence "$\parallel$". Uvážíme-li libovolnou z nich, pak přímky jí náležící mají vždy shodný směr. To znamená, že výběrem kterékoliv třídy ekvivalence je směr jednoznačně určen podle náležících přímek. Tedy celkově: za směr přímky $p$ prohlásíme třídu ekvivalence $[p]_\parallel$.
\medskip

Tato úvaha nám zde velmi pomůže. Ačkoliv sice netušíme, co je to mohutnost množiny, přesto dokážeme určit (v principu), zda libovolné množiny $X$ a~$Y$ mají stejnou mohutnost, nebo nemají.
\begin{lemma}
    Ekvipotence $\approx$ je relací ekvivalence\footnote{Zde se jedná o~tzv. \emph{třídovou relaci} na tzv. \emph{univerzální třídě} (často označované $\mathbb{V}$), tedy "souhrnu" všech množin. Tento "souhrn" nemůže být množinou, neboť bychom tak došli ke sporu v~\ZF{}. Třídy představují v~teorii množin "nadstavbu" termínu množina. Obecně platí, že každá množina je třída, ale ne každá třída je množina. Pro hlubší pochopení doporučuji knihu \cite{BalcarStepanek1986}, str. 45--50}.
\end{lemma}
\begin{proof}
    Z definice relace ekvivalence stačí ověřit, že $\approx$ je reflexivní, symetrická a~tranzitivní. Mějme libovolné množiny $X,Y,Z$.
    \begin{itemize}
        \item \textbf{Reflexivita}. Jistě platí $X\approx X$. Za bijektivní zobrazení stačí zvolit identitu $1_X$.
        \item \textbf{Symetrie}. Pokud $X\approx Y$, pak existuje bijekce $\map{f}{X}{Y}$. Protože $f$ je bijekce, existuje inverzní zobrazení $\map{f^{-1}}{Y}{X}$, které je též bijekcí. Tzn. platí i~$Y\approx X$.
        \item \textbf{Tranzitivita}. Nechť $X\approx Y$ a~zároveň $Y\approx Z$. Pišme $\map{f}{X}{Y}$ a~$\map{g}{Y}{Z}$. Definujme zobrazení $h=g\circ f$. Podle tvrzení \ref{prop:vlastnosti_skladani_zobrazeni} je $\map{h}{X}{Z}$ bijekce a~tedy $X\approx Z$.
    \end{itemize}
    Tedy $\approx$ je relací ekvivalence.
\end{proof}
Pro libovolnou množinu $X$ tak třída ekvivalence $[X]_\approx$ obsahuje všechny množiny, které mají stejnou mohutnost jako $X$. Tzn.
\begin{equation*}
    \forall Y\in [X]_\approx: Y\approx X.
\end{equation*}
Víme tak, že každá z množin v~libovolné třídě ekvivalence má stejnou mohutnost. Tuto třídu bychom pak mohli prohlásit za její mohutnost. V případě konečných množin, kde za mohutnost považujeme přirozené číslo, se jedná o~alternativní pohled.\par
Mohutnosti představují v~teorii množin tzv. \emph{kardinální čísla}, která lze definovat způsobem popsaným výše. Jedná se tak o~zobecnění myšlenky počtu prvků u konečných množin. Podobně jako na přirozených číslech, i~na kardinálních číslech lze definovat smysluplnou aritmetiku.