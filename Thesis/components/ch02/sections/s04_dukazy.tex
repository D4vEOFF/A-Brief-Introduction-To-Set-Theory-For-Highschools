\section{Důkazy}\label{sec:dukazy}
V matematice se lze setkat s celou řadou různých tvrzení. Od primitivních, jejichž platnost je zřejmá až po složitější, nad jejich platností je třeba se více zamyslet. Čtenář se nejspíše zatím spíše setkával s matematikou, která zahrnovala užívání jistých postupů. Např. zjednodušování algebraických výrazů, řešení soustav rovnic, ověřování trigonometrických identit, aj. Avšak hodně postupů v matematice je založeno na již známých výsledcích, o nichž bylo dokázáno, že jsou pravdivé. Pokud ovšem máme dokázat určité tvrzení, je třeba, aby bylo naše zdůvodnění jednoznačné a logicky správné. V této sekci se proto podíváme na důkazové techniky používané v matematice, které budeme dále v textu využívat.\par
Matematická tvrzení jsou často různě klasifikována v závislosti na jejich povaze. Základními typy jsou tyto. 
\begin{itemize}
    \item \emph{Axióm}. Tvrzení, které implicitně považujeme za pravdivé a nedokazujeme jej. S axiomatikou jsme se již částečně seznámili v historické předmluvě (viz \ref{subsec:tm_soucasnost}).
    \item \emph{Věta}. Matematické tvrzení, jehož pravdivost můžeme ověřit důkazem.
    \item \emph{Lemma}. Pomocné tvrzení, které běžně využíváme pro důkaz jiného (typicky složitějšího) tvrzení.
    \item \emph{Důsledek}. Tvrzení, které je přímým důsledkem jiného tvrzení.
\end{itemize}
Čistě formálně však mezi \textbf{větou}, \textbf{lemmatem} a \textbf{důsledkem} není žádný rozdíl.

\subsection{Důkaz přímý}\label{subsec:dukaz_primy}
Jedná se o asi nejjednodušší typ důkazu. Často jsou matematická tvrzení formulována jako implikace, tzn. "Jestliže platí $A$, pak platí $B$.". Konkrétně, např. "Je-li $x<0$, pak $x^2>0$.".\par
Myšlenka důkazu je taková, že začínáme od předpokladu $A$, z něhož dále odvozujeme dílčí tvrzení tak dlouho, až dojdeme k požadovanému závěru $B$. Symbolicky, pokud si označíme dílčí tvrzení v důkazu $X_1, X_2, \ldots, X_n$, pak vlastně dokazujeme výrokovou formuli
\begin{equation}\label{eq:primy_dukaz_formule}
    (A \implies X_1) \land (X_1 \implies X_2) \land \cdots \land (X_{n-1}\implies X_n) \land (X_n \implies B)\; .
\end{equation}
V tomto procesu dokazování se využívá tautologie \ref{item:tautologie_4} z věty \ref{thm:vyznamne_tautologie}
\begin{equation*}
    (A \implies B) \land (B \implies C) \iff (A \implies C)\; .
\end{equation*}
Z tohoto faktu vyplývá, že pokud je každá z dílčích implikací pravdivá, pak je nutně pravdivá i implikace $A \implies B$, kterou jsme chtěli dokázat\footnote{Výrokové proměnné lze v konkrétním případě nahradit příslušnými predikáty.}. Podívejme se na příklad podobný příklad z úvodu.

\begin{assertion}
    Nechť $x\in\R$. Je-li $x<0$, pak $x^2+1>0$.
\end{assertion}
\begin{proof}
    Předpokladem našeho tvrzení je $x\in\R \land x<0$. Víme, že pro každé reálné číslo $x$ platí, že $x^2\geq 0$. Z toho již plyne
    \begin{equation*}
        0 \leq x^2 < x^2+1 \implies x^2+1>0\; .
    \end{equation*}
\end{proof}
Posloupnost dokázaných implikací bychom mohli podle \eqref{eq:primy_dukaz_formule} zapsat nyní jako
\begin{equation*}
    (x\in\R \land x<0 \implies x^2\geq 0) \land (x^2\geq 0 \implies x^2+1>0)\; ,
\end{equation*}
a tedy jsme dokázali i implikaci v původním tvrzení $x<0 \implies x^2+1>0$ .

Všimněte si, že důkaz našeho tvrzení ve skutečnosti vůbec nezávisel na předpokladu, že $x<0$, neboť jak jsme správně konstatovali, tak platí
\begin{equation*}
    \forall x\in\R : x^2>0\; .
\end{equation*}
Tj. dokázané tvrzení lze zobecnit klidně zobecnit.
\begin{assertion*}
    Nechť $x\in\R$. Pak platí $x^2+1>0$ .
\end{assertion*}
\begin{remark}
    Též bychom mohli psát "Je-li $x\in\R$, pak $x^2+1>0$."; význam by byl stejný.
\end{remark}

\begin{assertion}
    Nechť $n\in\N$ je liché číslo. Pak $3n+7$ je sudé číslo.
\end{assertion}
\begin{proof}
    Začneme opět u předpokladu, že $n\in\N$ je liché číslo. To znamená, že
    \begin{equation*}
        \exists k\in\N : n=2k+1\; .
    \end{equation*}
    Po dosazení obdržíme
    \begin{equation*}
        3(2k+1)+7=6k+3+7=6k+10=2(3k+5)\; .
    \end{equation*}
    Protože $3k+5$ je přirozené číslo, pak $3n+7$ je dělitelné dvěma a je tedy sudé, což jsme chtěli dokázat.
\end{proof}
\begin{assertion}[AG nerovnost]
    Pro $a,b\in\R_0^+$ platí
    \begin{equation*}
        \sqrt{ab}\leq\dfrac{a+b}{2}\; .
    \end{equation*}
\end{assertion}
\begin{proof}
    Při důkazu tohoto tvrzení vyjdeme z jednoduchého pozorování:
    \begin{equation*}
        (\sqrt{a}+\sqrt{b})^2\geq 0\; .
    \end{equation*}
    Nyní stačí výraz upravit a dostaneme požadovanou nerovnost.
    \begin{equation*}
        (\sqrt{a}+\sqrt{b})^2 = a+2\sqrt{ab}+b\geq 0 \implies \sqrt{ab}\leq \dfrac{a+b}{2}\; .
    \end{equation*}
\end{proof}
\begin{assertion}
    Pro $\forall x,y\in\R$ platí
    \begin{equation*}
        x<y \implies x < \dfrac{x+y}{2} < y\; .
    \end{equation*}
\end{assertion}
\begin{proof}
    Zde je třeba si všimnout "dvojité" nerovnosti v dokazovaném tvrzení. To nám již napovídá, že ve skutečnosti musíme dokázat 2 dílčí tvrzení, konkrétně
    \begin{equation*}
        x < \dfrac{x+y}{2}\;\;\;\text{a}\;\;\;\dfrac{x+y}{2} < y\; .
    \end{equation*}
    Při důkazu obou částí vyjdeme opět z předpokladu. Tedy mějme libovolná čísla $x,y\in\R$ taková, že $x<y$. Pak jistě platí
    \begin{equation*}
        x+x<x+y \implies 2x<x+y \implies x<\dfrac{x+y}{2}\; .
    \end{equation*}
    Tím jsme dokázali první nerovnost. Platnost druhé dokážeme analogicky:
    \begin{equation*}
        x+y<y+y \implies x+y<2y \implies \dfrac{x+y}{2}<y\; .
    \end{equation*}
\end{proof}
(Převzato z \cite{ChartrandPolimeniZhang2014} a \cite{MatematickaLogikaUK2010}.)
V sekci \ref{sec:mnoziny_a_cisla} jsme si ukázali \emph{de Morganovy vzorce} pro množiny. Nyní si ukážeme důkaz jejich platnosti. Zároveň si tak ukážeme příklad "množinového" důkazu.
\begin{assertion}[de Morganovy vzorce]
    Nechť $A,X_1,X_2,\ldots,X_n$ jsou libovolné množiny. Pak platí
    \begin{enumerate}[label=(\roman*)]
        \item\label{item:de_morgan_sets_1} $\displaystyle A \setminus \left(\bigcup\limits_{i=1}^{n}{X_i}\right)=\bigcap\limits_{i=1}^{n}{(A \setminus X_i)}\; ,$
        \item\label{item:de_morgan_sets_2} $\displaystyle A \setminus \left(\bigcap\limits_{i=1}^{n}{X_i}\right)=\bigcup\limits_{i=1}^{n}{(A \setminus X_i)}\; .$
    \end{enumerate}
\end{assertion}
\begin{proof}
    Nejdříve si uvědomme, co vlastně říkají dané rovnosti. Vyjadřují, že výsledné množiny jsou stejné. Kdy toto platí? Právě když dané množiny obsahují stejné prvky, tj.
    \begin{equation*}
        A=B \iff \forall x : x\in A \iff x\in B\; .     
    \end{equation*}
    Myšlenka našeho důkazu bude tedy taková, že uvážíme libovolný prvek $x$ náležící pravé straně rovnosti a ukážeme, že rovněž náleží i straně druhé.\par
    Ukážeme si platnost \ref{item:de_morgan_sets_1}, avšak důkaz \ref{item:de_morgan_sets_2} je zcela analogický. Tedy budiž dáno $x\in A \setminus \left(\bigcup_{i=1}^{n}{X_i}\right)$. Z definice rozdílu množin (viz \ref{def:zakladni_mnozinove_operace}) tedy musí platit
    \begin{equation*}
        x \in A \setminus \left(\bigcup\limits_{i=1}^{n}{X_i}\right) \iff x\in A \land x\notin \bigcup\limits_{i=1}^{n}{X_i}\; .
    \end{equation*}
    Protože však prvek $x$ nenáleží sjednocení množin $X_1,\ldots,X_n$, pak nenáleží žádné z nich:
    \begin{equation*}
        x\notin \bigcup\limits_{i=1}^{n}{X_i} \iff \forall i\in\set{1,\ldots,n} : x\notin X_i\; .
    \end{equation*}
    Celkově tedy nyní víme, že
    \begin{equation}\label{eq:de_morgan_leva_strana}
        x\in A \land \forall i\in\set{1,\ldots,n} : x\notin X_i
    \end{equation}
    Nyní se podívejme na pravou stranu zkoumané rovnosti. Z \eqref{eq:de_morgan_leva_strana} již plyne
    \begin{align*}
        x\in A \land \forall i\in\set{1,\ldots,n} : x\notin X_i &\iff \forall i\in\set{1,\ldots,n} : x\in (A \setminus X_i)\\
        &\iff x\in\bigcap\limits_{i=1}^{n}{(A \setminus X_i)}\; ,
    \end{align*}
    a celkově tedy jsme dokázali implikaci
    \begin{equation*}
        x \in A \setminus \left(\bigcup\limits_{i=1}^{n}{X_i}\right) \implies x\in\bigcap\limits_{i=1}^{n}{(A \setminus X_i)}\; .
    \end{equation*}
    Prvek $x$ jsme však volili libovolně a náš argument výše je tak univerzální (platí pro všechny prvky). Zároveň si však všimněme, že všechna odvozená dílčí tvrzení jsou ekvivalencemi a tedy platí i opačná implikace
    \begin{equation*}
        x \in A \setminus \left(\bigcup\limits_{i=1}^{n}{X_i}\right) \impliedby x\in\bigcap\limits_{i=1}^{n}{(A \setminus X_i)}\; .
    \end{equation*}

    Z toho dostáváme platnost původní rovnosti v \ref{item:de_morgan_sets_1}. 
\end{proof}
Obecně pokud je tvrzení takové, že platí pro všechny objekty z jisté množiny, pak bývá právě častým způsobem důkazu, že se platnost ukáže obecně pro kterýkoliv objekt z dané množiny.

\subsection{Důkaz nepřímý}\label{subsec:dukaz_neprimy}
\subsection{Důkaz sporem}\label{subsec:dukaz_sporem}
\subsection{Důkaz matematickou indukcí}\label{subsec:dukaz_indukci}

\todo{Doplnit cvičení}