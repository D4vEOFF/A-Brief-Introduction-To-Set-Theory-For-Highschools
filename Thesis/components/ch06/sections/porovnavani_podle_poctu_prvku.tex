\section{Porovnávání podle počtu prvků}\label{sec:porovnavani_podle_poctu_prvku}
Nekonečné množiny mají tedy dost pozoruhodné vlastnosti. V souvislosti s předešlou úvahou by nás tak mohlo napadnout, že při plně obsazeném hotelu je hostů "stejně mnoho" jako pokojů. Ovšem z případu \ref{item:novy_host} jsme mohli vidět, že po přesunutí hostů do vedlejšího pokoje byly pak již obsazeny "pouze" pokoje $2,3,\dots$ a první tak zůstal volný. Pokud by toto byla výchozí situace, mohlo by se nám tak zdát, že hostů je méně, neboť v prvním pokoji žádný není. Avšak jediné, co se stalo je, že hosté změnili svůj, čímž se celkem přirozeně nemohl porušit jejich počet. Změnil se tedy počet hostů a pokojů, nebo jich je stejně mnoho?\par
Náš náhled se pochopitelně odráží od porovnávání velikostí konečných množin. U těch toto nečinní žádný problém, stačí spočítat jejich prvky. U nekonečných množin již prvky "spočítat" nemůžeme. Existuje však ještě jeden způsob, který ve skutečnosti vůbec nevyžaduje schopnost počítání.
\medskip

\noindent\textbf{O domorodém kmenu a slepicích}. \textit{Představme si, že jistý domorodý kmen hledá nového náčelníka. Po několika vyřazovacích kolech zbyli poslední dva kandidáti. Ostatní členové kmenu rozhodli, že náčelníkem se stane ten z kandidátů, který vlastní více slepic. Tento kmen je však matematicky velmi primitivní a jeho příslušníci umí počítat pouze do pěti. Oba dva kandidáti však mají ostře více jak pět slepic. Kmen se tedy rozhodl postupovat takto: každý z kandidátů přinese vždy jednu svojí slepici a tímto způsobem pokračují, dokud jednomu z nich slepice nedojdou. Ten, kterému dříve dojdou slepice, prohraje a druhý se tak stane náčelníkem.} \cite{Pick2019}
\medskip

Z tohoto (možná lehce humorného) příkladu je již nejspíše jasné, jak lze množiny porovnávat. Pokud mají dvě množiny stejný počet prvků, pak lze jednotlivé prvky "spárovat". V matematické řeči náčelníci pouze sestrojují zobrazení z jedné množiny slepic do druhé. Toto dává jistě smysl v kontextu konečných množin. Pokud mají množiny stejně prvků, pak mezi nimi existuje bijekce (viz obrázek \ref{fig:bijekce_konecne_mnoziny}).
\begin{figure}[h]
    \centering
    \includegraphics[scale=\normalipe]{ch06_bijekce_konecne_mnoziny.pdf}
    \caption{Bijekce mezi konečnými množinami $A,B$ se stejným počtem prvků.}
    \label{fig:bijekce_konecne_mnoziny}
\end{figure}
Stejným způsobem však můžeme porovnávat i nekonečné množiny! Zkusme si vzít např. množinu všech kladných sudých čísel $S=\set{2k\admid k\in\N}$ a přirozená čísla $\N$. Selský rozum by nám řekl, že sudých čísel musí být "méně" než přirozených, neboť $S$ nenáleží všechna lichá přirozená čísla, zatímco množině $\N$ náleží. Z jistého pohledu tomu tak může být, ale z "perspektivy" bijekce nikoliv. Pokud uvážíme zobrazení $\map{f}{\N}{S}$, kde pro $n\in\N$ je $f(n)=2n$, můžeme se snadno přesvědčit, že $f$ je bijekcí\footnote{Skutečně, inverzní zobrazení je $\map{f^{-1}}{S}{\N}$, kde $f^{-1}(n)=n/2$.}.
\begin{figure}[h]
    \centering
    \includegraphics[scale=\normalipe]{ch06_bijekce_suda_a_prirozena_cisla.pdf}
    \caption{Bijekce mezi množinami $S$ a $\N$.}
    \label{fig:bijekce_suda_a_prirozena_cisla}
\end{figure}
\begin{example}\label{ex:bijekce_prirozena_cisla}
    Další příklady bijekcí mezi $\N$ a jinými množinami.
    \begin{enumerate}[label=(\roman*)]
        \item Bijekce mezi $\N$ a kladnými lichými čísly: $\map{f_1}{\N}{\set{2k-1\admid k\in\N}}$, kde $f_1(n)=2n-1$.
        \item Bijekce mezi $\N$ a druhými mocninami: $\map{f_2}{\N}{\set{n^2\admid n\in\N}}$, kde $f_2(n)=n^2$.
        \item Bijekce mezi $\N$ a prvočísly.
    \end{enumerate}
\end{example}
Všimněme si, že všechny (zatím) uvažované množiny byly všechny vlastní podmnožiny $\N$. Toto však u konečných množin provést nelze, neboť libovolná vlastní podmnožina je vždy "menší" než původní množina (velikosti těchto množin se pak liší v počtu "chybějících" prvků). V případě nekonečných množin toto však není žádnou překážkou. Archimédův logický axiom, že \emph{celek je větší než část} tak skutečně patří pouze do oblasti konečných množin. Toto se zdá být jako pěkná charakteristika odlišující konečné a nekonečné množiny.\par
Do této chvíle jsme termín "konečná" a "nekonečná" množina chápali intuitivně bez formální definice. Vzhledem k jasnosti těchto termínů v použitých kontextech nejspíše nebyl problém s jejich chápáním. Avšak díky výše zmíněnému máme již dostupný nástroj, jak definovat nekonečnou množinu.
\begin{definition}[Nekonečnost množiny]\label{def:nekonecnost_mnoziny}
    Množinu libovolnou množinu $X$ nazveme \emph{nekonečnou}, pokud existuje množina $X^\prime\subset X$ taková, že existuje bijektivní zobrazení $\map{f}{X}{X^\prime}$. 
\end{definition}
Jednoduše, množina je nekonečná, pokud existuje bijekce množiny na některou její vlastní podmnožinu. Nabízí se otázka, jak bychom mohli definovat konečnou množinu. Vzhledem k formalizaci "nekonečná množina" v definici \ref{def:nekonecnost_mnoziny} bychom za konečnou množinu mohli jednoduše prohlásit množinu, která není nekonečná. Avšak tento termín můžeme zavést i více přirozeně. Pokud o množině řekneme, že je "konečná", pak tím intuitivně rozumíme, že její počet prvků odpovídá nějakému přirozenému číslu, tj. $\sizeof{A}=n$, kde $n\in\N_0$. Pokud si však vzpomeneme, přirozená čísla jsme si zaváděli pomocí množin, což znamená že se opět jedná o množiny. Za konečnou množinu tak můžeme prohlásit množinu, u níž \textbf{existuje bijekce na nějaké přirozené číslo}.\par
Pokud se nebudeme omezovat pouze na přirozená čísla, příklady bijekcí najdeme i u reálných čísel $\R$.
\begin{example}\label{ex:bijekce_realna_cisla}
    Ukázky bijekcí mezi množinou $\R$ některými jejími vlastními podmnožinami.
    \begin{enumerate}[label=(\roman*)]
        \item\label{item:funkce_arctg} Bijekce mezi $\R$ a intervalem $\displaystyle\left(-\frac{\pi}{2},\frac{\pi}{2}\right)$: $\displaystyle\map{g_1}{\R}{\left(-\frac{\pi}{2},\frac{\pi}{2}\right)}$, kde $g_1(x)=\arctg{x}$.
        \begin{figure}[H]
            \centering
            \begin{tikzpicture}[line cap=round,line join=round,>=triangle 45,x=1.0cm,y=1.0cm]
    \begin{axis}[
    x=1.0cm,y=1.0cm,
    axis lines=middle,
    xmin=-6.5,
    xmax=6.5,
    ymin=-1.7,
    ymax=1.7,
    xtick={-6.0,-5.0,...,6.0},
    ytick={-1.0,0.0,...,1.0},]
    \clip(-6.5,-1.7) rectangle (6.5,1.7);
    \draw[line width=1.pt] (-6.5,-1.4181469983996315) -- (-6.5,-1.4181469983996315);
    \draw[line width=1.pt] (-6.5,-1.4181469983996315) -- (-6.4675,-1.4173918650878474);
    \draw[line width=1.pt] (-6.4675,-1.4173918650878474) -- (-6.4350000000000005,-1.4166292831444403);
    \draw[line width=1.pt] (-6.4350000000000005,-1.4166292831444403) -- (-6.402500000000001,-1.415859142713462);
    \draw[line width=1.pt] (-6.402500000000001,-1.415859142713462) -- (-6.370000000000001,-1.4150813317862707);
    \draw[line width=1.pt] (-6.370000000000001,-1.4150813317862707) -- (-6.337500000000001,-1.414295736149008);
    \draw[line width=1.pt] (-6.337500000000001,-1.414295736149008) -- (-6.3050000000000015,-1.4135022393285461);
    \draw[line width=1.pt] (-6.3050000000000015,-1.4135022393285461) -- (-6.272500000000002,-1.4127007225368495);
    \draw[line width=1.pt] (-6.272500000000002,-1.4127007225368495) -- (-6.240000000000002,-1.4118910646137035);
    \draw[line width=1.pt] (-6.240000000000002,-1.4118910646137035) -- (-6.207500000000002,-1.4110731419677478);
    \draw[line width=1.pt] (-6.207500000000002,-1.4110731419677478) -- (-6.1750000000000025,-1.4102468285157603);
    \draw[line width=1.pt] (-6.1750000000000025,-1.4102468285157603) -- (-6.142500000000003,-1.409411995620132);
    \draw[line width=1.pt] (-6.142500000000003,-1.409411995620132) -- (-6.110000000000003,-1.4085685120244655);
    \draw[line width=1.pt] (-6.110000000000003,-1.4085685120244655) -- (-6.077500000000003,-1.4077162437872364);
    \draw[line width=1.pt] (-6.077500000000003,-1.4077162437872364) -- (-6.0450000000000035,-1.4068550542134495);
    \draw[line width=1.pt] (-6.0450000000000035,-1.4068550542134495) -- (-6.012500000000004,-1.4059848037842175);
    \draw[line width=1.pt] (-6.012500000000004,-1.4059848037842175) -- (-5.980000000000004,-1.4051053500841904);
    \draw[line width=1.pt] (-5.980000000000004,-1.4051053500841904) -- (-5.947500000000004,-1.4042165477267579);
    \draw[line width=1.pt] (-5.947500000000004,-1.4042165477267579) -- (-5.9150000000000045,-1.403318248276949);
    \draw[line width=1.pt] (-5.9150000000000045,-1.403318248276949) -- (-5.882500000000005,-1.4024103001719435);
    \draw[line width=1.pt] (-5.882500000000005,-1.4024103001719435) -- (-5.850000000000005,-1.4014925486391092);
    \draw[line width=1.pt] (-5.850000000000005,-1.4014925486391092) -- (-5.817500000000005,-1.4005648356114806);
    \draw[line width=1.pt] (-5.817500000000005,-1.4005648356114806) -- (-5.7850000000000055,-1.3996269996405784);
    \draw[line width=1.pt] (-5.7850000000000055,-1.3996269996405784) -- (-5.752500000000006,-1.3986788758064812);
    \draw[line width=1.pt] (-5.752500000000006,-1.3986788758064812) -- (-5.720000000000006,-1.3977202956250439);
    \draw[line width=1.pt] (-5.720000000000006,-1.3977202956250439) -- (-5.687500000000006,-1.396751086952158);
    \draw[line width=1.pt] (-5.687500000000006,-1.396751086952158) -- (-5.6550000000000065,-1.3957710738849496);
    \draw[line width=1.pt] (-5.6550000000000065,-1.3957710738849496) -- (-5.622500000000007,-1.394780076659793);
    \draw[line width=1.pt] (-5.622500000000007,-1.394780076659793) -- (-5.590000000000007,-1.3937779115470315);
    \draw[line width=1.pt] (-5.590000000000007,-1.3937779115470315) -- (-5.557500000000007,-1.3927643907422729);
    \draw[line width=1.pt] (-5.557500000000007,-1.3927643907422729) -- (-5.5250000000000075,-1.3917393222541365);
    \draw[line width=1.pt] (-5.5250000000000075,-1.3917393222541365) -- (-5.492500000000008,-1.390702509788316);
    \draw[line width=1.pt] (-5.492500000000008,-1.390702509788316) -- (-5.460000000000008,-1.389653752627819);
    \draw[line width=1.pt] (-5.460000000000008,-1.389653752627819) -- (-5.427500000000008,-1.3885928455092331);
    \draw[line width=1.pt] (-5.427500000000008,-1.3885928455092331) -- (-5.3950000000000085,-1.3875195784948755);
    \draw[line width=1.pt] (-5.3950000000000085,-1.3875195784948755) -- (-5.362500000000009,-1.3864337368406556);
    \draw[line width=1.pt] (-5.362500000000009,-1.3864337368406556) -- (-5.330000000000009,-1.3853351008594963);
    \draw[line width=1.pt] (-5.330000000000009,-1.3853351008594963) -- (-5.297500000000009,-1.3842234457801315);
    \draw[line width=1.pt] (-5.297500000000009,-1.3842234457801315) -- (-5.2650000000000095,-1.3830985416011055);
    \draw[line width=1.pt] (-5.2650000000000095,-1.3830985416011055) -- (-5.23250000000001,-1.3819601529397847);
    \draw[line width=1.pt] (-5.23250000000001,-1.3819601529397847) -- (-5.20000000000001,-1.3808080388761812);
    \draw[line width=1.pt] (-5.20000000000001,-1.3808080388761812) -- (-5.16750000000001,-1.379641952791387);
    \draw[line width=1.pt] (-5.16750000000001,-1.379641952791387) -- (-5.1350000000000104,-1.3784616422004017);
    \draw[line width=1.pt] (-5.1350000000000104,-1.3784616422004017) -- (-5.102500000000011,-1.3772668485791295);
    \draw[line width=1.pt] (-5.102500000000011,-1.3772668485791295) -- (-5.070000000000011,-1.3760573071853115);
    \draw[line width=1.pt] (-5.070000000000011,-1.3760573071853115) -- (-5.037500000000011,-1.3748327468731467);
    \draw[line width=1.pt] (-5.037500000000011,-1.3748327468731467) -- (-5.005000000000011,-1.373592889901346);
    \draw[line width=1.pt] (-5.005000000000011,-1.373592889901346) -- (-4.972500000000012,-1.372337451734352);
    \draw[line width=1.pt] (-4.972500000000012,-1.372337451734352) -- (-4.940000000000012,-1.3710661408364413);
    \draw[line width=1.pt] (-4.940000000000012,-1.3710661408364413) -- (-4.907500000000012,-1.3697786584584186);
    \draw[line width=1.pt] (-4.907500000000012,-1.3697786584584186) -- (-4.875000000000012,-1.3684746984165934);
    \draw[line width=1.pt] (-4.875000000000012,-1.3684746984165934) -- (-4.842500000000013,-1.367153946863719);
    \draw[line width=1.pt] (-4.842500000000013,-1.367153946863719) -- (-4.810000000000013,-1.3658160820515561);
    \draw[line width=1.pt] (-4.810000000000013,-1.3658160820515561) -- (-4.777500000000013,-1.3644607740847097);
    \draw[line width=1.pt] (-4.777500000000013,-1.3644607740847097) -- (-4.745000000000013,-1.3630876846653692);
    \draw[line width=1.pt] (-4.745000000000013,-1.3630876846653692) -- (-4.712500000000014,-1.3616964668285658);
    \draw[line width=1.pt] (-4.712500000000014,-1.3616964668285658) -- (-4.680000000000014,-1.3602867646675412);
    \draw[line width=1.pt] (-4.680000000000014,-1.3602867646675412) -- (-4.647500000000014,-1.3588582130488036);
    \draw[line width=1.pt] (-4.647500000000014,-1.3588582130488036) -- (-4.615000000000014,-1.3574104373164266);
    \draw[line width=1.pt] (-4.615000000000014,-1.3574104373164266) -- (-4.582500000000015,-1.3559430529851235);
    \draw[line width=1.pt] (-4.582500000000015,-1.3559430529851235) -- (-4.550000000000015,-1.3544556654216067);
    \draw[line width=1.pt] (-4.550000000000015,-1.3544556654216067) -- (-4.517500000000015,-1.352947869513723);
    \draw[line width=1.pt] (-4.517500000000015,-1.352947869513723) -- (-4.485000000000015,-1.3514192493268211);
    \draw[line width=1.pt] (-4.485000000000015,-1.3514192493268211) -- (-4.452500000000016,-1.3498693777467925);
    \draw[line width=1.pt] (-4.452500000000016,-1.3498693777467925) -- (-4.420000000000016,-1.348297816109187);
    \draw[line width=1.pt] (-4.420000000000016,-1.348297816109187) -- (-4.387500000000016,-1.3467041138137854);
    \draw[line width=1.pt] (-4.387500000000016,-1.3467041138137854) -- (-4.355000000000016,-1.3450878079239736);
    \draw[line width=1.pt] (-4.355000000000016,-1.3450878079239736) -- (-4.322500000000017,-1.343448422750233);
    \draw[line width=1.pt] (-4.322500000000017,-1.343448422750233) -- (-4.290000000000017,-1.341785469417027);
    \draw[line width=1.pt] (-4.290000000000017,-1.341785469417027) -- (-4.257500000000017,-1.340098445412325);
    \draw[line width=1.pt] (-4.257500000000017,-1.340098445412325) -- (-4.225000000000017,-1.3383868341189702);
    \draw[line width=1.pt] (-4.225000000000017,-1.3383868341189702) -- (-4.192500000000018,-1.336650104327055);
    \draw[line width=1.pt] (-4.192500000000018,-1.336650104327055) -- (-4.160000000000018,-1.334887709726426);
    \draw[line width=1.pt] (-4.160000000000018,-1.334887709726426) -- (-4.127500000000018,-1.3330990883783898);
    \draw[line width=1.pt] (-4.127500000000018,-1.3330990883783898) -- (-4.095000000000018,-1.331283662165657);
    \draw[line width=1.pt] (-4.095000000000018,-1.331283662165657) -- (-4.062500000000019,-1.3294408362194938);
    \draw[line width=1.pt] (-4.062500000000019,-1.3294408362194938) -- (-4.030000000000019,-1.3275699983230118);
    \draw[line width=1.pt] (-4.030000000000019,-1.3275699983230118) -- (-3.9975000000000187,-1.3256705182894604);
    \draw[line width=1.pt] (-3.9975000000000187,-1.3256705182894604) -- (-3.9650000000000185,-1.3237417473143374);
    \draw[line width=1.pt] (-3.9650000000000185,-1.3237417473143374) -- (-3.9325000000000183,-1.3217830173000569);
    \draw[line width=1.pt] (-3.9325000000000183,-1.3217830173000569) -- (-3.900000000000018,-1.319793640151863);
    \draw[line width=1.pt] (-3.900000000000018,-1.319793640151863) -- (-3.867500000000018,-1.3177729070435944);
    \draw[line width=1.pt] (-3.867500000000018,-1.3177729070435944) -- (-3.8350000000000177,-1.31572008765184);
    \draw[line width=1.pt] (-3.8350000000000177,-1.31572008765184) -- (-3.8025000000000175,-1.3136344293569449);
    \draw[line width=1.pt] (-3.8025000000000175,-1.3136344293569449) -- (-3.7700000000000173,-1.3115151564092409);
    \draw[line width=1.pt] (-3.7700000000000173,-1.3115151564092409) -- (-3.737500000000017,-1.309361469058795);
    \draw[line width=1.pt] (-3.737500000000017,-1.309361469058795) -- (-3.705000000000017,-1.3071725426468694);
    \draw[line width=1.pt] (-3.705000000000017,-1.3071725426468694) -- (-3.6725000000000168,-1.3049475266571977);
    \draw[line width=1.pt] (-3.6725000000000168,-1.3049475266571977) -- (-3.6400000000000166,-1.3026855437250706);
    \draw[line width=1.pt] (-3.6400000000000166,-1.3026855437250706) -- (-3.6075000000000164,-1.300385688602126);
    \draw[line width=1.pt] (-3.6075000000000164,-1.300385688602126) -- (-3.575000000000016,-1.2980470270746127);
    \draw[line width=1.pt] (-3.575000000000016,-1.2980470270746127) -- (-3.542500000000016,-1.2956685948327833);
    \draw[line width=1.pt] (-3.542500000000016,-1.2956685948327833) -- (-3.5100000000000158,-1.293249396288946);
    \draw[line width=1.pt] (-3.5100000000000158,-1.293249396288946) -- (-3.4775000000000156,-1.290788403341557);
    \draw[line width=1.pt] (-3.4775000000000156,-1.290788403341557) -- (-3.4450000000000154,-1.2882845540826133);
    \draw[line width=1.pt] (-3.4450000000000154,-1.2882845540826133) -- (-3.412500000000015,-1.2857367514454303);
    \draw[line width=1.pt] (-3.412500000000015,-1.2857367514454303) -- (-3.380000000000015,-1.283143861789756);
    \draw[line width=1.pt] (-3.380000000000015,-1.283143861789756) -- (-3.347500000000015,-1.2805047134209833);
    \draw[line width=1.pt] (-3.347500000000015,-1.2805047134209833) -- (-3.3150000000000146,-1.2778180950400593);
    \draw[line width=1.pt] (-3.3150000000000146,-1.2778180950400593) -- (-3.2825000000000144,-1.2750827541204997);
    \draw[line width=1.pt] (-3.2825000000000144,-1.2750827541204997) -- (-3.250000000000014,-1.2722973952087187);
    \draw[line width=1.pt] (-3.250000000000014,-1.2722973952087187) -- (-3.217500000000014,-1.2694606781436801);
    \draw[line width=1.pt] (-3.217500000000014,-1.2694606781436801) -- (-3.185000000000014,-1.2665712161916605);
    \draw[line width=1.pt] (-3.185000000000014,-1.2665712161916605) -- (-3.1525000000000136,-1.2636275740916783);
    \draw[line width=1.pt] (-3.1525000000000136,-1.2636275740916783) -- (-3.1200000000000134,-1.260628266006912);
    \draw[line width=1.pt] (-3.1200000000000134,-1.260628266006912) -- (-3.0875000000000132,-1.2575717533771726);
    \draw[line width=1.pt] (-3.0875000000000132,-1.2575717533771726) -- (-3.055000000000013,-1.2544564426672329);
    \draw[line width=1.pt] (-3.055000000000013,-1.2544564426672329) -- (-3.022500000000013,-1.2512806830055363);
    \draw[line width=1.pt] (-3.022500000000013,-1.2512806830055363) -- (-2.9900000000000126,-1.2480427637075255);
    \draw[line width=1.pt] (-2.9900000000000126,-1.2480427637075255) -- (-2.9575000000000125,-1.2447409116775185);
    \draw[line width=1.pt] (-2.9575000000000125,-1.2447409116775185) -- (-2.9250000000000123,-1.2413732886827504);
    \draw[line width=1.pt] (-2.9250000000000123,-1.2413732886827504) -- (-2.892500000000012,-1.2379379884928705);
    \draw[line width=1.pt] (-2.892500000000012,-1.2379379884928705) -- (-2.860000000000012,-1.2344330338778438);
    \draw[line width=1.pt] (-2.860000000000012,-1.2344330338778438) -- (-2.8275000000000117,-1.2308563734568452);
    \draw[line width=1.pt] (-2.8275000000000117,-1.2308563734568452) -- (-2.7950000000000115,-1.2272058783903836);
    \draw[line width=1.pt] (-2.7950000000000115,-1.2272058783903836) -- (-2.7625000000000113,-1.2234793389075076);
    \draw[line width=1.pt] (-2.7625000000000113,-1.2234793389075076) -- (-2.730000000000011,-1.2196744606595613);
    \draw[line width=1.pt] (-2.730000000000011,-1.2196744606595613) -- (-2.697500000000011,-1.2157888608915735);
    \draw[line width=1.pt] (-2.697500000000011,-1.2157888608915735) -- (-2.6650000000000107,-1.2118200644219592);
    \draw[line width=1.pt] (-2.6650000000000107,-1.2118200644219592) -- (-2.6325000000000105,-1.207765499420813);
    \draw[line width=1.pt] (-2.6325000000000105,-1.207765499420813) -- (-2.6000000000000103,-1.2036224929766788);
    \draw[line width=1.pt] (-2.6000000000000103,-1.2036224929766788) -- (-2.56750000000001,-1.1993882664412787);
    \draw[line width=1.pt] (-2.56750000000001,-1.1993882664412787) -- (-2.53500000000001,-1.1950599305413034);
    \draw[line width=1.pt] (-2.53500000000001,-1.1950599305413034) -- (-2.5025000000000097,-1.1906344802459898);
    \draw[line width=1.pt] (-2.5025000000000097,-1.1906344802459898) -- (-2.4700000000000095,-1.1861087893788702);
    \draw[line width=1.pt] (-2.4700000000000095,-1.1861087893788702) -- (-2.4375000000000093,-1.181479604961757);
    \draw[line width=1.pt] (-2.4375000000000093,-1.181479604961757) -- (-2.405000000000009,-1.176743541278751);
    \draw[line width=1.pt] (-2.405000000000009,-1.176743541278751) -- (-2.372500000000009,-1.171897073647842);
    \draw[line width=1.pt] (-2.372500000000009,-1.171897073647842) -- (-2.3400000000000087,-1.1669365318875213);
    \draw[line width=1.pt] (-2.3400000000000087,-1.1669365318875213) -- (-2.3075000000000085,-1.1618580934657514);
    \draw[line width=1.pt] (-2.3075000000000085,-1.1618580934657514) -- (-2.2750000000000083,-1.1566577763186863);
    \draw[line width=1.pt] (-2.2750000000000083,-1.1566577763186863) -- (-2.242500000000008,-1.1513314313267011);
    \draw[line width=1.pt] (-2.242500000000008,-1.1513314313267011) -- (-2.210000000000008,-1.1458747344356222);
    \draw[line width=1.pt] (-2.210000000000008,-1.1458747344356222) -- (-2.1775000000000078,-1.1402831784115657);
    \draw[line width=1.pt] (-2.1775000000000078,-1.1402831784115657) -- (-2.1450000000000076,-1.1345520642185474);
    \draw[line width=1.pt] (-2.1450000000000076,-1.1345520642185474) -- (-2.1125000000000074,-1.1286764920090533);
    \draw[line width=1.pt] (-2.1125000000000074,-1.1286764920090533) -- (-2.080000000000007,-1.1226513517191083);
    \draw[line width=1.pt] (-2.080000000000007,-1.1226513517191083) -- (-2.047500000000007,-1.1164713132611337);
    \draw[line width=1.pt] (-2.047500000000007,-1.1164713132611337) -- (-2.015000000000007,-1.1101308163100791);
    \draw[line width=1.pt] (-2.015000000000007,-1.1101308163100791) -- (-1.9825000000000068,-1.1036240596810634);
    \draw[line width=1.pt] (-1.9825000000000068,-1.1036240596810634) -- (-1.9500000000000068,-1.0969449903001376);
    \draw[line width=1.pt] (-1.9500000000000068,-1.0969449903001376) -- (-1.9175000000000069,-1.0900872917739144);
    \draw[line width=1.pt] (-1.9175000000000069,-1.0900872917739144) -- (-1.885000000000007,-1.0830443725687977);
    \draw[line width=1.pt] (-1.885000000000007,-1.0830443725687977) -- (-1.852500000000007,-1.0758093538165685);
    \draw[line width=1.pt] (-1.852500000000007,-1.0758093538165685) -- (-1.820000000000007,-1.0683750567702675);
    \draw[line width=1.pt] (-1.820000000000007,-1.0683750567702675) -- (-1.787500000000007,-1.0607339899428774);
    \draw[line width=1.pt] (-1.787500000000007,-1.0607339899428774) -- (-1.755000000000007,-1.0528783359714489);
    \draw[line width=1.pt] (-1.755000000000007,-1.0528783359714489) -- (-1.722500000000007,-1.044799938261272);
    \draw[line width=1.pt] (-1.722500000000007,-1.044799938261272) -- (-1.690000000000007,-1.036490287478758);
    \draw[line width=1.pt] (-1.690000000000007,-1.036490287478758) -- (-1.657500000000007,-1.0279405079781463);
    \draw[line width=1.pt] (-1.657500000000007,-1.0279405079781463) -- (-1.625000000000007,-1.0191413442663517);
    \draw[line width=1.pt] (-1.625000000000007,-1.0191413442663517) -- (-1.5925000000000071,-1.0100831476325802);
    \draw[line width=1.pt] (-1.5925000000000071,-1.0100831476325802) -- (-1.5600000000000072,-1.0007558630951885);
    \draw[line width=1.pt] (-1.5600000000000072,-1.0007558630951885) -- (-1.5275000000000072,-0.9911490168480865);
    \draw[line width=1.pt] (-1.5275000000000072,-0.9911490168480865) -- (-1.4950000000000072,-0.9812517044232874);
    \draw[line width=1.pt] (-1.4950000000000072,-0.9812517044232874) -- (-1.4625000000000072,-0.9710525798254753);
    \draw[line width=1.pt] (-1.4625000000000072,-0.9710525798254753) -- (-1.4300000000000073,-0.9605398459392718);
    \draw[line width=1.pt] (-1.4300000000000073,-0.9605398459392718) -- (-1.3975000000000073,-0.949701246560719);
    \draw[line width=1.pt] (-1.3975000000000073,-0.949701246560719) -- (-1.3650000000000073,-0.9385240604619587);
    \draw[line width=1.pt] (-1.3650000000000073,-0.9385240604619587) -- (-1.3325000000000073,-0.9269950979626065);
    \draw[line width=1.pt] (-1.3325000000000073,-0.9269950979626065) -- (-1.3000000000000074,-0.9151007005533631);
    \draw[line width=1.pt] (-1.3000000000000074,-0.9151007005533631) -- (-1.2675000000000074,-0.9028267441972526);
    \draw[line width=1.pt] (-1.2675000000000074,-0.9028267441972526) -- (-1.2350000000000074,-0.8901586470216551);
    \draw[line width=1.pt] (-1.2350000000000074,-0.8901586470216551) -- (-1.2025000000000075,-0.8770813822098735);
    \draw[line width=1.pt] (-1.2025000000000075,-0.8770813822098735) -- (-1.1700000000000075,-0.8635794970038384);
    \draw[line width=1.pt] (-1.1700000000000075,-0.8635794970038384) -- (-1.1375000000000075,-0.8496371388387414);
    \draw[line width=1.pt] (-1.1375000000000075,-0.8496371388387414) -- (-1.1050000000000075,-0.8352380897442823);
    \draw[line width=1.pt] (-1.1050000000000075,-0.8352380897442823) -- (-1.0725000000000076,-0.8203658102634246);
    \draw[line width=1.pt] (-1.0725000000000076,-0.8203658102634246) -- (-1.0400000000000076,-0.8050034942546567);
    \draw[line width=1.pt] (-1.0400000000000076,-0.8050034942546567) -- (-1.0075000000000076,-0.7891341360531126);
    \draw[line width=1.pt] (-1.0075000000000076,-0.7891341360531126) -- (-0.9750000000000076,-0.7727406115634002);
    \draw[line width=1.pt] (-0.9750000000000076,-0.7727406115634002) -- (-0.9425000000000077,-0.7558057749347429);
    \draw[line width=1.pt] (-0.9425000000000077,-0.7558057749347429) -- (-0.9100000000000077,-0.7383125725172323);
    \draw[line width=1.pt] (-0.9100000000000077,-0.7383125725172323) -- (-0.8775000000000077,-0.7202441758045788);
    \draw[line width=1.pt] (-0.8775000000000077,-0.7202441758045788) -- (-0.8450000000000077,-0.7015841350193626);
    \draw[line width=1.pt] (-0.8450000000000077,-0.7015841350193626) -- (-0.8125000000000078,-0.6823165548747527);
    \draw[line width=1.pt] (-0.8125000000000078,-0.6823165548747527) -- (-0.7800000000000078,-0.662426293833156);
    \draw[line width=1.pt] (-0.7800000000000078,-0.662426293833156) -- (-0.7475000000000078,-0.6418991878567795);
    \draw[line width=1.pt] (-0.7475000000000078,-0.6418991878567795) -- (-0.7150000000000079,-0.6207222991863646);
    \draw[line width=1.pt] (-0.7150000000000079,-0.6207222991863646) -- (-0.6825000000000079,-0.598884190071563);
    \draw[line width=1.pt] (-0.6825000000000079,-0.598884190071563) -- (-0.6500000000000079,-0.5763752205911892);
    \draw[line width=1.pt] (-0.6500000000000079,-0.5763752205911892) -- (-0.6175000000000079,-0.5531878687303761);
    \draw[line width=1.pt] (-0.6175000000000079,-0.5531878687303761) -- (-0.585000000000008,-0.5293170697190619);
    \draw[line width=1.pt] (-0.585000000000008,-0.5293170697190619) -- (-0.552500000000008,-0.5047605702885752);
    \draw[line width=1.pt] (-0.552500000000008,-0.5047605702885752) -- (-0.520000000000008,-0.47951929199260246);
    \draw[line width=1.pt] (-0.520000000000008,-0.47951929199260246) -- (-0.48750000000000804,-0.4535976961077782);
    \draw[line width=1.pt] (-0.48750000000000804,-0.4535976961077782) -- (-0.45500000000000806,-0.4270041409433469);
    \draw[line width=1.pt] (-0.45500000000000806,-0.4270041409433469) -- (-0.4225000000000081,-0.39975122074057173);
    \draw[line width=1.pt] (-0.4225000000000081,-0.39975122074057173) -- (-0.3900000000000081,-0.3718560738485883);
    \draw[line width=1.pt] (-0.3900000000000081,-0.3718560738485883) -- (-0.35750000000000814,-0.3433406466650703);
    \draw[line width=1.pt] (-0.35750000000000814,-0.3433406466650703) -- (-0.32500000000000817,-0.3142318990843457);
    \draw[line width=1.pt] (-0.32500000000000817,-0.3142318990843457) -- (-0.2925000000000082,-0.2845619370639685);
    \draw[line width=1.pt] (-0.2925000000000082,-0.2845619370639685) -- (-0.2600000000000082,-0.25436805855327366);
    \draw[line width=1.pt] (-0.2600000000000082,-0.25436805855327366) -- (-0.22750000000000822,-0.2236927005430036);
    \draw[line width=1.pt] (-0.22750000000000822,-0.2236927005430036) -- (-0.19500000000000822,-0.19258327745992646);
    \draw[line width=1.pt] (-0.19500000000000822,-0.19258327745992646) -- (-0.16250000000000822,-0.1610919045375885);
    \draw[line width=1.pt] (-0.16250000000000822,-0.1610919045375885) -- (-0.13000000000000822,-0.12927500404815115);
    \draw[line width=1.pt] (-0.13000000000000822,-0.12927500404815115) -- (-0.09750000000000822,-0.09719279718858866);
    \draw[line width=1.pt] (-0.09750000000000822,-0.09719279718858866) -- (-0.06500000000000822,-0.06490868969344164);
    \draw[line width=1.pt] (-0.06500000000000822,-0.06490868969344164) -- (-0.03250000000000822,-0.03248856453802453);
    \draw[line width=1.pt] (-0.03250000000000822,-0.03248856453802453) -- (0.0,0.0);
    \draw[line width=1.pt] (0.0,0.0) -- (0.032499999999991785,0.03248856453800812);
    \draw[line width=1.pt] (0.032499999999991785,0.03248856453800812) -- (0.06499999999999179,0.06490868969342528);
    \draw[line width=1.pt] (0.06499999999999179,0.06490868969342528) -- (0.09749999999999179,0.09719279718857238);
    \draw[line width=1.pt] (0.09749999999999179,0.09719279718857238) -- (0.1299999999999918,0.12927500404813497);
    \draw[line width=1.pt] (0.1299999999999918,0.12927500404813497) -- (0.1624999999999918,0.1610919045375725);
    \draw[line width=1.pt] (0.1624999999999918,0.1610919045375725) -- (0.1949999999999918,0.19258327745991063);
    \draw[line width=1.pt] (0.1949999999999918,0.19258327745991063) -- (0.2274999999999918,0.22369270054298798);
    \draw[line width=1.pt] (0.2274999999999918,0.22369270054298798) -- (0.2599999999999918,0.2543680585532582);
    \draw[line width=1.pt] (0.2599999999999918,0.2543680585532582) -- (0.29249999999999177,0.28456193706395333);
    \draw[line width=1.pt] (0.29249999999999177,0.28456193706395333) -- (0.32499999999999174,0.31423189908433086);
    \draw[line width=1.pt] (0.32499999999999174,0.31423189908433086) -- (0.3574999999999917,0.34334064666505576);
    \draw[line width=1.pt] (0.3574999999999917,0.34334064666505576) -- (0.3899999999999917,0.37185607384857405);
    \draw[line width=1.pt] (0.3899999999999917,0.37185607384857405) -- (0.42249999999999166,0.3997512207405578);
    \draw[line width=1.pt] (0.42249999999999166,0.3997512207405578) -- (0.45499999999999163,0.4270041409433332);
    \draw[line width=1.pt] (0.45499999999999163,0.4270041409433332) -- (0.4874999999999916,0.45359769610776496);
    \draw[line width=1.pt] (0.4874999999999916,0.45359769610776496) -- (0.5199999999999916,0.4795192919925895);
    \draw[line width=1.pt] (0.5199999999999916,0.4795192919925895) -- (0.5524999999999916,0.5047605702885626);
    \draw[line width=1.pt] (0.5524999999999916,0.5047605702885626) -- (0.5849999999999915,0.5293170697190497);
    \draw[line width=1.pt] (0.5849999999999915,0.5293170697190497) -- (0.6174999999999915,0.5531878687303642);
    \draw[line width=1.pt] (0.6174999999999915,0.5531878687303642) -- (0.6499999999999915,0.5763752205911776);
    \draw[line width=1.pt] (0.6499999999999915,0.5763752205911776) -- (0.6824999999999914,0.5988841900715518);
    \draw[line width=1.pt] (0.6824999999999914,0.5988841900715518) -- (0.7149999999999914,0.6207222991863537);
    \draw[line width=1.pt] (0.7149999999999914,0.6207222991863537) -- (0.7474999999999914,0.641899187856769);
    \draw[line width=1.pt] (0.7474999999999914,0.641899187856769) -- (0.7799999999999914,0.6624262938331458);
    \draw[line width=1.pt] (0.7799999999999914,0.6624262938331458) -- (0.8124999999999913,0.6823165548747429);
    \draw[line width=1.pt] (0.8124999999999913,0.6823165548747429) -- (0.8449999999999913,0.701584135019353);
    \draw[line width=1.pt] (0.8449999999999913,0.701584135019353) -- (0.8774999999999913,0.7202441758045696);
    \draw[line width=1.pt] (0.8774999999999913,0.7202441758045696) -- (0.9099999999999913,0.7383125725172233);
    \draw[line width=1.pt] (0.9099999999999913,0.7383125725172233) -- (0.9424999999999912,0.7558057749347342);
    \draw[line width=1.pt] (0.9424999999999912,0.7558057749347342) -- (0.9749999999999912,0.7727406115633918);
    \draw[line width=1.pt] (0.9749999999999912,0.7727406115633918) -- (1.0074999999999912,0.7891341360531043);
    \draw[line width=1.pt] (1.0074999999999912,0.7891341360531043) -- (1.0399999999999912,0.8050034942546488);
    \draw[line width=1.pt] (1.0399999999999912,0.8050034942546488) -- (1.0724999999999911,0.820365810263417);
    \draw[line width=1.pt] (1.0724999999999911,0.820365810263417) -- (1.104999999999991,0.8352380897442749);
    \draw[line width=1.pt] (1.104999999999991,0.8352380897442749) -- (1.137499999999991,0.8496371388387343);
    \draw[line width=1.pt] (1.137499999999991,0.8496371388387343) -- (1.169999999999991,0.8635794970038315);
    \draw[line width=1.pt] (1.169999999999991,0.8635794970038315) -- (1.202499999999991,0.8770813822098668);
    \draw[line width=1.pt] (1.202499999999991,0.8770813822098668) -- (1.234999999999991,0.8901586470216487);
    \draw[line width=1.pt] (1.234999999999991,0.8901586470216487) -- (1.267499999999991,0.9028267441972464);
    \draw[line width=1.pt] (1.267499999999991,0.9028267441972464) -- (1.299999999999991,0.915100700553357);
    \draw[line width=1.pt] (1.299999999999991,0.915100700553357) -- (1.332499999999991,0.9269950979626006);
    \draw[line width=1.pt] (1.332499999999991,0.9269950979626006) -- (1.3649999999999909,0.9385240604619529);
    \draw[line width=1.pt] (1.3649999999999909,0.9385240604619529) -- (1.3974999999999909,0.9497012465607134);
    \draw[line width=1.pt] (1.3974999999999909,0.9497012465607134) -- (1.4299999999999908,0.9605398459392664);
    \draw[line width=1.pt] (1.4299999999999908,0.9605398459392664) -- (1.4624999999999908,0.9710525798254701);
    \draw[line width=1.pt] (1.4624999999999908,0.9710525798254701) -- (1.4949999999999908,0.9812517044232822);
    \draw[line width=1.pt] (1.4949999999999908,0.9812517044232822) -- (1.5274999999999908,0.9911490168480817);
    \draw[line width=1.pt] (1.5274999999999908,0.9911490168480817) -- (1.5599999999999907,1.0007558630951836);
    \draw[line width=1.pt] (1.5599999999999907,1.0007558630951836) -- (1.5924999999999907,1.0100831476325756);
    \draw[line width=1.pt] (1.5924999999999907,1.0100831476325756) -- (1.6249999999999907,1.0191413442663473);
    \draw[line width=1.pt] (1.6249999999999907,1.0191413442663473) -- (1.6574999999999906,1.027940507978142);
    \draw[line width=1.pt] (1.6574999999999906,1.027940507978142) -- (1.6899999999999906,1.0364902874787538);
    \draw[line width=1.pt] (1.6899999999999906,1.0364902874787538) -- (1.7224999999999906,1.0447999382612678);
    \draw[line width=1.pt] (1.7224999999999906,1.0447999382612678) -- (1.7549999999999906,1.0528783359714449);
    \draw[line width=1.pt] (1.7549999999999906,1.0528783359714449) -- (1.7874999999999905,1.0607339899428736);
    \draw[line width=1.pt] (1.7874999999999905,1.0607339899428736) -- (1.8199999999999905,1.0683750567702637);
    \draw[line width=1.pt] (1.8199999999999905,1.0683750567702637) -- (1.8524999999999905,1.075809353816565);
    \draw[line width=1.pt] (1.8524999999999905,1.075809353816565) -- (1.8849999999999905,1.0830443725687942);
    \draw[line width=1.pt] (1.8849999999999905,1.0830443725687942) -- (1.9174999999999904,1.090087291773911);
    \draw[line width=1.pt] (1.9174999999999904,1.090087291773911) -- (1.9499999999999904,1.0969449903001343);
    \draw[line width=1.pt] (1.9499999999999904,1.0969449903001343) -- (1.9824999999999904,1.10362405968106);
    \draw[line width=1.pt] (1.9824999999999904,1.10362405968106) -- (2.0149999999999904,1.1101308163100758);
    \draw[line width=1.pt] (2.0149999999999904,1.1101308163100758) -- (2.0474999999999905,1.1164713132611306);
    \draw[line width=1.pt] (2.0474999999999905,1.1164713132611306) -- (2.0799999999999907,1.1226513517191052);
    \draw[line width=1.pt] (2.0799999999999907,1.1226513517191052) -- (2.112499999999991,1.1286764920090504);
    \draw[line width=1.pt] (2.112499999999991,1.1286764920090504) -- (2.144999999999991,1.1345520642185445);
    \draw[line width=1.pt] (2.144999999999991,1.1345520642185445) -- (2.1774999999999913,1.1402831784115628);
    \draw[line width=1.pt] (2.1774999999999913,1.1402831784115628) -- (2.2099999999999915,1.1458747344356195);
    \draw[line width=1.pt] (2.2099999999999915,1.1458747344356195) -- (2.2424999999999917,1.1513314313266985);
    \draw[line width=1.pt] (2.2424999999999917,1.1513314313266985) -- (2.274999999999992,1.1566577763186836);
    \draw[line width=1.pt] (2.274999999999992,1.1566577763186836) -- (2.307499999999992,1.161858093465749);
    \draw[line width=1.pt] (2.307499999999992,1.161858093465749) -- (2.3399999999999923,1.166936531887519);
    \draw[line width=1.pt] (2.3399999999999923,1.166936531887519) -- (2.3724999999999925,1.1718970736478396);
    \draw[line width=1.pt] (2.3724999999999925,1.1718970736478396) -- (2.4049999999999927,1.1767435412787486);
    \draw[line width=1.pt] (2.4049999999999927,1.1767435412787486) -- (2.437499999999993,1.1814796049617549);
    \draw[line width=1.pt] (2.437499999999993,1.1814796049617549) -- (2.469999999999993,1.1861087893788678);
    \draw[line width=1.pt] (2.469999999999993,1.1861087893788678) -- (2.5024999999999933,1.1906344802459874);
    \draw[line width=1.pt] (2.5024999999999933,1.1906344802459874) -- (2.5349999999999935,1.1950599305413012);
    \draw[line width=1.pt] (2.5349999999999935,1.1950599305413012) -- (2.5674999999999937,1.1993882664412765);
    \draw[line width=1.pt] (2.5674999999999937,1.1993882664412765) -- (2.599999999999994,1.2036224929766766);
    \draw[line width=1.pt] (2.599999999999994,1.2036224929766766) -- (2.632499999999994,1.207765499420811);
    \draw[line width=1.pt] (2.632499999999994,1.207765499420811) -- (2.6649999999999943,1.2118200644219572);
    \draw[line width=1.pt] (2.6649999999999943,1.2118200644219572) -- (2.6974999999999945,1.2157888608915717);
    \draw[line width=1.pt] (2.6974999999999945,1.2157888608915717) -- (2.7299999999999947,1.2196744606595593);
    \draw[line width=1.pt] (2.7299999999999947,1.2196744606595593) -- (2.762499999999995,1.2234793389075058);
    \draw[line width=1.pt] (2.762499999999995,1.2234793389075058) -- (2.794999999999995,1.2272058783903819);
    \draw[line width=1.pt] (2.794999999999995,1.2272058783903819) -- (2.8274999999999952,1.2308563734568434);
    \draw[line width=1.pt] (2.8274999999999952,1.2308563734568434) -- (2.8599999999999954,1.234433033877842);
    \draw[line width=1.pt] (2.8599999999999954,1.234433033877842) -- (2.8924999999999956,1.2379379884928687);
    \draw[line width=1.pt] (2.8924999999999956,1.2379379884928687) -- (2.924999999999996,1.2413732886827487);
    \draw[line width=1.pt] (2.924999999999996,1.2413732886827487) -- (2.957499999999996,1.244740911677517);
    \draw[line width=1.pt] (2.957499999999996,1.244740911677517) -- (2.989999999999996,1.248042763707524);
    \draw[line width=1.pt] (2.989999999999996,1.248042763707524) -- (3.0224999999999964,1.2512806830055347);
    \draw[line width=1.pt] (3.0224999999999964,1.2512806830055347) -- (3.0549999999999966,1.254456442667231);
    \draw[line width=1.pt] (3.0549999999999966,1.254456442667231) -- (3.087499999999997,1.257571753377171);
    \draw[line width=1.pt] (3.087499999999997,1.257571753377171) -- (3.119999999999997,1.2606282660069104);
    \draw[line width=1.pt] (3.119999999999997,1.2606282660069104) -- (3.152499999999997,1.2636275740916767);
    \draw[line width=1.pt] (3.152499999999997,1.2636275740916767) -- (3.1849999999999974,1.266571216191659);
    \draw[line width=1.pt] (3.1849999999999974,1.266571216191659) -- (3.2174999999999976,1.2694606781436786);
    \draw[line width=1.pt] (3.2174999999999976,1.2694606781436786) -- (3.249999999999998,1.272297395208717);
    \draw[line width=1.pt] (3.249999999999998,1.272297395208717) -- (3.282499999999998,1.2750827541204983);
    \draw[line width=1.pt] (3.282499999999998,1.2750827541204983) -- (3.314999999999998,1.277818095040058);
    \draw[line width=1.pt] (3.314999999999998,1.277818095040058) -- (3.3474999999999984,1.280504713420982);
    \draw[line width=1.pt] (3.3474999999999984,1.280504713420982) -- (3.3799999999999986,1.2831438617897546);
    \draw[line width=1.pt] (3.3799999999999986,1.2831438617897546) -- (3.4124999999999988,1.285736751445429);
    \draw[line width=1.pt] (3.4124999999999988,1.285736751445429) -- (3.444999999999999,1.288284554082612);
    \draw[line width=1.pt] (3.444999999999999,1.288284554082612) -- (3.477499999999999,1.2907884033415558);
    \draw[line width=1.pt] (3.477499999999999,1.2907884033415558) -- (3.5099999999999993,1.2932493962889446);
    \draw[line width=1.pt] (3.5099999999999993,1.2932493962889446) -- (3.5424999999999995,1.2956685948327822);
    \draw[line width=1.pt] (3.5424999999999995,1.2956685948327822) -- (3.5749999999999997,1.2980470270746114);
    \draw[line width=1.pt] (3.5749999999999997,1.2980470270746114) -- (3.6075,1.300385688602125);
    \draw[line width=1.pt] (3.6075,1.300385688602125) -- (3.64,1.3026855437250695);
    \draw[line width=1.pt] (3.64,1.3026855437250695) -- (3.6725000000000003,1.3049475266571966);
    \draw[line width=1.pt] (3.6725000000000003,1.3049475266571966) -- (3.7050000000000005,1.3071725426468683);
    \draw[line width=1.pt] (3.7050000000000005,1.3071725426468683) -- (3.7375000000000007,1.309361469058794);
    \draw[line width=1.pt] (3.7375000000000007,1.309361469058794) -- (3.770000000000001,1.3115151564092398);
    \draw[line width=1.pt] (3.770000000000001,1.3115151564092398) -- (3.802500000000001,1.3136344293569437);
    \draw[line width=1.pt] (3.802500000000001,1.3136344293569437) -- (3.8350000000000013,1.315720087651839);
    \draw[line width=1.pt] (3.8350000000000013,1.315720087651839) -- (3.8675000000000015,1.3177729070435933);
    \draw[line width=1.pt] (3.8675000000000015,1.3177729070435933) -- (3.9000000000000017,1.3197936401518622);
    \draw[line width=1.pt] (3.9000000000000017,1.3197936401518622) -- (3.932500000000002,1.321783017300056);
    \draw[line width=1.pt] (3.932500000000002,1.321783017300056) -- (3.965000000000002,1.3237417473143365);
    \draw[line width=1.pt] (3.965000000000002,1.3237417473143365) -- (3.9975000000000023,1.3256705182894595);
    \draw[line width=1.pt] (3.9975000000000023,1.3256705182894595) -- (4.030000000000002,1.3275699983230107);
    \draw[line width=1.pt] (4.030000000000002,1.3275699983230107) -- (4.062500000000002,1.329440836219493);
    \draw[line width=1.pt] (4.062500000000002,1.329440836219493) -- (4.0950000000000015,1.331283662165656);
    \draw[line width=1.pt] (4.0950000000000015,1.331283662165656) -- (4.127500000000001,1.3330990883783889);
    \draw[line width=1.pt] (4.127500000000001,1.3330990883783889) -- (4.160000000000001,1.334887709726425);
    \draw[line width=1.pt] (4.160000000000001,1.334887709726425) -- (4.192500000000001,1.3366501043270542);
    \draw[line width=1.pt] (4.192500000000001,1.3366501043270542) -- (4.2250000000000005,1.338386834118969);
    \draw[line width=1.pt] (4.2250000000000005,1.338386834118969) -- (4.2575,1.3400984454123241);
    \draw[line width=1.pt] (4.2575,1.3400984454123241) -- (4.29,1.3417854694170261);
    \draw[line width=1.pt] (4.29,1.3417854694170261) -- (4.3225,1.3434484227502321);
    \draw[line width=1.pt] (4.3225,1.3434484227502321) -- (4.3549999999999995,1.3450878079239728);
    \draw[line width=1.pt] (4.3549999999999995,1.3450878079239728) -- (4.387499999999999,1.3467041138137845);
    \draw[line width=1.pt] (4.387499999999999,1.3467041138137845) -- (4.419999999999999,1.3482978161091863);
    \draw[line width=1.pt] (4.419999999999999,1.3482978161091863) -- (4.452499999999999,1.3498693777467918);
    \draw[line width=1.pt] (4.452499999999999,1.3498693777467918) -- (4.4849999999999985,1.3514192493268204);
    \draw[line width=1.pt] (4.4849999999999985,1.3514192493268204) -- (4.517499999999998,1.352947869513722);
    \draw[line width=1.pt] (4.517499999999998,1.352947869513722) -- (4.549999999999998,1.354455665421606);
    \draw[line width=1.pt] (4.549999999999998,1.354455665421606) -- (4.582499999999998,1.3559430529851226);
    \draw[line width=1.pt] (4.582499999999998,1.3559430529851226) -- (4.6149999999999975,1.357410437316426);
    \draw[line width=1.pt] (4.6149999999999975,1.357410437316426) -- (4.647499999999997,1.3588582130488027);
    \draw[line width=1.pt] (4.647499999999997,1.3588582130488027) -- (4.679999999999997,1.3602867646675403);
    \draw[line width=1.pt] (4.679999999999997,1.3602867646675403) -- (4.712499999999997,1.3616964668285652);
    \draw[line width=1.pt] (4.712499999999997,1.3616964668285652) -- (4.7449999999999966,1.3630876846653686);
    \draw[line width=1.pt] (4.7449999999999966,1.3630876846653686) -- (4.777499999999996,1.364460774084709);
    \draw[line width=1.pt] (4.777499999999996,1.364460774084709) -- (4.809999999999996,1.3658160820515555);
    \draw[line width=1.pt] (4.809999999999996,1.3658160820515555) -- (4.842499999999996,1.3671539468637182);
    \draw[line width=1.pt] (4.842499999999996,1.3671539468637182) -- (4.874999999999996,1.3684746984165927);
    \draw[line width=1.pt] (4.874999999999996,1.3684746984165927) -- (4.907499999999995,1.369778658458418);
    \draw[line width=1.pt] (4.907499999999995,1.369778658458418) -- (4.939999999999995,1.3710661408364406);
    \draw[line width=1.pt] (4.939999999999995,1.3710661408364406) -- (4.972499999999995,1.3723374517343514);
    \draw[line width=1.pt] (4.972499999999995,1.3723374517343514) -- (5.004999999999995,1.3735928899013454);
    \draw[line width=1.pt] (5.004999999999995,1.3735928899013454) -- (5.037499999999994,1.374832746873146);
    \draw[line width=1.pt] (5.037499999999994,1.374832746873146) -- (5.069999999999994,1.3760573071853108);
    \draw[line width=1.pt] (5.069999999999994,1.3760573071853108) -- (5.102499999999994,1.3772668485791288);
    \draw[line width=1.pt] (5.102499999999994,1.3772668485791288) -- (5.134999999999994,1.378461642200401);
    \draw[line width=1.pt] (5.134999999999994,1.378461642200401) -- (5.167499999999993,1.3796419527913864);
    \draw[line width=1.pt] (5.167499999999993,1.3796419527913864) -- (5.199999999999993,1.3808080388761805);
    \draw[line width=1.pt] (5.199999999999993,1.3808080388761805) -- (5.232499999999993,1.3819601529397842);
    \draw[line width=1.pt] (5.232499999999993,1.3819601529397842) -- (5.264999999999993,1.3830985416011048);
    \draw[line width=1.pt] (5.264999999999993,1.3830985416011048) -- (5.297499999999992,1.3842234457801308);
    \draw[line width=1.pt] (5.297499999999992,1.3842234457801308) -- (5.329999999999992,1.3853351008594958);
    \draw[line width=1.pt] (5.329999999999992,1.3853351008594958) -- (5.362499999999992,1.3864337368406552);
    \draw[line width=1.pt] (5.362499999999992,1.3864337368406552) -- (5.394999999999992,1.3875195784948748);
    \draw[line width=1.pt] (5.394999999999992,1.3875195784948748) -- (5.427499999999991,1.3885928455092327);
    \draw[line width=1.pt] (5.427499999999991,1.3885928455092327) -- (5.459999999999991,1.3896537526278183);
    \draw[line width=1.pt] (5.459999999999991,1.3896537526278183) -- (5.492499999999991,1.3907025097883157);
    \draw[line width=1.pt] (5.492499999999991,1.3907025097883157) -- (5.524999999999991,1.3917393222541359);
    \draw[line width=1.pt] (5.524999999999991,1.3917393222541359) -- (5.55749999999999,1.3927643907422724);
    \draw[line width=1.pt] (5.55749999999999,1.3927643907422724) -- (5.58999999999999,1.393777911547031);
    \draw[line width=1.pt] (5.58999999999999,1.393777911547031) -- (5.62249999999999,1.3947800766597924);
    \draw[line width=1.pt] (5.62249999999999,1.3947800766597924) -- (5.65499999999999,1.395771073884949);
    \draw[line width=1.pt] (5.65499999999999,1.395771073884949) -- (5.687499999999989,1.3967510869521576);
    \draw[line width=1.pt] (5.687499999999989,1.3967510869521576) -- (5.719999999999989,1.3977202956250432);
    \draw[line width=1.pt] (5.719999999999989,1.3977202956250432) -- (5.752499999999989,1.3986788758064808);
    \draw[line width=1.pt] (5.752499999999989,1.3986788758064808) -- (5.784999999999989,1.399626999640578);
    \draw[line width=1.pt] (5.784999999999989,1.399626999640578) -- (5.817499999999988,1.4005648356114802);
    \draw[line width=1.pt] (5.817499999999988,1.4005648356114802) -- (5.849999999999988,1.4014925486391088);
    \draw[line width=1.pt] (5.849999999999988,1.4014925486391088) -- (5.882499999999988,1.402410300171943);
    \draw[line width=1.pt] (5.882499999999988,1.402410300171943) -- (5.914999999999988,1.4033182482769486);
    \draw[line width=1.pt] (5.914999999999988,1.4033182482769486) -- (5.947499999999987,1.4042165477267574);
    \draw[line width=1.pt] (5.947499999999987,1.4042165477267574) -- (5.979999999999987,1.40510535008419);
    \draw[line width=1.pt] (5.979999999999987,1.40510535008419) -- (6.012499999999987,1.405984803784217);
    \draw[line width=1.pt] (6.012499999999987,1.405984803784217) -- (6.044999999999987,1.406855054213449);
    \draw[line width=1.pt] (6.044999999999987,1.406855054213449) -- (6.077499999999986,1.407716243787236);
    \draw[line width=1.pt] (6.077499999999986,1.407716243787236) -- (6.109999999999986,1.408568512024465);
    \draw[line width=1.pt] (6.109999999999986,1.408568512024465) -- (6.142499999999986,1.4094119956201319);
    \draw[line width=1.pt] (6.142499999999986,1.4094119956201319) -- (6.174999999999986,1.41024682851576);
    \draw[line width=1.pt] (6.174999999999986,1.41024682851576) -- (6.207499999999985,1.4110731419677474);
    \draw[line width=1.pt] (6.207499999999985,1.4110731419677474) -- (6.239999999999985,1.411891064613703);
    \draw[line width=1.pt] (6.239999999999985,1.411891064613703) -- (6.272499999999985,1.4127007225368493);
    \draw[line width=1.pt] (6.272499999999985,1.4127007225368493) -- (6.304999999999985,1.4135022393285457);
    \draw[line width=1.pt] (6.304999999999985,1.4135022393285457) -- (6.337499999999984,1.4142957361490076);
    \draw[line width=1.pt] (6.337499999999984,1.4142957361490076) -- (6.369999999999984,1.4150813317862703);
    \draw[line width=1.pt] (6.369999999999984,1.4150813317862703) -- (6.402499999999984,1.4158591427134615);
    \draw[line width=1.pt] (6.402499999999984,1.4158591427134615) -- (6.434999999999984,1.4166292831444398);
    \draw[line width=1.pt] (6.434999999999984,1.4166292831444398) -- (6.467499999999983,1.417391865087847);
    \draw[line width=1.pt] (6.467499999999983,1.417391865087847) -- (6.499999999999983,1.418146998399631);
    \draw[line width=1.pt,dotted,smooth,samples=100,domain=-6.5:6.5] plot(\x,{3.141592653589793/2});
    \draw[line width=1.pt,dotted,smooth,samples=100,domain=-6.5:6.5] plot(\x,{0-3.141592653589793/2});
    \begin{scriptsize}
    \draw[color=black] (-7.774790164207696,-1.341063968686404) node {$f$};
    \draw[color=black] (-7.774790164207696,1.6680449742324768) node {$g$};
    \draw[color=black] (-7.774790164207696,-1.465970377637754) node {$h$};
    \end{scriptsize}
    \end{axis}
\end{tikzpicture}
            \caption{Graf funkce $g_1$.}
            \label{fig:graf_g1}
        \end{figure}
        \item\label{item:bijekce_0_a_1} Bijekce mezi $\R$ a intervalem $(0,1)$: $\map{g_2}{\R}{(0,1)}$, kde
        \begin{equation*}
            g_2(x)=\frac{1}{\pi}\left(\arctg{x}+\frac{\pi}{2}\right)
        \end{equation*}
        \begin{figure}[H]
            \centering
            \begin{tikzpicture}[line cap=round,line join=round,>=triangle 45,x=1.0cm,y=1.0cm]
    \begin{axis}[
    x=1.0cm,y=1.0cm,
    axis lines=middle,
    xmin=-6.5,
    xmax=6.5,
    ymin=-0.2,
    ymax=1.2,
    xtick={-6.0,-5.0,...,6.0},
    ytick={-0.0,1.0,...,1.0},]
    \clip(-6.5,-0.2) rectangle (6.5,1.2);
    \draw[line width=1.pt,dotted,smooth,samples=100,domain=-6.5:6.5] plot(\x,{1});
    \draw[line width=1.pt] (-6.5,0.048589790347528895) -- (-6.5,0.048589790347528895);
    \draw[line width=1.pt] (-6.5,0.048589790347528895) -- (-6.4675,0.04883015674605649);
    \draw[line width=1.pt] (-6.4675,0.04883015674605649) -- (-6.4350000000000005,0.049072894117668235);
    \draw[line width=1.pt] (-6.4350000000000005,0.049072894117668235) -- (-6.402500000000001,0.04931803743059849);
    \draw[line width=1.pt] (-6.402500000000001,0.04931803743059849) -- (-6.370000000000001,0.04956562233830523);
    \draw[line width=1.pt] (-6.370000000000001,0.04956562233830523) -- (-6.337500000000001,0.049815685196188786);
    \draw[line width=1.pt] (-6.337500000000001,0.049815685196188786) -- (-6.3050000000000015,0.05006826307879722);
    \draw[line width=1.pt] (-6.3050000000000015,0.05006826307879722) -- (-6.272500000000002,0.050323393797536566);
    \draw[line width=1.pt] (-6.272500000000002,0.050323393797536566) -- (-6.240000000000002,0.05058111591890098);
    \draw[line width=1.pt] (-6.240000000000002,0.05058111591890098) -- (-6.207500000000002,0.050841468783242284);
    \draw[line width=1.pt] (-6.207500000000002,0.050841468783242284) -- (-6.1750000000000025,0.05110449252409655);
    \draw[line width=1.pt] (-6.1750000000000025,0.05110449252409655) -- (-6.142500000000003,0.051370228088086466);
    \draw[line width=1.pt] (-6.142500000000003,0.051370228088086466) -- (-6.110000000000003,0.05163871725542098);
    \draw[line width=1.pt] (-6.110000000000003,0.05163871725542098) -- (-6.077500000000003,0.05191000266101146);
    \draw[line width=1.pt] (-6.077500000000003,0.05191000266101146) -- (-6.0450000000000035,0.05218412781622623);
    \draw[line width=1.pt] (-6.0450000000000035,0.05218412781622623) -- (-6.012500000000004,0.05246113713130644);
    \draw[line width=1.pt] (-6.012500000000004,0.05246113713130644) -- (-5.980000000000004,0.052741075938466005);
    \draw[line width=1.pt] (-5.980000000000004,0.052741075938466005) -- (-5.947500000000004,0.053023990515700226);
    \draw[line width=1.pt] (-5.947500000000004,0.053023990515700226) -- (-5.9150000000000045,0.05330992811132783);
    \draw[line width=1.pt] (-5.9150000000000045,0.05330992811132783) -- (-5.882500000000005,0.05359893696929295);
    \draw[line width=1.pt] (-5.882500000000005,0.05359893696929295) -- (-5.850000000000005,0.05389106635525442);
    \draw[line width=1.pt] (-5.850000000000005,0.05389106635525442) -- (-5.817500000000005,0.0541863665834901);
    \draw[line width=1.pt] (-5.817500000000005,0.0541863665834901) -- (-5.7850000000000055,0.05448488904464706);
    \draw[line width=1.pt] (-5.7850000000000055,0.05448488904464706) -- (-5.752500000000006,0.054786686234366666);
    \draw[line width=1.pt] (-5.752500000000006,0.054786686234366666) -- (-5.720000000000006,0.05509181178281802);
    \draw[line width=1.pt] (-5.720000000000006,0.05509181178281802) -- (-5.687500000000006,0.05540032048517266);
    \draw[line width=1.pt] (-5.687500000000006,0.05540032048517266) -- (-5.6550000000000065,0.0557122683330544);
    \draw[line width=1.pt] (-5.6550000000000065,0.0557122683330544) -- (-5.622500000000007,0.056027712547002416);
    \draw[line width=1.pt] (-5.622500000000007,0.056027712547002416) -- (-5.590000000000007,0.05634671160998294);
    \draw[line width=1.pt] (-5.590000000000007,0.05634671160998294) -- (-5.557500000000007,0.05666932530199055);
    \draw[line width=1.pt] (-5.557500000000007,0.05666932530199055) -- (-5.5250000000000075,0.05699561473577981);
    \draw[line width=1.pt] (-5.5250000000000075,0.05699561473577981) -- (-5.492500000000008,0.05732564239376905);
    \draw[line width=1.pt] (-5.492500000000008,0.05732564239376905) -- (-5.460000000000008,0.057659472166161324);
    \draw[line width=1.pt] (-5.460000000000008,0.057659472166161324) -- (-5.427500000000008,0.057997169390329964);
    \draw[line width=1.pt] (-5.427500000000008,0.057997169390329964) -- (-5.3950000000000085,0.058338800891514955);
    \draw[line width=1.pt] (-5.3950000000000085,0.058338800891514955) -- (-5.362500000000009,0.058684435024883304);
    \draw[line width=1.pt] (-5.362500000000009,0.058684435024883304) -- (-5.330000000000009,0.05903414171900355);
    \draw[line width=1.pt] (-5.330000000000009,0.05903414171900355) -- (-5.297500000000009,0.059387992520791806);
    \draw[line width=1.pt] (-5.297500000000009,0.059387992520791806) -- (-5.2650000000000095,0.05974606064198524);
    \draw[line width=1.pt] (-5.2650000000000095,0.05974606064198524) -- (-5.23250000000001,0.06010842100720317);
    \draw[line width=1.pt] (-5.23250000000001,0.06010842100720317) -- (-5.20000000000001,0.06047515030365954);
    \draw[line width=1.pt] (-5.20000000000001,0.06047515030365954) -- (-5.16750000000001,0.06084632703259088);
    \draw[line width=1.pt] (-5.16750000000001,0.06084632703259088) -- (-5.1350000000000104,0.06122203156246894);
    \draw[line width=1.pt] (-5.1350000000000104,0.06122203156246894) -- (-5.102500000000011,0.06160234618406921);
    \draw[line width=1.pt] (-5.102500000000011,0.06160234618406921) -- (-5.070000000000011,0.061987355167470004);
    \draw[line width=1.pt] (-5.070000000000011,0.061987355167470004) -- (-5.037500000000011,0.06237714482106036);
    \draw[line width=1.pt] (-5.037500000000011,0.06237714482106036) -- (-5.005000000000011,0.0627718035526384);
    \draw[line width=1.pt] (-5.005000000000011,0.0627718035526384) -- (-4.972500000000012,0.06317142193268506);
    \draw[line width=1.pt] (-4.972500000000012,0.06317142193268506) -- (-4.940000000000012,0.06357609275990324);
    \draw[line width=1.pt] (-4.940000000000012,0.06357609275990324) -- (-4.907500000000012,0.06398591112911528);
    \draw[line width=1.pt] (-4.907500000000012,0.06398591112911528) -- (-4.875000000000012,0.06440097450161687);
    \draw[line width=1.pt] (-4.875000000000012,0.06440097450161687) -- (-4.842500000000013,0.06482138277808937);
    \draw[line width=1.pt] (-4.842500000000013,0.06482138277808937) -- (-4.810000000000013,0.06524723837417826);
    \draw[line width=1.pt] (-4.810000000000013,0.06524723837417826) -- (-4.777500000000013,0.06567864629884913);
    \draw[line width=1.pt] (-4.777500000000013,0.06567864629884913) -- (-4.745000000000013,0.06611571423563956);
    \draw[line width=1.pt] (-4.745000000000013,0.06611571423563956) -- (-4.712500000000014,0.06655855262692931);
    \draw[line width=1.pt] (-4.712500000000014,0.06655855262692931) -- (-4.680000000000014,0.06700727476135811);
    \draw[line width=1.pt] (-4.680000000000014,0.06700727476135811) -- (-4.647500000000014,0.06746199686452613);
    \draw[line width=1.pt] (-4.647500000000014,0.06746199686452613) -- (-4.615000000000014,0.0679228381931187);
    \draw[line width=1.pt] (-4.615000000000014,0.0679228381931187) -- (-4.582500000000015,0.06838992113260367);
    \draw[line width=1.pt] (-4.582500000000015,0.06838992113260367) -- (-4.550000000000015,0.06886337129865791);
    \draw[line width=1.pt] (-4.550000000000015,0.06886337129865791) -- (-4.517500000000015,0.06934331764248478);
    \draw[line width=1.pt] (-4.517500000000015,0.06934331764248478) -- (-4.485000000000015,0.06982989256019573);
    \draw[line width=1.pt] (-4.485000000000015,0.06982989256019573) -- (-4.452500000000016,0.07032323200643413);
    \draw[line width=1.pt] (-4.452500000000016,0.07032323200643413) -- (-4.420000000000016,0.07082347561243116);
    \draw[line width=1.pt] (-4.420000000000016,0.07082347561243116) -- (-4.387500000000016,0.07133076680869127);
    \draw[line width=1.pt] (-4.387500000000016,0.07133076680869127) -- (-4.355000000000016,0.07184525295251544);
    \draw[line width=1.pt] (-4.355000000000016,0.07184525295251544) -- (-4.322500000000017,0.07236708546058022);
    \draw[line width=1.pt] (-4.322500000000017,0.07236708546058022) -- (-4.290000000000017,0.07289641994680197);
    \draw[line width=1.pt] (-4.290000000000017,0.07289641994680197) -- (-4.257500000000017,0.07343341636572799);
    \draw[line width=1.pt] (-4.257500000000017,0.07343341636572799) -- (-4.225000000000017,0.07397823916170666);
    \draw[line width=1.pt] (-4.225000000000017,0.07397823916170666) -- (-4.192500000000018,0.07453105742410314);
    \draw[line width=1.pt] (-4.192500000000018,0.07453105742410314) -- (-4.160000000000018,0.07509204504884032);
    \draw[line width=1.pt] (-4.160000000000018,0.07509204504884032) -- (-4.127500000000018,0.07566138090655965);
    \draw[line width=1.pt] (-4.127500000000018,0.07566138090655965) -- (-4.095000000000018,0.07623924901770969);
    \draw[line width=1.pt] (-4.095000000000018,0.07623924901770969) -- (-4.062500000000019,0.0768258387348894);
    \draw[line width=1.pt] (-4.062500000000019,0.0768258387348894) -- (-4.030000000000019,0.07742134493278692);
    \draw[line width=1.pt] (-4.030000000000019,0.07742134493278692) -- (-3.9975000000000187,0.07802596820607506);
    \draw[line width=1.pt] (-3.9975000000000187,0.07802596820607506) -- (-3.9650000000000185,0.07863991507564107);
    \draw[line width=1.pt] (-3.9650000000000185,0.07863991507564107) -- (-3.9325000000000183,0.07926339820355147);
    \draw[line width=1.pt] (-3.9325000000000183,0.07926339820355147) -- (-3.900000000000018,0.07989663661716967);
    \draw[line width=1.pt] (-3.900000000000018,0.07989663661716967) -- (-3.867500000000018,0.08053985594287048);
    \draw[line width=1.pt] (-3.867500000000018,0.08053985594287048) -- (-3.8350000000000177,0.0811932886498157);
    \draw[line width=1.pt] (-3.8350000000000177,0.0811932886498157) -- (-3.8025000000000175,0.08185717430428206);
    \draw[line width=1.pt] (-3.8025000000000175,0.08185717430428206) -- (-3.7700000000000173,0.0825317598350581);
    \draw[line width=1.pt] (-3.7700000000000173,0.0825317598350581) -- (-3.737500000000017,0.08321729981045403);
    \draw[line width=1.pt] (-3.737500000000017,0.08321729981045403) -- (-3.705000000000017,0.08391405672749871);
    \draw[line width=1.pt] (-3.705000000000017,0.08391405672749871) -- (-3.6725000000000168,0.08462230131392824);
    \draw[line width=1.pt] (-3.6725000000000168,0.08462230131392824) -- (-3.6400000000000166,0.08534231284360329);
    \draw[line width=1.pt] (-3.6400000000000166,0.08534231284360329) -- (-3.6075000000000164,0.08607437946602696);
    \draw[line width=1.pt] (-3.6075000000000164,0.08607437946602696) -- (-3.575000000000016,0.08681879855067216);
    \draw[line width=1.pt] (-3.575000000000016,0.08681879855067216) -- (-3.542500000000016,0.08757587704686474);
    \draw[line width=1.pt] (-3.542500000000016,0.08757587704686474) -- (-3.5100000000000158,0.0883459318600096);
    \draw[line width=1.pt] (-3.5100000000000158,0.0883459318600096) -- (-3.4775000000000156,0.08912929024499207);
    \draw[line width=1.pt] (-3.4775000000000156,0.08912929024499207) -- (-3.4450000000000154,0.08992629021762782);
    \draw[line width=1.pt] (-3.4450000000000154,0.08992629021762782) -- (-3.412500000000015,0.0907372809850883);
    \draw[line width=1.pt] (-3.412500000000015,0.0907372809850883) -- (-3.380000000000015,0.09156262339627315);
    \draw[line width=1.pt] (-3.380000000000015,0.09156262339627315) -- (-3.347500000000015,0.09240269041315931);
    \draw[line width=1.pt] (-3.347500000000015,0.09240269041315931) -- (-3.3150000000000146,0.09325786760421051);
    \draw[line width=1.pt] (-3.3150000000000146,0.09325786760421051) -- (-3.2825000000000144,0.09412855366098939);
    \draw[line width=1.pt] (-3.2825000000000144,0.09412855366098939) -- (-3.250000000000014,0.09501516093917942);
    \draw[line width=1.pt] (-3.250000000000014,0.09501516093917942) -- (-3.217500000000014,0.09591811602528744);
    \draw[line width=1.pt] (-3.217500000000014,0.09591811602528744) -- (-3.185000000000014,0.09683786033036722);
    \draw[line width=1.pt] (-3.185000000000014,0.09683786033036722) -- (-3.1525000000000136,0.09777485071217837);
    \draw[line width=1.pt] (-3.1525000000000136,0.09777485071217837) -- (-3.1200000000000134,0.09872956012727047);
    \draw[line width=1.pt] (-3.1200000000000134,0.09872956012727047) -- (-3.0875000000000132,0.0997024783145621);
    \draw[line width=1.pt] (-3.0875000000000132,0.0997024783145621) -- (-3.055000000000013,0.10069411251207015);
    \draw[line width=1.pt] (-3.055000000000013,0.10069411251207015) -- (-3.022500000000013,0.10170498820853187);
    \draw[line width=1.pt] (-3.022500000000013,0.10170498820853187) -- (-2.9900000000000126,0.10273564993175398);
    \draw[line width=1.pt] (-2.9900000000000126,0.10273564993175398) -- (-2.9575000000000125,0.10378666207562122);
    \draw[line width=1.pt] (-2.9575000000000125,0.10378666207562122) -- (-2.9250000000000123,0.10485860976779578);
    \draw[line width=1.pt] (-2.9250000000000123,0.10485860976779578) -- (-2.892500000000012,0.10595209978024361);
    \draw[line width=1.pt] (-2.892500000000012,0.10595209978024361) -- (-2.860000000000012,0.10706776148483213);
    \draw[line width=1.pt] (-2.860000000000012,0.10706776148483213) -- (-2.8275000000000117,0.10820624785635825);
    \draw[line width=1.pt] (-2.8275000000000117,0.10820624785635825) -- (-2.7950000000000115,0.10936823652547811);
    \draw[line width=1.pt] (-2.7950000000000115,0.10936823652547811) -- (-2.7625000000000113,0.11055443088413179);
    \draw[line width=1.pt] (-2.7625000000000113,0.11055443088413179) -- (-2.730000000000011,0.11176556124617876);
    \draw[line width=1.pt] (-2.730000000000011,0.11176556124617876) -- (-2.697500000000011,0.11300238606608273);
    \draw[line width=1.pt] (-2.697500000000011,0.11300238606608273) -- (-2.6650000000000107,0.11426569321861227);
    \draw[line width=1.pt] (-2.6650000000000107,0.11426569321861227) -- (-2.6325000000000105,0.11555630134265193);
    \draw[line width=1.pt] (-2.6325000000000105,0.11555630134265193) -- (-2.6000000000000103,0.11687506125234298);
    \draw[line width=1.pt] (-2.6000000000000103,0.11687506125234298) -- (-2.56750000000001,0.11822285741890255);
    \draw[line width=1.pt] (-2.56750000000001,0.11822285741890255) -- (-2.53500000000001,0.11960060952658891);
    \draw[line width=1.pt] (-2.53500000000001,0.11960060952658891) -- (-2.5025000000000097,0.1210092741064022);
    \draw[line width=1.pt] (-2.5025000000000097,0.1210092741064022) -- (-2.4700000000000095,0.12244984625121808);
    \draw[line width=1.pt] (-2.4700000000000095,0.12244984625121808) -- (-2.4375000000000093,0.12392336141615312);
    \draw[line width=1.pt] (-2.4375000000000093,0.12392336141615312) -- (-2.405000000000009,0.12543089730804996);
    \draw[line width=1.pt] (-2.405000000000009,0.12543089730804996) -- (-2.372500000000009,0.12697357586803804);
    \draw[line width=1.pt] (-2.372500000000009,0.12697357586803804) -- (-2.3400000000000087,0.12855256535117565);
    \draw[line width=1.pt] (-2.3400000000000087,0.12855256535117565) -- (-2.3075000000000085,0.13016908250720063);
    \draw[line width=1.pt] (-2.3075000000000085,0.13016908250720063) -- (-2.2750000000000083,0.13182439486640254);
    \draw[line width=1.pt] (-2.2750000000000083,0.13182439486640254) -- (-2.242500000000008,0.13351982313457694);
    \draw[line width=1.pt] (-2.242500000000008,0.13351982313457694) -- (-2.210000000000008,0.13525674370091573);
    \draw[line width=1.pt] (-2.210000000000008,0.13525674370091573) -- (-2.1775000000000078,0.13703659126252343);
    \draw[line width=1.pt] (-2.1775000000000078,0.13703659126252343) -- (-2.1450000000000076,0.1388608615690094);
    \draw[line width=1.pt] (-2.1450000000000076,0.1388608615690094) -- (-2.1125000000000074,0.14073111429027813);
    \draw[line width=1.pt] (-2.1125000000000074,0.14073111429027813) -- (-2.080000000000007,0.14264897601021187);
    \draw[line width=1.pt] (-2.080000000000007,0.14264897601021187) -- (-2.047500000000007,0.14461614334838122);
    \draw[line width=1.pt] (-2.047500000000007,0.14461614334838122) -- (-2.015000000000007,0.14663438621122007);
    \draw[line width=1.pt] (-2.015000000000007,0.14663438621122007) -- (-1.9825000000000068,0.1487055511732277);
    \draw[line width=1.pt] (-1.9825000000000068,0.1487055511732277) -- (-1.9500000000000068,0.15083156498768382);
    \draw[line width=1.pt] (-1.9500000000000068,0.15083156498768382) -- (-1.9175000000000069,0.15301443822504868);
    \draw[line width=1.pt] (-1.9175000000000069,0.15301443822504868) -- (-1.885000000000007,0.155256269035631);
    \draw[line width=1.pt] (-1.885000000000007,0.155256269035631) -- (-1.852500000000007,0.15755924703119067);
    \draw[line width=1.pt] (-1.852500000000007,0.15755924703119067) -- (-1.820000000000007,0.15992565727785527);
    \draw[line width=1.pt] (-1.820000000000007,0.15992565727785527) -- (-1.787500000000007,0.16235788439000454);
    \draw[line width=1.pt] (-1.787500000000007,0.16235788439000454) -- (-1.755000000000007,0.1648584167115492);
    \draw[line width=1.pt] (-1.755000000000007,0.1648584167115492) -- (-1.722500000000007,0.16742985056722295);
    \draw[line width=1.pt] (-1.722500000000007,0.16742985056722295) -- (-1.690000000000007,0.17007489456203204);
    \draw[line width=1.pt] (-1.690000000000007,0.17007489456203204) -- (-1.657500000000007,0.17279637390176827);
    \draw[line width=1.pt] (-1.657500000000007,0.17279637390176827) -- (-1.625000000000007,0.17559723470138214);
    \draw[line width=1.pt] (-1.625000000000007,0.17559723470138214) -- (-1.5925000000000071,0.17848054824090834);
    \draw[line width=1.pt] (-1.5925000000000071,0.17848054824090834) -- (-1.5600000000000072,0.18144951512040935);
    \draw[line width=1.pt] (-1.5600000000000072,0.18144951512040935) -- (-1.5275000000000072,0.18450746925590955);
    \draw[line width=1.pt] (-1.5275000000000072,0.18450746925590955) -- (-1.4950000000000072,0.1876578816473728);
    \draw[line width=1.pt] (-1.4950000000000072,0.1876578816473728) -- (-1.4625000000000072,0.19090436383727663);
    \draw[line width=1.pt] (-1.4625000000000072,0.19090436383727663) -- (-1.4300000000000073,0.19425067096407456);
    \draw[line width=1.pt] (-1.4300000000000073,0.19425067096407456) -- (-1.3975000000000073,0.1977007042986534);
    \draw[line width=1.pt] (-1.3975000000000073,0.1977007042986534) -- (-1.3650000000000073,0.20125851313360485);
    \draw[line width=1.pt] (-1.3650000000000073,0.20125851313360485) -- (-1.3325000000000073,0.20492829587459083);
    \draw[line width=1.pt] (-1.3325000000000073,0.20492829587459083) -- (-1.3000000000000074,0.20871440016015186);
    \draw[line width=1.pt] (-1.3000000000000074,0.20871440016015186) -- (-1.2675000000000074,0.21262132181089022);
    \draw[line width=1.pt] (-1.2675000000000074,0.21262132181089022) -- (-1.2350000000000074,0.21665370238101986);
    \draw[line width=1.pt] (-1.2350000000000074,0.21665370238101986) -- (-1.2025000000000075,0.22081632505485335);
    \draw[line width=1.pt] (-1.2025000000000075,0.22081632505485335) -- (-1.1700000000000075,0.225114108598053);
    \draw[line width=1.pt] (-1.1700000000000075,0.225114108598053) -- (-1.1375000000000075,0.22955209903871868);
    \draw[line width=1.pt] (-1.1375000000000075,0.22955209903871868) -- (-1.1050000000000075,0.23413545871713076);
    \draw[line width=1.pt] (-1.1050000000000075,0.23413545871713076) -- (-1.0725000000000076,0.2388694523059761);
    \draw[line width=1.pt] (-1.0725000000000076,0.2388694523059761) -- (-1.0400000000000076,0.24375942936624645);
    \draw[line width=1.pt] (-1.0400000000000076,0.24375942936624645) -- (-1.0075000000000076,0.24881080296918975);
    \draw[line width=1.pt] (-1.0075000000000076,0.24881080296918975) -- (-0.9750000000000076,0.2540290238836613);
    \draw[line width=1.pt] (-0.9750000000000076,0.2540290238836613) -- (-0.9425000000000077,0.25941954980347026);
    \draw[line width=1.pt] (-0.9425000000000077,0.25941954980347026) -- (-0.9100000000000077,0.2649878090739781);
    \draw[line width=1.pt] (-0.9100000000000077,0.2649878090739781) -- (-0.8775000000000077,0.2707391583751064);
    \draw[line width=1.pt] (-0.8775000000000077,0.2707391583751064) -- (-0.8450000000000077,0.27667883383363345);
    \draw[line width=1.pt] (-0.8450000000000077,0.27667883383363345) -- (-0.8125000000000078,0.2828118950765013);
    \draw[line width=1.pt] (-0.8125000000000078,0.2828118950765013) -- (-0.7800000000000078,0.28914316180481786);
    \draw[line width=1.pt] (-0.7800000000000078,0.28914316180481786) -- (-0.7475000000000078,0.29567714257184086);
    \draw[line width=1.pt] (-0.7475000000000078,0.29567714257184086) -- (-0.7150000000000079,0.30241795559424745);
    \draw[line width=1.pt] (-0.7150000000000079,0.30241795559424745) -- (-0.6825000000000079,0.3093692416210491);
    \draw[line width=1.pt] (-0.6825000000000079,0.3093692416210491) -- (-0.6500000000000079,0.31653406913446136);
    \draw[line width=1.pt] (-0.6500000000000079,0.31653406913446136) -- (-0.6175000000000079,0.3239148324661803);
    \draw[line width=1.pt] (-0.6175000000000079,0.3239148324661803) -- (-0.585000000000008,0.33151314378258784);
    \draw[line width=1.pt] (-0.585000000000008,0.33151314378258784) -- (-0.552500000000008,0.33932972032137837);
    \draw[line width=1.pt] (-0.552500000000008,0.33932972032137837) -- (-0.520000000000008,0.3473642687429029);
    \draw[line width=1.pt] (-0.520000000000008,0.3473642687429029) -- (-0.48750000000000804,0.35561536897870344);
    \draw[line width=1.pt] (-0.48750000000000804,0.35561536897870344) -- (-0.45500000000000806,0.3640803604963159);
    \draw[line width=1.pt] (-0.45500000000000806,0.3640803604963159) -- (-0.4225000000000081,0.3727552344242372);
    \draw[line width=1.pt] (-0.4225000000000081,0.3727552344242372) -- (-0.3900000000000081,0.3816345354565046);
    \draw[line width=1.pt] (-0.3900000000000081,0.3816345354565046) -- (-0.35750000000000814,0.3907112778377724);
    \draw[line width=1.pt] (-0.35750000000000814,0.3907112778377724) -- (-0.32500000000000817,0.3999768799671456);
    \draw[line width=1.pt] (-0.32500000000000817,0.3999768799671456) -- (-0.2925000000000082,0.4094211222009292);
    \draw[line width=1.pt] (-0.2925000000000082,0.4094211222009292) -- (-0.2600000000000082,0.41903213223311564);
    \draw[line width=1.pt] (-0.2600000000000082,0.41903213223311564) -- (-0.22750000000000822,0.4287964019500118);
    \draw[line width=1.pt] (-0.22750000000000822,0.4287964019500118) -- (-0.19500000000000822,0.4386988388708294);
    \draw[line width=1.pt] (-0.19500000000000822,0.4386988388708294) -- (-0.16250000000000822,0.44872285420151015);
    \draw[line width=1.pt] (-0.16250000000000822,0.44872285420151015) -- (-0.13000000000000822,0.45885048817502394);
    \draw[line width=1.pt] (-0.13000000000000822,0.45885048817502394) -- (-0.09750000000000822,0.4690625717890161);
    \draw[line width=1.pt] (-0.09750000000000822,0.4690625717890161) -- (-0.06500000000000822,0.4793389223713416);
    \draw[line width=1.pt] (-0.06500000000000822,0.4793389223713416) -- (-0.03250000000000822,0.48965856871962665);
    \draw[line width=1.pt] (-0.03250000000000822,0.48965856871962665) -- (0.0,0.4999999999999974);
    \draw[line width=1.pt] (0.0,0.4999999999999974) -- (0.032499999999991785,0.5103414312803681);
    \draw[line width=1.pt] (0.032499999999991785,0.5103414312803681) -- (0.06499999999999179,0.5206610776286532);
    \draw[line width=1.pt] (0.06499999999999179,0.5206610776286532) -- (0.09749999999999179,0.5309374282109788);
    \draw[line width=1.pt] (0.09749999999999179,0.5309374282109788) -- (0.1299999999999918,0.541149511824971);
    \draw[line width=1.pt] (0.1299999999999918,0.541149511824971) -- (0.1624999999999918,0.5512771457984847);
    \draw[line width=1.pt] (0.1624999999999918,0.5512771457984847) -- (0.1949999999999918,0.5613011611291655);
    \draw[line width=1.pt] (0.1949999999999918,0.5613011611291655) -- (0.2274999999999918,0.5712035980499833);
    \draw[line width=1.pt] (0.2274999999999918,0.5712035980499833) -- (0.2599999999999918,0.5809678677668795);
    \draw[line width=1.pt] (0.2599999999999918,0.5809678677668795) -- (0.29249999999999177,0.590578877799066);
    \draw[line width=1.pt] (0.29249999999999177,0.590578877799066) -- (0.32499999999999174,0.6000231200328497);
    \draw[line width=1.pt] (0.32499999999999174,0.6000231200328497) -- (0.3574999999999917,0.609288722162223);
    \draw[line width=1.pt] (0.3574999999999917,0.609288722162223) -- (0.3899999999999917,0.618365464543491);
    \draw[line width=1.pt] (0.3899999999999917,0.618365464543491) -- (0.42249999999999166,0.6272447655757584);
    \draw[line width=1.pt] (0.42249999999999166,0.6272447655757584) -- (0.45499999999999163,0.6359196395036797);
    \draw[line width=1.pt] (0.45499999999999163,0.6359196395036797) -- (0.4874999999999916,0.6443846310212924);
    \draw[line width=1.pt] (0.4874999999999916,0.6443846310212924) -- (0.5199999999999916,0.6526357312570931);
    \draw[line width=1.pt] (0.5199999999999916,0.6526357312570931) -- (0.5524999999999916,0.6606702796786177);
    \draw[line width=1.pt] (0.5524999999999916,0.6606702796786177) -- (0.5849999999999915,0.6684868562174084);
    \draw[line width=1.pt] (0.5849999999999915,0.6684868562174084) -- (0.6174999999999915,0.6760851675338161);
    \draw[line width=1.pt] (0.6174999999999915,0.6760851675338161) -- (0.6499999999999915,0.6834659308655351);
    \draw[line width=1.pt] (0.6499999999999915,0.6834659308655351) -- (0.6824999999999914,0.6906307583789474);
    \draw[line width=1.pt] (0.6824999999999914,0.6906307583789474) -- (0.7149999999999914,0.6975820444057491);
    \draw[line width=1.pt] (0.7149999999999914,0.6975820444057491) -- (0.7474999999999914,0.7043228574281558);
    \draw[line width=1.pt] (0.7474999999999914,0.7043228574281558) -- (0.7799999999999914,0.710856838195179);
    \draw[line width=1.pt] (0.7799999999999914,0.710856838195179) -- (0.8124999999999913,0.7171881049234955);
    \draw[line width=1.pt] (0.8124999999999913,0.7171881049234955) -- (0.8449999999999913,0.7233211661663635);
    \draw[line width=1.pt] (0.8449999999999913,0.7233211661663635) -- (0.8774999999999913,0.7292608416248907);
    \draw[line width=1.pt] (0.8774999999999913,0.7292608416248907) -- (0.9099999999999913,0.7350121909260191);
    \draw[line width=1.pt] (0.9099999999999913,0.7350121909260191) -- (0.9424999999999912,0.740580450196527);
    \draw[line width=1.pt] (0.9424999999999912,0.740580450196527) -- (0.9749999999999912,0.745970976116336);
    \draw[line width=1.pt] (0.9749999999999912,0.745970976116336) -- (1.0074999999999912,0.7511891970308078);
    \draw[line width=1.pt] (1.0074999999999912,0.7511891970308078) -- (1.0399999999999912,0.756240570633751);
    \draw[line width=1.pt] (1.0399999999999912,0.756240570633751) -- (1.0724999999999911,0.7611305476940216);
    \draw[line width=1.pt] (1.0724999999999911,0.7611305476940216) -- (1.104999999999991,0.765864541282867);
    \draw[line width=1.pt] (1.104999999999991,0.765864541282867) -- (1.137499999999991,0.7704479009612791);
    \draw[line width=1.pt] (1.137499999999991,0.7704479009612791) -- (1.169999999999991,0.7748858914019449);
    \draw[line width=1.pt] (1.169999999999991,0.7748858914019449) -- (1.202499999999991,0.7791836749451445);
    \draw[line width=1.pt] (1.202499999999991,0.7791836749451445) -- (1.234999999999991,0.7833462976189781);
    \draw[line width=1.pt] (1.234999999999991,0.7833462976189781) -- (1.267499999999991,0.7873786781891078);
    \draw[line width=1.pt] (1.267499999999991,0.7873786781891078) -- (1.299999999999991,0.7912855998398461);
    \draw[line width=1.pt] (1.299999999999991,0.7912855998398461) -- (1.332499999999991,0.7950717041254073);
    \draw[line width=1.pt] (1.332499999999991,0.7950717041254073) -- (1.3649999999999909,0.7987414868663933);
    \draw[line width=1.pt] (1.3649999999999909,0.7987414868663933) -- (1.3974999999999909,0.8022992957013448);
    \draw[line width=1.pt] (1.3974999999999909,0.8022992957013448) -- (1.4299999999999908,0.8057493290359238);
    \draw[line width=1.pt] (1.4299999999999908,0.8057493290359238) -- (1.4624999999999908,0.8090956361627217);
    \draw[line width=1.pt] (1.4624999999999908,0.8090956361627217) -- (1.4949999999999908,0.8123421183526256);
    \draw[line width=1.pt] (1.4949999999999908,0.8123421183526256) -- (1.5274999999999908,0.8154925307440889);
    \draw[line width=1.pt] (1.5274999999999908,0.8154925307440889) -- (1.5599999999999907,0.8185504848795891);
    \draw[line width=1.pt] (1.5599999999999907,0.8185504848795891) -- (1.5924999999999907,0.8215194517590901);
    \draw[line width=1.pt] (1.5924999999999907,0.8215194517590901) -- (1.6249999999999907,0.8244027652986164);
    \draw[line width=1.pt] (1.6249999999999907,0.8244027652986164) -- (1.6574999999999906,0.8272036260982305);
    \draw[line width=1.pt] (1.6574999999999906,0.8272036260982305) -- (1.6899999999999906,0.8299251054379666);
    \draw[line width=1.pt] (1.6899999999999906,0.8299251054379666) -- (1.7224999999999906,0.8325701494327757);
    \draw[line width=1.pt] (1.7224999999999906,0.8325701494327757) -- (1.7549999999999906,0.8351415832884497);
    \draw[line width=1.pt] (1.7549999999999906,0.8351415832884497) -- (1.7874999999999905,0.8376421156099942);
    \draw[line width=1.pt] (1.7874999999999905,0.8376421156099942) -- (1.8199999999999905,0.8400743427221435);
    \draw[line width=1.pt] (1.8199999999999905,0.8400743427221435) -- (1.8524999999999905,0.8424407529688082);
    \draw[line width=1.pt] (1.8524999999999905,0.8424407529688082) -- (1.8849999999999905,0.8447437309643678);
    \draw[line width=1.pt] (1.8849999999999905,0.8447437309643678) -- (1.9174999999999904,0.8469855617749503);
    \draw[line width=1.pt] (1.9174999999999904,0.8469855617749503) -- (1.9499999999999904,0.8491684350123151);
    \draw[line width=1.pt] (1.9499999999999904,0.8491684350123151) -- (1.9824999999999904,0.8512944488267713);
    \draw[line width=1.pt] (1.9824999999999904,0.8512944488267713) -- (2.0149999999999904,0.8533656137887788);
    \draw[line width=1.pt] (2.0149999999999904,0.8533656137887788) -- (2.0474999999999905,0.8553838566516178);
    \draw[line width=1.pt] (2.0474999999999905,0.8553838566516178) -- (2.0799999999999907,0.8573510239897871);
    \draw[line width=1.pt] (2.0799999999999907,0.8573510239897871) -- (2.112499999999991,0.859268885709721);
    \draw[line width=1.pt] (2.112499999999991,0.859268885709721) -- (2.144999999999991,0.8611391384309897);
    \draw[line width=1.pt] (2.144999999999991,0.8611391384309897) -- (2.1774999999999913,0.8629634087374757);
    \draw[line width=1.pt] (2.1774999999999913,0.8629634087374757) -- (2.2099999999999915,0.8647432562990834);
    \draw[line width=1.pt] (2.2099999999999915,0.8647432562990834) -- (2.2424999999999917,0.8664801768654223);
    \draw[line width=1.pt] (2.2424999999999917,0.8664801768654223) -- (2.274999999999992,0.8681756051335966);
    \draw[line width=1.pt] (2.274999999999992,0.8681756051335966) -- (2.307499999999992,0.8698309174927986);
    \draw[line width=1.pt] (2.307499999999992,0.8698309174927986) -- (2.3399999999999923,0.8714474346488235);
    \draw[line width=1.pt] (2.3399999999999923,0.8714474346488235) -- (2.3724999999999925,0.8730264241319612);
    \draw[line width=1.pt] (2.3724999999999925,0.8730264241319612) -- (2.4049999999999927,0.8745691026919493);
    \draw[line width=1.pt] (2.4049999999999927,0.8745691026919493) -- (2.437499999999993,0.8760766385838462);
    \draw[line width=1.pt] (2.437499999999993,0.8760766385838462) -- (2.469999999999993,0.8775501537487812);
    \draw[line width=1.pt] (2.469999999999993,0.8775501537487812) -- (2.5024999999999933,0.878990725893597);
    \draw[line width=1.pt] (2.5024999999999933,0.878990725893597) -- (2.5349999999999935,0.8803993904734105);
    \draw[line width=1.pt] (2.5349999999999935,0.8803993904734105) -- (2.5674999999999937,0.8817771425810967);
    \draw[line width=1.pt] (2.5674999999999937,0.8817771425810967) -- (2.599999999999994,0.8831249387476563);
    \draw[line width=1.pt] (2.599999999999994,0.8831249387476563) -- (2.632499999999994,0.8844436986573474);
    \draw[line width=1.pt] (2.632499999999994,0.8844436986573474) -- (2.6649999999999943,0.8857343067813871);
    \draw[line width=1.pt] (2.6649999999999943,0.8857343067813871) -- (2.6974999999999945,0.8869976139339167);
    \draw[line width=1.pt] (2.6974999999999945,0.8869976139339167) -- (2.7299999999999947,0.8882344387538207);
    \draw[line width=1.pt] (2.7299999999999947,0.8882344387538207) -- (2.762499999999995,0.8894455691158677);
    \draw[line width=1.pt] (2.762499999999995,0.8894455691158677) -- (2.794999999999995,0.8906317634745213);
    \draw[line width=1.pt] (2.794999999999995,0.8906317634745213) -- (2.8274999999999952,0.8917937521436412);
    \draw[line width=1.pt] (2.8274999999999952,0.8917937521436412) -- (2.8599999999999954,0.8929322385151673);
    \draw[line width=1.pt] (2.8599999999999954,0.8929322385151673) -- (2.8924999999999956,0.8940479002197558);
    \draw[line width=1.pt] (2.8924999999999956,0.8940479002197558) -- (2.924999999999996,0.8951413902322036);
    \draw[line width=1.pt] (2.924999999999996,0.8951413902322036) -- (2.957499999999996,0.8962133379243782);
    \draw[line width=1.pt] (2.957499999999996,0.8962133379243782) -- (2.989999999999996,0.8972643500682455);
    \draw[line width=1.pt] (2.989999999999996,0.8972643500682455) -- (3.0224999999999964,0.8982950117914676);
    \draw[line width=1.pt] (3.0224999999999964,0.8982950117914676) -- (3.0549999999999966,0.8993058874879293);
    \draw[line width=1.pt] (3.0549999999999966,0.8993058874879293) -- (3.087499999999997,0.9002975216854374);
    \draw[line width=1.pt] (3.087499999999997,0.9002975216854374) -- (3.119999999999997,0.901270439872729);
    \draw[line width=1.pt] (3.119999999999997,0.901270439872729) -- (3.152499999999997,0.9022251492878213);
    \draw[line width=1.pt] (3.152499999999997,0.9022251492878213) -- (3.1849999999999974,0.9031621396696324);
    \draw[line width=1.pt] (3.1849999999999974,0.9031621396696324) -- (3.2174999999999976,0.9040818839747121);
    \draw[line width=1.pt] (3.2174999999999976,0.9040818839747121) -- (3.249999999999998,0.9049848390608202);
    \draw[line width=1.pt] (3.249999999999998,0.9049848390608202) -- (3.282499999999998,0.9058714463390102);
    \draw[line width=1.pt] (3.282499999999998,0.9058714463390102) -- (3.314999999999998,0.906742132395789);
    \draw[line width=1.pt] (3.314999999999998,0.906742132395789) -- (3.3474999999999984,0.9075973095868403);
    \draw[line width=1.pt] (3.3474999999999984,0.9075973095868403) -- (3.3799999999999986,0.9084373766037265);
    \draw[line width=1.pt] (3.3799999999999986,0.9084373766037265) -- (3.4124999999999988,0.9092627190149113);
    \draw[line width=1.pt] (3.4124999999999988,0.9092627190149113) -- (3.444999999999999,0.9100737097823717);
    \draw[line width=1.pt] (3.444999999999999,0.9100737097823717) -- (3.477499999999999,0.9108707097550075);
    \draw[line width=1.pt] (3.477499999999999,0.9108707097550075) -- (3.5099999999999993,0.9116540681399901);
    \draw[line width=1.pt] (3.5099999999999993,0.9116540681399901) -- (3.5424999999999995,0.9124241229531349);
    \draw[line width=1.pt] (3.5424999999999995,0.9124241229531349) -- (3.5749999999999997,0.9131812014493275);
    \draw[line width=1.pt] (3.5749999999999997,0.9131812014493275) -- (3.6075,0.9139256205339727);
    \draw[line width=1.pt] (3.6075,0.9139256205339727) -- (3.64,0.9146576871563964);
    \draw[line width=1.pt] (3.64,0.9146576871563964) -- (3.6725000000000003,0.9153776986860713);
    \draw[line width=1.pt] (3.6725000000000003,0.9153776986860713) -- (3.7050000000000005,0.9160859432725009);
    \draw[line width=1.pt] (3.7050000000000005,0.9160859432725009) -- (3.7375000000000007,0.9167827001895457);
    \draw[line width=1.pt] (3.7375000000000007,0.9167827001895457) -- (3.770000000000001,0.9174682401649416);
    \draw[line width=1.pt] (3.770000000000001,0.9174682401649416) -- (3.802500000000001,0.9181428256957176);
    \draw[line width=1.pt] (3.802500000000001,0.9181428256957176) -- (3.8350000000000013,0.918806711350184);
    \draw[line width=1.pt] (3.8350000000000013,0.918806711350184) -- (3.8675000000000015,0.9194601440571292);
    \draw[line width=1.pt] (3.8675000000000015,0.9194601440571292) -- (3.9000000000000017,0.92010336338283);
    \draw[line width=1.pt] (3.9000000000000017,0.92010336338283) -- (3.932500000000002,0.9207366017964482);
    \draw[line width=1.pt] (3.932500000000002,0.9207366017964482) -- (3.965000000000002,0.9213600849243587);
    \draw[line width=1.pt] (3.965000000000002,0.9213600849243587) -- (3.9975000000000023,0.9219740317939247);
    \draw[line width=1.pt] (3.9975000000000023,0.9219740317939247) -- (4.030000000000002,0.9225786550672127);
    \draw[line width=1.pt] (4.030000000000002,0.9225786550672127) -- (4.062500000000002,0.9231741612651102);
    \draw[line width=1.pt] (4.062500000000002,0.9231741612651102) -- (4.0950000000000015,0.92376075098229);
    \draw[line width=1.pt] (4.0950000000000015,0.92376075098229) -- (4.127500000000001,0.9243386190934401);
    \draw[line width=1.pt] (4.127500000000001,0.9243386190934401) -- (4.160000000000001,0.9249079549511593);
    \draw[line width=1.pt] (4.160000000000001,0.9249079549511593) -- (4.192500000000001,0.9254689425758966);
    \draw[line width=1.pt] (4.192500000000001,0.9254689425758966) -- (4.2250000000000005,0.9260217608382931);
    \draw[line width=1.pt] (4.2250000000000005,0.9260217608382931) -- (4.2575,0.9265665836342718);
    \draw[line width=1.pt] (4.2575,0.9265665836342718) -- (4.29,0.9271035800531977);
    \draw[line width=1.pt] (4.29,0.9271035800531977) -- (4.3225,0.9276329145394195);
    \draw[line width=1.pt] (4.3225,0.9276329145394195) -- (4.3549999999999995,0.9281547470474842);
    \draw[line width=1.pt] (4.3549999999999995,0.9281547470474842) -- (4.387499999999999,0.9286692331913085);
    \draw[line width=1.pt] (4.387499999999999,0.9286692331913085) -- (4.419999999999999,0.9291765243875686);
    \draw[line width=1.pt] (4.419999999999999,0.9291765243875686) -- (4.452499999999999,0.9296767679935658);
    \draw[line width=1.pt] (4.452499999999999,0.9296767679935658) -- (4.4849999999999985,0.9301701074398041);
    \draw[line width=1.pt] (4.4849999999999985,0.9301701074398041) -- (4.517499999999998,0.9306566823575149);
    \draw[line width=1.pt] (4.517499999999998,0.9306566823575149) -- (4.549999999999998,0.931136628701342);
    \draw[line width=1.pt] (4.549999999999998,0.931136628701342) -- (4.582499999999998,0.9316100788673961);
    \draw[line width=1.pt] (4.582499999999998,0.9316100788673961) -- (4.6149999999999975,0.9320771618068812);
    \draw[line width=1.pt] (4.6149999999999975,0.9320771618068812) -- (4.647499999999997,0.9325380031354736);
    \draw[line width=1.pt] (4.647499999999997,0.9325380031354736) -- (4.679999999999997,0.9329927252386416);
    \draw[line width=1.pt] (4.679999999999997,0.9329927252386416) -- (4.712499999999997,0.9334414473730706);
    \draw[line width=1.pt] (4.712499999999997,0.9334414473730706) -- (4.7449999999999966,0.9338842857643602);
    \draw[line width=1.pt] (4.7449999999999966,0.9338842857643602) -- (4.777499999999996,0.9343213537011507);
    \draw[line width=1.pt] (4.777499999999996,0.9343213537011507) -- (4.809999999999996,0.9347527616258215);
    \draw[line width=1.pt] (4.809999999999996,0.9347527616258215) -- (4.842499999999996,0.9351786172219103);
    \draw[line width=1.pt] (4.842499999999996,0.9351786172219103) -- (4.874999999999996,0.935599025498383);
    \draw[line width=1.pt] (4.874999999999996,0.935599025498383) -- (4.907499999999995,0.9360140888708844);
    \draw[line width=1.pt] (4.907499999999995,0.9360140888708844) -- (4.939999999999995,0.9364239072400966);
    \draw[line width=1.pt] (4.939999999999995,0.9364239072400966) -- (4.972499999999995,0.9368285780673147);
    \draw[line width=1.pt] (4.972499999999995,0.9368285780673147) -- (5.004999999999995,0.9372281964473614);
    \draw[line width=1.pt] (5.004999999999995,0.9372281964473614) -- (5.037499999999994,0.9376228551789394);
    \draw[line width=1.pt] (5.037499999999994,0.9376228551789394) -- (5.069999999999994,0.9380126448325298);
    \draw[line width=1.pt] (5.069999999999994,0.9380126448325298) -- (5.102499999999994,0.9383976538159307);
    \draw[line width=1.pt] (5.102499999999994,0.9383976538159307) -- (5.134999999999994,0.9387779684375309);
    \draw[line width=1.pt] (5.134999999999994,0.9387779684375309) -- (5.167499999999993,0.939153672967409);
    \draw[line width=1.pt] (5.167499999999993,0.939153672967409) -- (5.199999999999993,0.9395248496963403);
    \draw[line width=1.pt] (5.199999999999993,0.9395248496963403) -- (5.232499999999993,0.9398915789927967);
    \draw[line width=1.pt] (5.232499999999993,0.9398915789927967) -- (5.264999999999993,0.9402539393580146);
    \draw[line width=1.pt] (5.264999999999993,0.9402539393580146) -- (5.297499999999992,0.9406120074792079);
    \draw[line width=1.pt] (5.297499999999992,0.9406120074792079) -- (5.329999999999992,0.9409658582809963);
    \draw[line width=1.pt] (5.329999999999992,0.9409658582809963) -- (5.362499999999992,0.9413155649751167);
    \draw[line width=1.pt] (5.362499999999992,0.9413155649751167) -- (5.394999999999992,0.9416611991084849);
    \draw[line width=1.pt] (5.394999999999992,0.9416611991084849) -- (5.427499999999991,0.9420028306096698);
    \draw[line width=1.pt] (5.427499999999991,0.9420028306096698) -- (5.459999999999991,0.9423405278338385);
    \draw[line width=1.pt] (5.459999999999991,0.9423405278338385) -- (5.492499999999991,0.9426743576062309);
    \draw[line width=1.pt] (5.492499999999991,0.9426743576062309) -- (5.524999999999991,0.9430043852642199);
    \draw[line width=1.pt] (5.524999999999991,0.9430043852642199) -- (5.55749999999999,0.9433306746980094);
    \draw[line width=1.pt] (5.55749999999999,0.9433306746980094) -- (5.58999999999999,0.9436532883900168);
    \draw[line width=1.pt] (5.58999999999999,0.9436532883900168) -- (5.62249999999999,0.9439722874529973);
    \draw[line width=1.pt] (5.62249999999999,0.9439722874529973) -- (5.65499999999999,0.9442877316669454);
    \draw[line width=1.pt] (5.65499999999999,0.9442877316669454) -- (5.687499999999989,0.9445996795148273);
    \draw[line width=1.pt] (5.687499999999989,0.9445996795148273) -- (5.719999999999989,0.9449081882171818);
    \draw[line width=1.pt] (5.719999999999989,0.9449081882171818) -- (5.752499999999989,0.9452133137656333);
    \draw[line width=1.pt] (5.752499999999989,0.9452133137656333) -- (5.784999999999989,0.9455151109553528);
    \draw[line width=1.pt] (5.784999999999989,0.9455151109553528) -- (5.817499999999988,0.9458136334165098);
    \draw[line width=1.pt] (5.817499999999988,0.9458136334165098) -- (5.849999999999988,0.9461089336447454);
    \draw[line width=1.pt] (5.849999999999988,0.9461089336447454) -- (5.882499999999988,0.9464010630307069);
    \draw[line width=1.pt] (5.882499999999988,0.9464010630307069) -- (5.914999999999988,0.9466900718886722);
    \draw[line width=1.pt] (5.914999999999988,0.9466900718886722) -- (5.947499999999987,0.9469760094842996);
    \draw[line width=1.pt] (5.947499999999987,0.9469760094842996) -- (5.979999999999987,0.9472589240615339);
    \draw[line width=1.pt] (5.979999999999987,0.9472589240615339) -- (6.012499999999987,0.9475388628686935);
    \draw[line width=1.pt] (6.012499999999987,0.9475388628686935) -- (6.044999999999987,0.9478158721837736);
    \draw[line width=1.pt] (6.044999999999987,0.9478158721837736) -- (6.077499999999986,0.9480899973389885);
    \draw[line width=1.pt] (6.077499999999986,0.9480899973389885) -- (6.109999999999986,0.948361282744579);
    \draw[line width=1.pt] (6.109999999999986,0.948361282744579) -- (6.142499999999986,0.9486297719119134);
    \draw[line width=1.pt] (6.142499999999986,0.9486297719119134) -- (6.174999999999986,0.9488955074759033);
    \draw[line width=1.pt] (6.174999999999986,0.9488955074759033) -- (6.207499999999985,0.9491585312167576);
    \draw[line width=1.pt] (6.207499999999985,0.9491585312167576) -- (6.239999999999985,0.9494188840810989);
    \draw[line width=1.pt] (6.239999999999985,0.9494188840810989) -- (6.272499999999985,0.9496766062024633);
    \draw[line width=1.pt] (6.272499999999985,0.9496766062024633) -- (6.304999999999985,0.9499317369212027);
    \draw[line width=1.pt] (6.304999999999985,0.9499317369212027) -- (6.337499999999984,0.950184314803811);
    \draw[line width=1.pt] (6.337499999999984,0.950184314803811) -- (6.369999999999984,0.9504343776616947);
    \draw[line width=1.pt] (6.369999999999984,0.9504343776616947) -- (6.402499999999984,0.9506819625694015);
    \draw[line width=1.pt] (6.402499999999984,0.9506819625694015) -- (6.434999999999984,0.9509271058823316);
    \draw[line width=1.pt] (6.434999999999984,0.9509271058823316) -- (6.467499999999983,0.9511698432539434);
    \draw[line width=1.pt] (6.467499999999983,0.9511698432539434) -- (6.499999999999983,0.951410209652471);
    \begin{scriptsize}
    \draw[color=black] (-7.774790164207696,1.1002885699081597) node {$g$};
    \draw[color=black] (-7.774790164207696,0.14645781064330685) node {$p$};
    \end{scriptsize}
    \end{axis}
\end{tikzpicture}
            \caption{Graf funkce $g_2$ ("přeškálovaný" a posunutý graf $\arctg$).}
            \label{fig:graf_g2}
        \end{figure}
        \item\label{item:funkce_tg} Stejně tak lze sestrojit bijekci z intervalu $\displaystyle\left(-\frac{\pi}{2},\frac{\pi}{2}\right)$ do $\R$. Stačí vzít inverzní funkci ke $g_2$, tj. $\displaystyle\map{g_3}{\left(-\frac{\pi}{2},\frac{\pi}{2}\right)}{\R}$, kde $g_3(x)=g_1^{-1}(x)=\tg{x}$.
        \item\label{item:bijekce_slozena} Bijekci můžeme sestrojit i mezi samotnými intervaly $\displaystyle\left(-\frac{\pi}{2},\frac{\pi}{2}\right)$ a $(0,1)$. Tu můžeme nalézt složením funkcí $g_3$ a $g_2$, tj. $g_4=g_3\circ g_2$, protože
        \begin{equation*}
            \map{g_3}{\left(-\frac{\pi}{2},\frac{\pi}{2}\right)}{\R}\;\text{a}\;\map{g_2}{\R}{(0,1)}
        \end{equation*}
        Tedy celkově $\displaystyle\map{g_4}{\left(-\frac{\pi}{2},\frac{\pi}{2}\right)}{(0,1)}$, kde
        \begin{equation*}
            g_4(x)=g_2(g_3(x))=\frac{1}{\pi}\left(\underbrace{\arctg{(\tg{x})}}_{\text{identita}}+\frac{\pi}{2}\right)=\frac{1}{\pi}\left(x+\frac{\pi}{2}\right)=\frac{1}{\pi}x+\frac{1}{2}.
        \end{equation*}
        \begin{figure}[H]
            \centering
            \definecolor{ffqqqq}{rgb}{1.,0.,0.}
\begin{tikzpicture}[line cap=round,line join=round,>=triangle 45,x=1.0cm,y=1.0cm]
\begin{axis}[
x=2.0cm,y=2.0cm,
axis lines=middle,
xmin=-2.3,
xmax=2.3,
ymin=-0.3,
ymax=1.3,
xtick={-2.0,-1.0,...,2.0},
ytick={-0.0,1.0,...,1.0},]
\clip(-2.3,-0.3) rectangle (2.3,1.3);
\draw[line width=1.pt] (-1.5707942000000081,0.0) -- (-1.5707942000000081,0.0);
\draw[line width=1.pt] (-1.5707942000000081,0.0) -- (-1.5629402327673967,0.0025006723957426424);
\draw[line width=1.pt] (-1.5629402327673967,0.0025006723957426424) -- (-1.5550862655347852,0.005000667811646431);
\draw[line width=1.pt] (-1.5550862655347852,0.005000667811646431) -- (-1.5472322983021738,0.007500663227550164);
\draw[line width=1.pt] (-1.5472322983021738,0.007500663227550164) -- (-1.5393783310695623,0.010000658643453952);
\draw[line width=1.pt] (-1.5393783310695623,0.010000658643453952) -- (-1.531524363836951,0.012500654059357741);
\draw[line width=1.pt] (-1.531524363836951,0.012500654059357741) -- (-1.5236703966043394,0.015000649475261474);
\draw[line width=1.pt] (-1.5236703966043394,0.015000649475261474) -- (-1.515816429371728,0.017500644891165262);
\draw[line width=1.pt] (-1.515816429371728,0.017500644891165262) -- (-1.5079624621391166,0.02000064030706905);
\draw[line width=1.pt] (-1.5079624621391166,0.02000064030706905) -- (-1.500108494906505,0.02250063572297284);
\draw[line width=1.pt] (-1.500108494906505,0.02250063572297284) -- (-1.4922545276738937,0.025000631138876572);
\draw[line width=1.pt] (-1.4922545276738937,0.025000631138876572) -- (-1.4844005604412822,0.02750062655478036);
\draw[line width=1.pt] (-1.4844005604412822,0.02750062655478036) -- (-1.4765465932086708,0.03000062197068415);
\draw[line width=1.pt] (-1.4765465932086708,0.03000062197068415) -- (-1.4686926259760593,0.03250061738658788);
\draw[line width=1.pt] (-1.4686926259760593,0.03250061738658788) -- (-1.4608386587434479,0.03500061280249167);
\draw[line width=1.pt] (-1.4608386587434479,0.03500061280249167) -- (-1.4529846915108364,0.03750060821839546);
\draw[line width=1.pt] (-1.4529846915108364,0.03750060821839546) -- (-1.445130724278225,0.04000060363429919);
\draw[line width=1.pt] (-1.445130724278225,0.04000060363429919) -- (-1.4372767570456135,0.04250059905020298);
\draw[line width=1.pt] (-1.4372767570456135,0.04250059905020298) -- (-1.429422789813002,0.04500059446610677);
\draw[line width=1.pt] (-1.429422789813002,0.04500059446610677) -- (-1.4215688225803906,0.0475005898820105);
\draw[line width=1.pt] (-1.4215688225803906,0.0475005898820105) -- (-1.4137148553477792,0.05000058529791429);
\draw[line width=1.pt] (-1.4137148553477792,0.05000058529791429) -- (-1.4058608881151677,0.05250058071381808);
\draw[line width=1.pt] (-1.4058608881151677,0.05250058071381808) -- (-1.3980069208825563,0.05500057612972181);
\draw[line width=1.pt] (-1.3980069208825563,0.05500057612972181) -- (-1.3901529536499448,0.0575005715456256);
\draw[line width=1.pt] (-1.3901529536499448,0.0575005715456256) -- (-1.3822989864173334,0.06000056696152939);
\draw[line width=1.pt] (-1.3822989864173334,0.06000056696152939) -- (-1.374445019184722,0.06250056237743318);
\draw[line width=1.pt] (-1.374445019184722,0.06250056237743318) -- (-1.3665910519521105,0.06500055779333691);
\draw[line width=1.pt] (-1.3665910519521105,0.06500055779333691) -- (-1.358737084719499,0.0675005532092407);
\draw[line width=1.pt] (-1.358737084719499,0.0675005532092407) -- (-1.3508831174868876,0.07000054862514449);
\draw[line width=1.pt] (-1.3508831174868876,0.07000054862514449) -- (-1.3430291502542762,0.07250054404104822);
\draw[line width=1.pt] (-1.3430291502542762,0.07250054404104822) -- (-1.3351751830216647,0.07500053945695201);
\draw[line width=1.pt] (-1.3351751830216647,0.07500053945695201) -- (-1.3273212157890533,0.0775005348728558);
\draw[line width=1.pt] (-1.3273212157890533,0.0775005348728558) -- (-1.3194672485564418,0.08000053028875953);
\draw[line width=1.pt] (-1.3194672485564418,0.08000053028875953) -- (-1.3116132813238304,0.08250052570466332);
\draw[line width=1.pt] (-1.3116132813238304,0.08250052570466332) -- (-1.303759314091219,0.08500052112056711);
\draw[line width=1.pt] (-1.303759314091219,0.08500052112056711) -- (-1.2959053468586075,0.08750051653647084);
\draw[line width=1.pt] (-1.2959053468586075,0.08750051653647084) -- (-1.288051379625996,0.09000051195237463);
\draw[line width=1.pt] (-1.288051379625996,0.09000051195237463) -- (-1.2801974123933846,0.09250050736827842);
\draw[line width=1.pt] (-1.2801974123933846,0.09250050736827842) -- (-1.2723434451607731,0.0950005027841822);
\draw[line width=1.pt] (-1.2723434451607731,0.0950005027841822) -- (-1.2644894779281617,0.09750049820008594);
\draw[line width=1.pt] (-1.2644894779281617,0.09750049820008594) -- (-1.2566355106955502,0.10000049361598973);
\draw[line width=1.pt] (-1.2566355106955502,0.10000049361598973) -- (-1.2487815434629388,0.10250048903189352);
\draw[line width=1.pt] (-1.2487815434629388,0.10250048903189352) -- (-1.2409275762303273,0.10500048444779725);
\draw[line width=1.pt] (-1.2409275762303273,0.10500048444779725) -- (-1.233073608997716,0.10750047986370104);
\draw[line width=1.pt] (-1.233073608997716,0.10750047986370104) -- (-1.2252196417651045,0.11000047527960483);
\draw[line width=1.pt] (-1.2252196417651045,0.11000047527960483) -- (-1.217365674532493,0.11250047069550856);
\draw[line width=1.pt] (-1.217365674532493,0.11250047069550856) -- (-1.2095117072998816,0.11500046611141235);
\draw[line width=1.pt] (-1.2095117072998816,0.11500046611141235) -- (-1.20165774006727,0.11750046152731614);
\draw[line width=1.pt] (-1.20165774006727,0.11750046152731614) -- (-1.1938037728346587,0.12000045694321987);
\draw[line width=1.pt] (-1.1938037728346587,0.12000045694321987) -- (-1.1859498056020472,0.12250045235912366);
\draw[line width=1.pt] (-1.1859498056020472,0.12250045235912366) -- (-1.1780958383694358,0.12500044777502745);
\draw[line width=1.pt] (-1.1780958383694358,0.12500044777502745) -- (-1.1702418711368243,0.12750044319093123);
\draw[line width=1.pt] (-1.1702418711368243,0.12750044319093123) -- (-1.1623879039042129,0.13000043860683497);
\draw[line width=1.pt] (-1.1623879039042129,0.13000043860683497) -- (-1.1545339366716014,0.13250043402273876);
\draw[line width=1.pt] (-1.1545339366716014,0.13250043402273876) -- (-1.14667996943899,0.13500042943864254);
\draw[line width=1.pt] (-1.14667996943899,0.13500042943864254) -- (-1.1388260022063785,0.13750042485454628);
\draw[line width=1.pt] (-1.1388260022063785,0.13750042485454628) -- (-1.130972034973767,0.14000042027045007);
\draw[line width=1.pt] (-1.130972034973767,0.14000042027045007) -- (-1.1231180677411556,0.14250041568635385);
\draw[line width=1.pt] (-1.1231180677411556,0.14250041568635385) -- (-1.1152641005085442,0.1450004111022576);
\draw[line width=1.pt] (-1.1152641005085442,0.1450004111022576) -- (-1.1074101332759327,0.14750040651816138);
\draw[line width=1.pt] (-1.1074101332759327,0.14750040651816138) -- (-1.0995561660433213,0.15000040193406516);
\draw[line width=1.pt] (-1.0995561660433213,0.15000040193406516) -- (-1.0917021988107098,0.1525003973499689);
\draw[line width=1.pt] (-1.0917021988107098,0.1525003973499689) -- (-1.0838482315780984,0.15500039276587269);
\draw[line width=1.pt] (-1.0838482315780984,0.15500039276587269) -- (-1.075994264345487,0.15750038818177647);
\draw[line width=1.pt] (-1.075994264345487,0.15750038818177647) -- (-1.0681402971128755,0.1600003835976802);
\draw[line width=1.pt] (-1.0681402971128755,0.1600003835976802) -- (-1.060286329880264,0.162500379013584);
\draw[line width=1.pt] (-1.060286329880264,0.162500379013584) -- (-1.0524323626476526,0.16500037442948778);
\draw[line width=1.pt] (-1.0524323626476526,0.16500037442948778) -- (-1.0445783954150412,0.16750036984539157);
\draw[line width=1.pt] (-1.0445783954150412,0.16750036984539157) -- (-1.0367244281824297,0.1700003652612953);
\draw[line width=1.pt] (-1.0367244281824297,0.1700003652612953) -- (-1.0288704609498183,0.1725003606771991);
\draw[line width=1.pt] (-1.0288704609498183,0.1725003606771991) -- (-1.0210164937172068,0.17500035609310288);
\draw[line width=1.pt] (-1.0210164937172068,0.17500035609310288) -- (-1.0131625264845954,0.17750035150900662);
\draw[line width=1.pt] (-1.0131625264845954,0.17750035150900662) -- (-1.005308559251984,0.1800003469249104);
\draw[line width=1.pt] (-1.005308559251984,0.1800003469249104) -- (-0.9974545920193724,0.1825003423408142);
\draw[line width=1.pt] (-0.9974545920193724,0.1825003423408142) -- (-0.9896006247867608,0.18500033775671804);
\draw[line width=1.pt] (-0.9896006247867608,0.18500033775671804) -- (-0.9817466575541492,0.18750033317262182);
\draw[line width=1.pt] (-0.9817466575541492,0.18750033317262182) -- (-0.9738926903215377,0.1900003285885256);
\draw[line width=1.pt] (-0.9738926903215377,0.1900003285885256) -- (-0.9660387230889261,0.19250032400442946);
\draw[line width=1.pt] (-0.9660387230889261,0.19250032400442946) -- (-0.9581847558563146,0.19500031942033325);
\draw[line width=1.pt] (-0.9581847558563146,0.19500031942033325) -- (-0.950330788623703,0.19750031483623703);
\draw[line width=1.pt] (-0.950330788623703,0.19750031483623703) -- (-0.9424768213910915,0.20000031025214088);
\draw[line width=1.pt] (-0.9424768213910915,0.20000031025214088) -- (-0.9346228541584799,0.20250030566804467);
\draw[line width=1.pt] (-0.9346228541584799,0.20250030566804467) -- (-0.9267688869258683,0.20500030108394846);
\draw[line width=1.pt] (-0.9267688869258683,0.20500030108394846) -- (-0.9189149196932568,0.2075002964998523);
\draw[line width=1.pt] (-0.9189149196932568,0.2075002964998523) -- (-0.9110609524606452,0.2100002919157561);
\draw[line width=1.pt] (-0.9110609524606452,0.2100002919157561) -- (-0.9032069852280337,0.21250028733165988);
\draw[line width=1.pt] (-0.9032069852280337,0.21250028733165988) -- (-0.8953530179954221,0.21500028274756372);
\draw[line width=1.pt] (-0.8953530179954221,0.21500028274756372) -- (-0.8874990507628105,0.2175002781634675);
\draw[line width=1.pt] (-0.8874990507628105,0.2175002781634675) -- (-0.879645083530199,0.2200002735793713);
\draw[line width=1.pt] (-0.879645083530199,0.2200002735793713) -- (-0.8717911162975874,0.22250026899527514);
\draw[line width=1.pt] (-0.8717911162975874,0.22250026899527514) -- (-0.8639371490649759,0.22500026441117893);
\draw[line width=1.pt] (-0.8639371490649759,0.22500026441117893) -- (-0.8560831818323643,0.22750025982708272);
\draw[line width=1.pt] (-0.8560831818323643,0.22750025982708272) -- (-0.8482292145997528,0.23000025524298656);
\draw[line width=1.pt] (-0.8482292145997528,0.23000025524298656) -- (-0.8403752473671412,0.23250025065889035);
\draw[line width=1.pt] (-0.8403752473671412,0.23250025065889035) -- (-0.8325212801345296,0.23500024607479414);
\draw[line width=1.pt] (-0.8325212801345296,0.23500024607479414) -- (-0.8246673129019181,0.23750024149069798);
\draw[line width=1.pt] (-0.8246673129019181,0.23750024149069798) -- (-0.8168133456693065,0.24000023690660177);
\draw[line width=1.pt] (-0.8168133456693065,0.24000023690660177) -- (-0.808959378436695,0.24250023232250556);
\draw[line width=1.pt] (-0.808959378436695,0.24250023232250556) -- (-0.8011054112040834,0.2450002277384094);
\draw[line width=1.pt] (-0.8011054112040834,0.2450002277384094) -- (-0.7932514439714718,0.2475002231543132);
\draw[line width=1.pt] (-0.7932514439714718,0.2475002231543132) -- (-0.7853974767388603,0.250000218570217);
\draw[line width=1.pt] (-0.7853974767388603,0.250000218570217) -- (-0.7775435095062487,0.25250021398612077);
\draw[line width=1.pt] (-0.7775435095062487,0.25250021398612077) -- (-0.7696895422736372,0.2550002094020246);
\draw[line width=1.pt] (-0.7696895422736372,0.2550002094020246) -- (-0.7618355750410256,0.2575002048179284);
\draw[line width=1.pt] (-0.7618355750410256,0.2575002048179284) -- (-0.7539816078084141,0.2600002002338322);
\draw[line width=1.pt] (-0.7539816078084141,0.2600002002338322) -- (-0.7461276405758025,0.26250019564973603);
\draw[line width=1.pt] (-0.7461276405758025,0.26250019564973603) -- (-0.7382736733431909,0.2650001910656398);
\draw[line width=1.pt] (-0.7382736733431909,0.2650001910656398) -- (-0.7304197061105794,0.2675001864815436);
\draw[line width=1.pt] (-0.7304197061105794,0.2675001864815436) -- (-0.7225657388779678,0.27000018189744746);
\draw[line width=1.pt] (-0.7225657388779678,0.27000018189744746) -- (-0.7147117716453563,0.27250017731335124);
\draw[line width=1.pt] (-0.7147117716453563,0.27250017731335124) -- (-0.7068578044127447,0.27500017272925503);
\draw[line width=1.pt] (-0.7068578044127447,0.27500017272925503) -- (-0.6990038371801331,0.2775001681451589);
\draw[line width=1.pt] (-0.6990038371801331,0.2775001681451589) -- (-0.6911498699475216,0.28000016356106266);
\draw[line width=1.pt] (-0.6911498699475216,0.28000016356106266) -- (-0.68329590271491,0.28250015897696645);
\draw[line width=1.pt] (-0.68329590271491,0.28250015897696645) -- (-0.6754419354822985,0.2850001543928703);
\draw[line width=1.pt] (-0.6754419354822985,0.2850001543928703) -- (-0.6675879682496869,0.2875001498087741);
\draw[line width=1.pt] (-0.6675879682496869,0.2875001498087741) -- (-0.6597340010170754,0.2900001452246779);
\draw[line width=1.pt] (-0.6597340010170754,0.2900001452246779) -- (-0.6518800337844638,0.2925001406405817);
\draw[line width=1.pt] (-0.6518800337844638,0.2925001406405817) -- (-0.6440260665518522,0.2950001360564855);
\draw[line width=1.pt] (-0.6440260665518522,0.2950001360564855) -- (-0.6361720993192407,0.2975001314723893);
\draw[line width=1.pt] (-0.6361720993192407,0.2975001314723893) -- (-0.6283181320866291,0.30000012688829314);
\draw[line width=1.pt] (-0.6283181320866291,0.30000012688829314) -- (-0.6204641648540176,0.30250012230419693);
\draw[line width=1.pt] (-0.6204641648540176,0.30250012230419693) -- (-0.612610197621406,0.3050001177201007);
\draw[line width=1.pt] (-0.612610197621406,0.3050001177201007) -- (-0.6047562303887944,0.30750011313600456);
\draw[line width=1.pt] (-0.6047562303887944,0.30750011313600456) -- (-0.5969022631561829,0.31000010855190835);
\draw[line width=1.pt] (-0.5969022631561829,0.31000010855190835) -- (-0.5890482959235713,0.31250010396781214);
\draw[line width=1.pt] (-0.5890482959235713,0.31250010396781214) -- (-0.5811943286909598,0.315000099383716);
\draw[line width=1.pt] (-0.5811943286909598,0.315000099383716) -- (-0.5733403614583482,0.31750009479961977);
\draw[line width=1.pt] (-0.5733403614583482,0.31750009479961977) -- (-0.5654863942257367,0.32000009021552356);
\draw[line width=1.pt] (-0.5654863942257367,0.32000009021552356) -- (-0.5576324269931251,0.3225000856314274);
\draw[line width=1.pt] (-0.5576324269931251,0.3225000856314274) -- (-0.5497784597605135,0.3250000810473312);
\draw[line width=1.pt] (-0.5497784597605135,0.3250000810473312) -- (-0.541924492527902,0.327500076463235);
\draw[line width=1.pt] (-0.541924492527902,0.327500076463235) -- (-0.5340705252952904,0.3300000718791388);
\draw[line width=1.pt] (-0.5340705252952904,0.3300000718791388) -- (-0.5262165580626789,0.3325000672950426);
\draw[line width=1.pt] (-0.5262165580626789,0.3325000672950426) -- (-0.5183625908300673,0.3350000627109464);
\draw[line width=1.pt] (-0.5183625908300673,0.3350000627109464) -- (-0.5105086235974557,0.33750005812685024);
\draw[line width=1.pt] (-0.5105086235974557,0.33750005812685024) -- (-0.5026546563648442,0.34000005354275403);
\draw[line width=1.pt] (-0.5026546563648442,0.34000005354275403) -- (-0.4948006891322327,0.3425000489586578);
\draw[line width=1.pt] (-0.4948006891322327,0.3425000489586578) -- (-0.4869467218996212,0.3450000443745616);
\draw[line width=1.pt] (-0.4869467218996212,0.3450000443745616) -- (-0.4790927546670097,0.3475000397904654);
\draw[line width=1.pt] (-0.4790927546670097,0.3475000397904654) -- (-0.4712387874343982,0.3500000352063692);
\draw[line width=1.pt] (-0.4712387874343982,0.3500000352063692) -- (-0.4633848202017867,0.352500030622273);
\draw[line width=1.pt] (-0.4633848202017867,0.352500030622273) -- (-0.45553085296917517,0.35500002603817676);
\draw[line width=1.pt] (-0.45553085296917517,0.35500002603817676) -- (-0.44767688573656367,0.35750002145408055);
\draw[line width=1.pt] (-0.44767688573656367,0.35750002145408055) -- (-0.43982291850395216,0.36000001686998434);
\draw[line width=1.pt] (-0.43982291850395216,0.36000001686998434) -- (-0.43196895127134066,0.36250001228588813);
\draw[line width=1.pt] (-0.43196895127134066,0.36250001228588813) -- (-0.42411498403872916,0.3650000077017919);
\draw[line width=1.pt] (-0.42411498403872916,0.3650000077017919) -- (-0.41626101680611766,0.3675000031176957);
\draw[line width=1.pt] (-0.41626101680611766,0.3675000031176957) -- (-0.40840704957350615,0.3699999985335995);
\draw[line width=1.pt] (-0.40840704957350615,0.3699999985335995) -- (-0.40055308234089465,0.3724999939495033);
\draw[line width=1.pt] (-0.40055308234089465,0.3724999939495033) -- (-0.39269911510828315,0.37499998936540707);
\draw[line width=1.pt] (-0.39269911510828315,0.37499998936540707) -- (-0.38484514787567164,0.37749998478131086);
\draw[line width=1.pt] (-0.38484514787567164,0.37749998478131086) -- (-0.37699118064306014,0.37999998019721465);
\draw[line width=1.pt] (-0.37699118064306014,0.37999998019721465) -- (-0.36913721341044864,0.38249997561311844);
\draw[line width=1.pt] (-0.36913721341044864,0.38249997561311844) -- (-0.36128324617783714,0.3849999710290222);
\draw[line width=1.pt] (-0.36128324617783714,0.3849999710290222) -- (-0.35342927894522563,0.387499966444926);
\draw[line width=1.pt] (-0.35342927894522563,0.387499966444926) -- (-0.34557531171261413,0.3899999618608298);
\draw[line width=1.pt] (-0.34557531171261413,0.3899999618608298) -- (-0.3377213444800026,0.3924999572767336);
\draw[line width=1.pt] (-0.3377213444800026,0.3924999572767336) -- (-0.3298673772473911,0.3949999526926374);
\draw[line width=1.pt] (-0.3298673772473911,0.3949999526926374) -- (-0.3220134100147796,0.39749994810854117);
\draw[line width=1.pt] (-0.3220134100147796,0.39749994810854117) -- (-0.3141594427821681,0.39999994352444496);
\draw[line width=1.pt] (-0.3141594427821681,0.39999994352444496) -- (-0.3063054755495566,0.40249993894034874);
\draw[line width=1.pt] (-0.3063054755495566,0.40249993894034874) -- (-0.2984515083169451,0.40499993435625253);
\draw[line width=1.pt] (-0.2984515083169451,0.40499993435625253) -- (-0.2905975410843336,0.4074999297721563);
\draw[line width=1.pt] (-0.2905975410843336,0.4074999297721563) -- (-0.2827435738517221,0.4099999251880601);
\draw[line width=1.pt] (-0.2827435738517221,0.4099999251880601) -- (-0.2748896066191106,0.4124999206039639);
\draw[line width=1.pt] (-0.2748896066191106,0.4124999206039639) -- (-0.2670356393864991,0.4149999160198677);
\draw[line width=1.pt] (-0.2670356393864991,0.4149999160198677) -- (-0.2591816721538876,0.4174999114357715);
\draw[line width=1.pt] (-0.2591816721538876,0.4174999114357715) -- (-0.2513277049212761,0.41999990685167526);
\draw[line width=1.pt] (-0.2513277049212761,0.41999990685167526) -- (-0.2434737376886646,0.42249990226757905);
\draw[line width=1.pt] (-0.2434737376886646,0.42249990226757905) -- (-0.2356197704560531,0.42499989768348284);
\draw[line width=1.pt] (-0.2356197704560531,0.42499989768348284) -- (-0.2277658032234416,0.42749989309938663);
\draw[line width=1.pt] (-0.2277658032234416,0.42749989309938663) -- (-0.21991183599083008,0.4299998885152904);
\draw[line width=1.pt] (-0.21991183599083008,0.4299998885152904) -- (-0.21205786875821858,0.4324998839311942);
\draw[line width=1.pt] (-0.21205786875821858,0.4324998839311942) -- (-0.20420390152560708,0.434999879347098);
\draw[line width=1.pt] (-0.20420390152560708,0.434999879347098) -- (-0.19634993429299558,0.4374998747630018);
\draw[line width=1.pt] (-0.19634993429299558,0.4374998747630018) -- (-0.18849596706038407,0.43999987017890557);
\draw[line width=1.pt] (-0.18849596706038407,0.43999987017890557) -- (-0.18064199982777257,0.44249986559480936);
\draw[line width=1.pt] (-0.18064199982777257,0.44249986559480936) -- (-0.17278803259516107,0.44499986101071315);
\draw[line width=1.pt] (-0.17278803259516107,0.44499986101071315) -- (-0.16493406536254956,0.44749985642661694);
\draw[line width=1.pt] (-0.16493406536254956,0.44749985642661694) -- (-0.15708009812993806,0.4499998518425207);
\draw[line width=1.pt] (-0.15708009812993806,0.4499998518425207) -- (-0.14922613089732656,0.4524998472584245);
\draw[line width=1.pt] (-0.14922613089732656,0.4524998472584245) -- (-0.14137216366471506,0.4549998426743283);
\draw[line width=1.pt] (-0.14137216366471506,0.4549998426743283) -- (-0.13351819643210355,0.4574998380902321);
\draw[line width=1.pt] (-0.13351819643210355,0.4574998380902321) -- (-0.12566422919949205,0.4599998335061359);
\draw[line width=1.pt] (-0.12566422919949205,0.4599998335061359) -- (-0.11781026196688053,0.46249982892203967);
\draw[line width=1.pt] (-0.11781026196688053,0.46249982892203967) -- (-0.10995629473426902,0.4649998243379435);
\draw[line width=1.pt] (-0.10995629473426902,0.4649998243379435) -- (-0.1021023275016575,0.4674998197538473);
\draw[line width=1.pt] (-0.1021023275016575,0.4674998197538473) -- (-0.09424836026904598,0.4699998151697511);
\draw[line width=1.pt] (-0.09424836026904598,0.4699998151697511) -- (-0.08639439303643447,0.4724998105856549);
\draw[line width=1.pt] (-0.08639439303643447,0.4724998105856549) -- (-0.07854042580382295,0.47499980600155867);
\draw[line width=1.pt] (-0.07854042580382295,0.47499980600155867) -- (-0.07068645857121143,0.47749980141746245);
\draw[line width=1.pt] (-0.07068645857121143,0.47749980141746245) -- (-0.06283249133859992,0.47999979683336624);
\draw[line width=1.pt] (-0.06283249133859992,0.47999979683336624) -- (-0.0549785241059884,0.48249979224927003);
\draw[line width=1.pt] (-0.0549785241059884,0.48249979224927003) -- (-0.04712455687337688,0.4849997876651738);
\draw[line width=1.pt] (-0.04712455687337688,0.4849997876651738) -- (-0.039270589640765366,0.4874997830810776);
\draw[line width=1.pt] (-0.039270589640765366,0.4874997830810776) -- (-0.03141662240815385,0.4899997784969814);
\draw[line width=1.pt] (-0.03141662240815385,0.4899997784969814) -- (-0.023562655175542332,0.49249977391288524);
\draw[line width=1.pt] (-0.023562655175542332,0.49249977391288524) -- (-0.015708687942930816,0.49499976932878903);
\draw[line width=1.pt] (-0.015708687942930816,0.49499976932878903) -- (-0.0078547207103193,0.4974997647446928);
\draw[line width=1.pt] (-0.0078547207103193,0.4974997647446928) -- (0.0,0.4999997601605966);
\draw[line width=1.pt] (0.0,0.4999997601605966) -- (0.00785321375490373,0.5024997555765004);
\draw[line width=1.pt] (0.00785321375490373,0.5024997555765004) -- (0.015707180987515244,0.5049997509924041);
\draw[line width=1.pt] (0.015707180987515244,0.5049997509924041) -- (0.02356114822012676,0.507499746408308);
\draw[line width=1.pt] (0.02356114822012676,0.507499746408308) -- (0.03141511545273828,0.5099997418242118);
\draw[line width=1.pt] (0.03141511545273828,0.5099997418242118) -- (0.039269082685349795,0.5124997372401155);
\draw[line width=1.pt] (0.039269082685349795,0.5124997372401155) -- (0.04712304991796131,0.5149997326560194);
\draw[line width=1.pt] (0.04712304991796131,0.5149997326560194) -- (0.05497701715057283,0.5174997280719231);
\draw[line width=1.pt] (0.05497701715057283,0.5174997280719231) -- (0.06283098438318434,0.519999723487827);
\draw[line width=1.pt] (0.06283098438318434,0.519999723487827) -- (0.07068495161579585,0.5224997189037307);
\draw[line width=1.pt] (0.07068495161579585,0.5224997189037307) -- (0.07853891884840737,0.5249997143196345);
\draw[line width=1.pt] (0.07853891884840737,0.5249997143196345) -- (0.08639288608101889,0.5274997097355383);
\draw[line width=1.pt] (0.08639288608101889,0.5274997097355383) -- (0.0942468533136304,0.5299997051514421);
\draw[line width=1.pt] (0.0942468533136304,0.5299997051514421) -- (0.10210082054624192,0.5324997005673459);
\draw[line width=1.pt] (0.10210082054624192,0.5324997005673459) -- (0.10995478777885344,0.5349996959832497);
\draw[line width=1.pt] (0.10995478777885344,0.5349996959832497) -- (0.11780875501146496,0.5374996913991534);
\draw[line width=1.pt] (0.11780875501146496,0.5374996913991534) -- (0.12566272224407646,0.5399996868150573);
\draw[line width=1.pt] (0.12566272224407646,0.5399996868150573) -- (0.13351668947668796,0.5424996822309611);
\draw[line width=1.pt] (0.13351668947668796,0.5424996822309611) -- (0.14137065670929946,0.5449996776468649);
\draw[line width=1.pt] (0.14137065670929946,0.5449996776468649) -- (0.14922462394191097,0.5474996730627686);
\draw[line width=1.pt] (0.14922462394191097,0.5474996730627686) -- (0.15707859117452247,0.5499996684786724);
\draw[line width=1.pt] (0.15707859117452247,0.5499996684786724) -- (0.16493255840713397,0.5524996638945763);
\draw[line width=1.pt] (0.16493255840713397,0.5524996638945763) -- (0.17278652563974548,0.55499965931048);
\draw[line width=1.pt] (0.17278652563974548,0.55499965931048) -- (0.18064049287235698,0.5574996547263839);
\draw[line width=1.pt] (0.18064049287235698,0.5574996547263839) -- (0.18849446010496848,0.5599996501422876);
\draw[line width=1.pt] (0.18849446010496848,0.5599996501422876) -- (0.19634842733757998,0.5624996455581914);
\draw[line width=1.pt] (0.19634842733757998,0.5624996455581914) -- (0.2042023945701915,0.5649996409740952);
\draw[line width=1.pt] (0.2042023945701915,0.5649996409740952) -- (0.212056361802803,0.5674996363899989);
\draw[line width=1.pt] (0.212056361802803,0.5674996363899989) -- (0.2199103290354145,0.5699996318059027);
\draw[line width=1.pt] (0.2199103290354145,0.5699996318059027) -- (0.227764296268026,0.5724996272218066);
\draw[line width=1.pt] (0.227764296268026,0.5724996272218066) -- (0.2356182635006375,0.5749996226377103);
\draw[line width=1.pt] (0.2356182635006375,0.5749996226377103) -- (0.243472230733249,0.5774996180536142);
\draw[line width=1.pt] (0.243472230733249,0.5774996180536142) -- (0.25132619796586053,0.5799996134695179);
\draw[line width=1.pt] (0.25132619796586053,0.5799996134695179) -- (0.25918016519847203,0.5824996088854217);
\draw[line width=1.pt] (0.25918016519847203,0.5824996088854217) -- (0.26703413243108354,0.5849996043013255);
\draw[line width=1.pt] (0.26703413243108354,0.5849996043013255) -- (0.27488809966369504,0.5874995997172293);
\draw[line width=1.pt] (0.27488809966369504,0.5874995997172293) -- (0.28274206689630654,0.589999595133133);
\draw[line width=1.pt] (0.28274206689630654,0.589999595133133) -- (0.29059603412891805,0.5924995905490369);
\draw[line width=1.pt] (0.29059603412891805,0.5924995905490369) -- (0.29845000136152955,0.5949995859649406);
\draw[line width=1.pt] (0.29845000136152955,0.5949995859649406) -- (0.30630396859414105,0.5974995813808445);
\draw[line width=1.pt] (0.30630396859414105,0.5974995813808445) -- (0.31415793582675255,0.5999995767967482);
\draw[line width=1.pt] (0.31415793582675255,0.5999995767967482) -- (0.32201190305936406,0.602499572212652);
\draw[line width=1.pt] (0.32201190305936406,0.602499572212652) -- (0.32986587029197556,0.6049995676285558);
\draw[line width=1.pt] (0.32986587029197556,0.6049995676285558) -- (0.33771983752458706,0.6074995630444596);
\draw[line width=1.pt] (0.33771983752458706,0.6074995630444596) -- (0.34557380475719857,0.6099995584603634);
\draw[line width=1.pt] (0.34557380475719857,0.6099995584603634) -- (0.35342777198981007,0.6124995538762672);
\draw[line width=1.pt] (0.35342777198981007,0.6124995538762672) -- (0.36128173922242157,0.6149995492921709);
\draw[line width=1.pt] (0.36128173922242157,0.6149995492921709) -- (0.3691357064550331,0.6174995447080748);
\draw[line width=1.pt] (0.3691357064550331,0.6174995447080748) -- (0.3769896736876446,0.6199995401239785);
\draw[line width=1.pt] (0.3769896736876446,0.6199995401239785) -- (0.3848436409202561,0.6224995355398824);
\draw[line width=1.pt] (0.3848436409202561,0.6224995355398824) -- (0.3926976081528676,0.6249995309557861);
\draw[line width=1.pt] (0.3926976081528676,0.6249995309557861) -- (0.4005515753854791,0.6274995263716899);
\draw[line width=1.pt] (0.4005515753854791,0.6274995263716899) -- (0.4084055426180906,0.6299995217875937);
\draw[line width=1.pt] (0.4084055426180906,0.6299995217875937) -- (0.4162595098507021,0.6324995172034975);
\draw[line width=1.pt] (0.4162595098507021,0.6324995172034975) -- (0.4241134770833136,0.6349995126194012);
\draw[line width=1.pt] (0.4241134770833136,0.6349995126194012) -- (0.4319674443159251,0.6374995080353051);
\draw[line width=1.pt] (0.4319674443159251,0.6374995080353051) -- (0.4398214115485366,0.6399995034512088);
\draw[line width=1.pt] (0.4398214115485366,0.6399995034512088) -- (0.4476753787811481,0.6424994988671127);
\draw[line width=1.pt] (0.4476753787811481,0.6424994988671127) -- (0.4555293460137596,0.6449994942830164);
\draw[line width=1.pt] (0.4555293460137596,0.6449994942830164) -- (0.4633833132463711,0.6474994896989202);
\draw[line width=1.pt] (0.4633833132463711,0.6474994896989202) -- (0.4712372804789826,0.649999485114824);
\draw[line width=1.pt] (0.4712372804789826,0.649999485114824) -- (0.4790912477115941,0.6524994805307278);
\draw[line width=1.pt] (0.4790912477115941,0.6524994805307278) -- (0.4869452149442056,0.6549994759466315);
\draw[line width=1.pt] (0.4869452149442056,0.6549994759466315) -- (0.4947991821768171,0.6574994713625354);
\draw[line width=1.pt] (0.4947991821768171,0.6574994713625354) -- (0.5026531494094286,0.6599994667784391);
\draw[line width=1.pt] (0.5026531494094286,0.6599994667784391) -- (0.5105071166420402,0.662499462194343);
\draw[line width=1.pt] (0.5105071166420402,0.662499462194343) -- (0.5183610838746517,0.6649994576102467);
\draw[line width=1.pt] (0.5183610838746517,0.6649994576102467) -- (0.5262150511072633,0.6674994530261505);
\draw[line width=1.pt] (0.5262150511072633,0.6674994530261505) -- (0.5340690183398749,0.6699994484420544);
\draw[line width=1.pt] (0.5340690183398749,0.6699994484420544) -- (0.5419229855724864,0.6724994438579581);
\draw[line width=1.pt] (0.5419229855724864,0.6724994438579581) -- (0.549776952805098,0.674999439273862);
\draw[line width=1.pt] (0.549776952805098,0.674999439273862) -- (0.5576309200377095,0.6774994346897658);
\draw[line width=1.pt] (0.5576309200377095,0.6774994346897658) -- (0.5654848872703211,0.6799994301056695);
\draw[line width=1.pt] (0.5654848872703211,0.6799994301056695) -- (0.5733388545029326,0.6824994255215734);
\draw[line width=1.pt] (0.5733388545029326,0.6824994255215734) -- (0.5811928217355442,0.6849994209374772);
\draw[line width=1.pt] (0.5811928217355442,0.6849994209374772) -- (0.5890467889681558,0.687499416353381);
\draw[line width=1.pt] (0.5890467889681558,0.687499416353381) -- (0.5969007562007673,0.6899994117692848);
\draw[line width=1.pt] (0.5969007562007673,0.6899994117692848) -- (0.6047547234333789,0.6924994071851887);
\draw[line width=1.pt] (0.6047547234333789,0.6924994071851887) -- (0.6126086906659904,0.6949994026010924);
\draw[line width=1.pt] (0.6126086906659904,0.6949994026010924) -- (0.620462657898602,0.6974993980169962);
\draw[line width=1.pt] (0.620462657898602,0.6974993980169962) -- (0.6283166251312136,0.6999993934329001);
\draw[line width=1.pt] (0.6283166251312136,0.6999993934329001) -- (0.6361705923638251,0.7024993888488038);
\draw[line width=1.pt] (0.6361705923638251,0.7024993888488038) -- (0.6440245595964367,0.7049993842647077);
\draw[line width=1.pt] (0.6440245595964367,0.7049993842647077) -- (0.6518785268290482,0.7074993796806115);
\draw[line width=1.pt] (0.6518785268290482,0.7074993796806115) -- (0.6597324940616598,0.7099993750965152);
\draw[line width=1.pt] (0.6597324940616598,0.7099993750965152) -- (0.6675864612942713,0.7124993705124191);
\draw[line width=1.pt] (0.6675864612942713,0.7124993705124191) -- (0.6754404285268829,0.7149993659283229);
\draw[line width=1.pt] (0.6754404285268829,0.7149993659283229) -- (0.6832943957594945,0.7174993613442266);
\draw[line width=1.pt] (0.6832943957594945,0.7174993613442266) -- (0.691148362992106,0.7199993567601305);
\draw[line width=1.pt] (0.691148362992106,0.7199993567601305) -- (0.6990023302247176,0.7224993521760343);
\draw[line width=1.pt] (0.6990023302247176,0.7224993521760343) -- (0.7068562974573291,0.7249993475919381);
\draw[line width=1.pt] (0.7068562974573291,0.7249993475919381) -- (0.7147102646899407,0.7274993430078419);
\draw[line width=1.pt] (0.7147102646899407,0.7274993430078419) -- (0.7225642319225523,0.7299993384237458);
\draw[line width=1.pt] (0.7225642319225523,0.7299993384237458) -- (0.7304181991551638,0.7324993338396495);
\draw[line width=1.pt] (0.7304181991551638,0.7324993338396495) -- (0.7382721663877754,0.7349993292555533);
\draw[line width=1.pt] (0.7382721663877754,0.7349993292555533) -- (0.7461261336203869,0.7374993246714572);
\draw[line width=1.pt] (0.7461261336203869,0.7374993246714572) -- (0.7539801008529985,0.7399993200873609);
\draw[line width=1.pt] (0.7539801008529985,0.7399993200873609) -- (0.76183406808561,0.7424993155032648);
\draw[line width=1.pt] (0.76183406808561,0.7424993155032648) -- (0.7696880353182216,0.7449993109191686);
\draw[line width=1.pt] (0.7696880353182216,0.7449993109191686) -- (0.7775420025508332,0.7474993063350723);
\draw[line width=1.pt] (0.7775420025508332,0.7474993063350723) -- (0.7853959697834447,0.7499993017509762);
\draw[line width=1.pt] (0.7853959697834447,0.7499993017509762) -- (0.7932499370160563,0.75249929716688);
\draw[line width=1.pt] (0.7932499370160563,0.75249929716688) -- (0.8011039042486678,0.7549992925827838);
\draw[line width=1.pt] (0.8011039042486678,0.7549992925827838) -- (0.8089578714812794,0.7574992879986876);
\draw[line width=1.pt] (0.8089578714812794,0.7574992879986876) -- (0.816811838713891,0.7599992834145914);
\draw[line width=1.pt] (0.816811838713891,0.7599992834145914) -- (0.8246658059465025,0.7624992788304952);
\draw[line width=1.pt] (0.8246658059465025,0.7624992788304952) -- (0.8325197731791141,0.764999274246399);
\draw[line width=1.pt] (0.8325197731791141,0.764999274246399) -- (0.8403737404117256,0.7674992696623029);
\draw[line width=1.pt] (0.8403737404117256,0.7674992696623029) -- (0.8482277076443372,0.7699992650782066);
\draw[line width=1.pt] (0.8482277076443372,0.7699992650782066) -- (0.8560816748769488,0.7724992604941104);
\draw[line width=1.pt] (0.8560816748769488,0.7724992604941104) -- (0.8639356421095603,0.7749992559100143);
\draw[line width=1.pt] (0.8639356421095603,0.7749992559100143) -- (0.8717896093421719,0.777499251325918);
\draw[line width=1.pt] (0.8717896093421719,0.777499251325918) -- (0.8796435765747834,0.7799992467418219);
\draw[line width=1.pt] (0.8796435765747834,0.7799992467418219) -- (0.887497543807395,0.7824992421577257);
\draw[line width=1.pt] (0.887497543807395,0.7824992421577257) -- (0.8953515110400065,0.7849992375736294);
\draw[line width=1.pt] (0.8953515110400065,0.7849992375736294) -- (0.9032054782726181,0.7874992329895333);
\draw[line width=1.pt] (0.9032054782726181,0.7874992329895333) -- (0.9110594455052297,0.7899992284054371);
\draw[line width=1.pt] (0.9110594455052297,0.7899992284054371) -- (0.9189134127378412,0.7924992238213409);
\draw[line width=1.pt] (0.9189134127378412,0.7924992238213409) -- (0.9267673799704528,0.7949992192372447);
\draw[line width=1.pt] (0.9267673799704528,0.7949992192372447) -- (0.9346213472030643,0.7974992146531485);
\draw[line width=1.pt] (0.9346213472030643,0.7974992146531485) -- (0.9424753144356759,0.7999992100690523);
\draw[line width=1.pt] (0.9424753144356759,0.7999992100690523) -- (0.9503292816682875,0.8024992054849561);
\draw[line width=1.pt] (0.9503292816682875,0.8024992054849561) -- (0.958183248900899,0.80499920090086);
\draw[line width=1.pt] (0.958183248900899,0.80499920090086) -- (0.9660372161335106,0.8074991963167637);
\draw[line width=1.pt] (0.9660372161335106,0.8074991963167637) -- (0.9738911833661221,0.8099991917326675);
\draw[line width=1.pt] (0.9738911833661221,0.8099991917326675) -- (0.9817451505987337,0.8124991871485714);
\draw[line width=1.pt] (0.9817451505987337,0.8124991871485714) -- (0.9895991178313452,0.8149991825644751);
\draw[line width=1.pt] (0.9895991178313452,0.8149991825644751) -- (0.9974530850639568,0.817499177980379);
\draw[line width=1.pt] (0.9974530850639568,0.817499177980379) -- (1.0053070522965684,0.8199991733962828);
\draw[line width=1.pt] (1.0053070522965684,0.8199991733962828) -- (1.0131610195291798,0.8224991688121865);
\draw[line width=1.pt] (1.0131610195291798,0.8224991688121865) -- (1.0210149867617913,0.8249991642280903);
\draw[line width=1.pt] (1.0210149867617913,0.8249991642280903) -- (1.0288689539944027,0.8274991596439941);
\draw[line width=1.pt] (1.0288689539944027,0.8274991596439941) -- (1.0367229212270141,0.8299991550598979);
\draw[line width=1.pt] (1.0367229212270141,0.8299991550598979) -- (1.0445768884596256,0.8324991504758017);
\draw[line width=1.pt] (1.0445768884596256,0.8324991504758017) -- (1.052430855692237,0.8349991458917054);
\draw[line width=1.pt] (1.052430855692237,0.8349991458917054) -- (1.0602848229248485,0.8374991413076092);
\draw[line width=1.pt] (1.0602848229248485,0.8374991413076092) -- (1.06813879015746,0.8399991367235129);
\draw[line width=1.pt] (1.06813879015746,0.8399991367235129) -- (1.0759927573900714,0.8424991321394167);
\draw[line width=1.pt] (1.0759927573900714,0.8424991321394167) -- (1.0838467246226828,0.8449991275553205);
\draw[line width=1.pt] (1.0838467246226828,0.8449991275553205) -- (1.0917006918552943,0.8474991229712243);
\draw[line width=1.pt] (1.0917006918552943,0.8474991229712243) -- (1.0995546590879057,0.849999118387128);
\draw[line width=1.pt] (1.0995546590879057,0.849999118387128) -- (1.1074086263205172,0.8524991138030318);
\draw[line width=1.pt] (1.1074086263205172,0.8524991138030318) -- (1.1152625935531286,0.8549991092189355);
\draw[line width=1.pt] (1.1152625935531286,0.8549991092189355) -- (1.12311656078574,0.8574991046348394);
\draw[line width=1.pt] (1.12311656078574,0.8574991046348394) -- (1.1309705280183515,0.8599991000507432);
\draw[line width=1.pt] (1.1309705280183515,0.8599991000507432) -- (1.138824495250963,0.8624990954666469);
\draw[line width=1.pt] (1.138824495250963,0.8624990954666469) -- (1.1466784624835744,0.8649990908825507);
\draw[line width=1.pt] (1.1466784624835744,0.8649990908825507) -- (1.1545324297161859,0.8674990862984544);
\draw[line width=1.pt] (1.1545324297161859,0.8674990862984544) -- (1.1623863969487973,0.8699990817143581);
\draw[line width=1.pt] (1.1623863969487973,0.8699990817143581) -- (1.1702403641814088,0.872499077130262);
\draw[line width=1.pt] (1.1702403641814088,0.872499077130262) -- (1.1780943314140202,0.8749990725461658);
\draw[line width=1.pt] (1.1780943314140202,0.8749990725461658) -- (1.1859482986466316,0.8774990679620696);
\draw[line width=1.pt] (1.1859482986466316,0.8774990679620696) -- (1.193802265879243,0.8799990633779733);
\draw[line width=1.pt] (1.193802265879243,0.8799990633779733) -- (1.2016562331118545,0.882499058793877);
\draw[line width=1.pt] (1.2016562331118545,0.882499058793877) -- (1.209510200344466,0.8849990542097808);
\draw[line width=1.pt] (1.209510200344466,0.8849990542097808) -- (1.2173641675770774,0.8874990496256846);
\draw[line width=1.pt] (1.2173641675770774,0.8874990496256846) -- (1.2252181348096889,0.8899990450415884);
\draw[line width=1.pt] (1.2252181348096889,0.8899990450415884) -- (1.2330721020423003,0.8924990404574922);
\draw[line width=1.pt] (1.2330721020423003,0.8924990404574922) -- (1.2409260692749118,0.8949990358733959);
\draw[line width=1.pt] (1.2409260692749118,0.8949990358733959) -- (1.2487800365075232,0.8974990312892996);
\draw[line width=1.pt] (1.2487800365075232,0.8974990312892996) -- (1.2566340037401347,0.8999990267052035);
\draw[line width=1.pt] (1.2566340037401347,0.8999990267052035) -- (1.2644879709727461,0.9024990221211072);
\draw[line width=1.pt] (1.2644879709727461,0.9024990221211072) -- (1.2723419382053576,0.9049990175370111);
\draw[line width=1.pt] (1.2723419382053576,0.9049990175370111) -- (1.280195905437969,0.9074990129529148);
\draw[line width=1.pt] (1.280195905437969,0.9074990129529148) -- (1.2880498726705805,0.9099990083688185);
\draw[line width=1.pt] (1.2880498726705805,0.9099990083688185) -- (1.295903839903192,0.9124990037847223);
\draw[line width=1.pt] (1.295903839903192,0.9124990037847223) -- (1.3037578071358034,0.9149989992006261);
\draw[line width=1.pt] (1.3037578071358034,0.9149989992006261) -- (1.3116117743684148,0.9174989946165298);
\draw[line width=1.pt] (1.3116117743684148,0.9174989946165298) -- (1.3194657416010263,0.9199989900324337);
\draw[line width=1.pt] (1.3194657416010263,0.9199989900324337) -- (1.3273197088336377,0.9224989854483374);
\draw[line width=1.pt] (1.3273197088336377,0.9224989854483374) -- (1.3351736760662491,0.9249989808642411);
\draw[line width=1.pt] (1.3351736760662491,0.9249989808642411) -- (1.3430276432988606,0.9274989762801449);
\draw[line width=1.pt] (1.3430276432988606,0.9274989762801449) -- (1.350881610531472,0.9299989716960487);
\draw[line width=1.pt] (1.350881610531472,0.9299989716960487) -- (1.3587355777640835,0.9324989671119526);
\draw[line width=1.pt] (1.3587355777640835,0.9324989671119526) -- (1.366589544996695,0.9349989625278563);
\draw[line width=1.pt] (1.366589544996695,0.9349989625278563) -- (1.3744435122293064,0.93749895794376);
\draw[line width=1.pt] (1.3744435122293064,0.93749895794376) -- (1.3822974794619178,0.9399989533596638);
\draw[line width=1.pt] (1.3822974794619178,0.9399989533596638) -- (1.3901514466945293,0.9424989487755675);
\draw[line width=1.pt] (1.3901514466945293,0.9424989487755675) -- (1.3980054139271407,0.9449989441914713);
\draw[line width=1.pt] (1.3980054139271407,0.9449989441914713) -- (1.4058593811597522,0.9474989396073752);
\draw[line width=1.pt] (1.4058593811597522,0.9474989396073752) -- (1.4137133483923636,0.9499989350232789);
\draw[line width=1.pt] (1.4137133483923636,0.9499989350232789) -- (1.421567315624975,0.9524989304391827);
\draw[line width=1.pt] (1.421567315624975,0.9524989304391827) -- (1.4294212828575865,0.9549989258550864);
\draw[line width=1.pt] (1.4294212828575865,0.9549989258550864) -- (1.437275250090198,0.9574989212709901);
\draw[line width=1.pt] (1.437275250090198,0.9574989212709901) -- (1.4451292173228094,0.959998916686894);
\draw[line width=1.pt] (1.4451292173228094,0.959998916686894) -- (1.4529831845554209,0.9624989121027978);
\draw[line width=1.pt] (1.4529831845554209,0.9624989121027978) -- (1.4608371517880323,0.9649989075187015);
\draw[line width=1.pt] (1.4608371517880323,0.9649989075187015) -- (1.4686911190206438,0.9674989029346053);
\draw[line width=1.pt] (1.4686911190206438,0.9674989029346053) -- (1.4765450862532552,0.969998898350509);
\draw[line width=1.pt] (1.4765450862532552,0.969998898350509) -- (1.4843990534858666,0.9724988937664129);
\draw[line width=1.pt] (1.4843990534858666,0.9724988937664129) -- (1.492253020718478,0.9749988891823166);
\draw[line width=1.pt] (1.492253020718478,0.9749988891823166) -- (1.5001069879510895,0.9774988845982204);
\draw[line width=1.pt] (1.5001069879510895,0.9774988845982204) -- (1.507960955183701,0.9799988800141242);
\draw[line width=1.pt] (1.507960955183701,0.9799988800141242) -- (1.5158149224163124,0.9824988754300279);
\draw[line width=1.pt] (1.5158149224163124,0.9824988754300279) -- (1.5236688896489239,0.9849988708459316);
\draw[line width=1.pt] (1.5236688896489239,0.9849988708459316) -- (1.5315228568815353,0.9874988662618355);
\draw[line width=1.pt] (1.5315228568815353,0.9874988662618355) -- (1.5393768241141468,0.9899988616777392);
\draw[line width=1.pt] (1.5393768241141468,0.9899988616777392) -- (1.5472307913467582,0.992498857093643);
\draw[line width=1.pt] (1.5472307913467582,0.992498857093643) -- (1.5550847585793697,0.9949988525095468);
\draw[line width=1.pt] (1.5550847585793697,0.9949988525095468) -- (1.5629387258119811,0.9974988479254505);
\draw[line width=1.pt] (1.5629387258119811,0.9974988479254505) -- (1.5707926930445926,0.9999988433413542);
\draw (-1.8229746643444675,-0.004316350160548921) node[anchor=north west] {$-\frac{\pi}{2}$};
\draw (1.4485643849316332,-0.004316350160548921) node[anchor=north west] {$\frac{\pi}{2}$};
\draw [line width=1.pt,dash pattern=on 1pt off 3pt,color=ffqqqq] (-1.5707963267948966,1.)-- (1.5707963267948966,1.);
\draw [line width=1.pt,dash pattern=on 1pt off 3pt,color=ffqqqq] (-1.5707963267948966,0.)-- (-1.5707963267948966,1.);
\draw [line width=1.pt,dash pattern=on 1pt off 3pt,color=ffqqqq] (1.5707963267948966,0.)-- (1.5707963267948966,1.);
\end{axis}
\end{tikzpicture}
            \caption{Graf funkce $g_4$ (složení funkcí $g_3$ a $g_2$).}
            \label{fig:graf_g4}
        \end{figure}
        Lze vidět, že $g_4$ je skutečně bijekce, což ostatně plyne z bodu \ref{item:skladani_bijekce} tvrzení \ref{prop:vlastnosti_skladani_zobrazeni} o vlastnostech skládání zobrazení.
    \end{enumerate}
\end{example}