\section{Zobrazení}
Jedním z nejdůležitějších typů relací je tzv. \emph{zobrazení}. Zde se zároveň dostáváme trochu zpět k tématu, kterým se čtenář na střední škole jistě zabýval, akorát se o něm nemluvilo v souvislosti s relacemi, a to sice k \emph{funkcím}. Termíny jako definiční obor, obor hodnot, aj. nejspíše tak pro nás nebudou velkou neznámou, ale přesto nezanedbáme jejich formální zavedení.
\medskip

V následující definici si představíme variantu existenčního kvantifikátoru se symbolem "$!$", tj. $\exists!$. Jeho význam je "\textbf{existuje právě jeden/jedno}".
\begin{definition}[Zobrazení]\label{def:zobrazeni}
    \emph{Zobrazením z množiny $X$ do množiny $Y$} nazýváme relaci $f\subseteq X\times Y$, když platí
    \begin{equation*}
        \forall x\in X,\; \exists! y\in Y : xfy\; .
    \end{equation*}
\end{definition}
Že $f$ je zobrazení z $X$ do $Y$ zapisujeme jako
\begin{equation*}
    \map{f}{X}{Y}\; ,
\end{equation*}
a to, že zobrazení $f$ přiřazuje prvku $x$ prvek $y$ vyjádříme zápisem
\begin{equation*}
    \mapto{f}{x}{y}\; .
\end{equation*}
Tato definice by nám měla být povědomá, neboť takto jsme si nejspíše na střední škole definovali funkci. Jaký je tedy rozdíl mezi \textbf{funkcí} a \textbf{zobrazením}? Ve skutečnosti toto není v matematice jednotné. V určitých odvětvích se tyto termíny považují za synonyma a jinde se zase naopak funkcí nazývá speciální typ zobrazení, kdy množina $Y$ je číselná, tj. $\R$, $\C$, $\Q$, \dots (tedy funkce je zobrazení, avšak ne naopak). My tyto pojmy budeme v dalším textu rozlišovat, aby byl výklad jasnější.\par
Např. zobrazení $\map{f}{\set{1,2,3,4}}{\set{a,b,c}}$, kde $f=\set{(1,b),\,(2,b),\,(3,a),\,(4,c)}$ je znázorněno na obrázku \ref{fig:zobrazeni}.
\begin{figure}[h]
    \centering
    \includegraphics{ch02_zobrazeni.pdf}
    \caption{Grafické znázornění zobrazení $f=\set{(1,b),\,(2,b),\,(3,a),\,(4,c)}$.}
    \label{fig:zobrazeni}
\end{figure}
U zobrazení $\map{f}{X}{Y}$, kde $\mapto{f}{x}{y}$, se
\begin{itemize}
    \item $x$ nazývá \emph{vzor} prvku $y$ a
    \item $y$ se nazývá \emph{obraz} prvku $x$ nebo také \emph{hodnota zobrazení $f$ v bodě $x$}.
\end{itemize}
Množiny $X$ a $Y$ pak po řadě nazýváme \emph{množina vzorů} a \emph{množina obrazů}. U funkcí je zvykem tyto množiny nazývat \emph{definiční obor} a \emph{obor hodnot}.\par
Též se zavádí \emph{obraz množiny}, tj. je-li $A\subseteq X$, pak
\begin{equation*}
    f(A)=\set{f(a) \admid a\in X}\; .
\end{equation*}
V případě funkce, co by podmnožiny kartézského součinu, byl čtenář nejspíše zvyklý je zadávat pomocí tzv. \emph{funkčního předpisu}, např. $\map{f}{\R}{\R}$, přičemž $f(x)=x^3-x^2+1$. Tu jsme znázorňovali pomocí \emph{grafu} (viz \ref{fig:funkce_graf}).
\begin{figure}[h]
    \centering
    \begin{tikzpicture}[line cap=round,line join=round,>=triangle 45,x=2.0cm,y=2.0cm]
    \begin{axis}[
    x=2.0cm,y=2.0cm,
    axis lines=middle,
    xmin=-2.0,
    xmax=2.0,
    ymin=-1.0,
    ymax=3.0,
    xtick={-2.0,-1.0,...,2.0},
    ytick={-1.0,0.0,...,3.0},]
    \clip(-2.,-1.) rectangle (2.,3.);
    \draw[line width=1.2pt,smooth,samples=100,domain=-2.0:2.0] plot(\x,{(\x)^(3)-(\x)^(2)+1});
    \begin{scriptsize}
    \draw[color=black] (-1.296792449902299,-2.9137778105094547) node {$f$};
    \end{scriptsize}
    \end{axis}
\end{tikzpicture}
    \caption{Graf funkce $\map{f}{\R}{\R}$, kde $f(x)=x^3-x^2+1$.}
    \label{fig:funkce_graf}
\end{figure}
Tento způsob proto budeme používat i u zobrazení (tedy nejen u funkcí).
\medskip

Skládání zobrazení je zcela stejné, jako v případě relací (ostatně zobrazení je relace). Avšak pro ujasnění si jej zformulujme jako samostatnou definici.
\begin{definition}[Skládání zobrazení]\label{def:skladani_zobrazeni}
    Nechť $\map{f}{X}{Y}$ a $\map{g}{Y}{Z}$ jsou zobrazení. \emph{Složením zobrazení $f$ a $g$} nazveme zobrazení $\map{h}{X}{Z}$, pro které platí
    \begin{equation*}
        \forall x\in X: h(x)=g(f(x))\; .
    \end{equation*}
    Složení zobrazení $g$ a $f$ se značí (stejně jako u relací) $g\circ f$, tzn. $h=g \circ f$.
\end{definition}
Podle právě zformulované definice \ref{def:skladani_zobrazeni} tedy platí:
\begin{equation*}
    \forall x\in X: \big(g\circ f\big)(x)=g(f(x))\; .
\end{equation*}
\needspace{1cm}
\begin{definition}[Důležité druhy zobrazení]\label{def:druhy_zobrazeni}
    Nechť je dáno zobrazení $\map{f}{X}{Y}$. Pak $f$ je
    \begin{enumerate}[label=(\roman*)]
        \item \emph{prosté} (též \emph{injektivní} či \emph{injekce}), jestliže $\forall x,y\in X: x\neq y \implies f(x)\neq f(y)$.
        \item \emph{na} (též \emph{surjektivní}\footnote{Z francouzštiny, čteme "syrjektivní"/"syrjekce".} či \emph{surjekce}), jestliže $\forall y\in Y, \exists x\in X: f(x)=y$.
        \item \emph{vzájemně jednoznačné} (též \emph{bijektivní} či \emph{bijekce}), když $f$ je prosté a na.
    \end{enumerate}
\end{definition}
\begin{example}
    Ukázky některých zobrazení a jejich klasifikace podle \ref{def:druhy_zobrazeni}. ($A$ je libovolná množina.)
    \createcnt{funccnt}
    \begin{enumerate}[label=(\roman*)]
        \item Zobrazení $\map{f_{\printnstepcnt{funccnt}}}{\Z}{\Z}$, kde $f_{\printcnt{funccnt}}(n)=-n$, je \emph{bijekce}.
        \item Zobrazení $\map{f_{\printnstepcnt{funccnt}}}{\Z}{\N}$, kde $f_{\printcnt{funccnt}}(n)=\abs{n}+1$, je \emph{na}, avšak není \emph{prosté} a tedy ani \emph{bijekce}.
        \item Zobrazení $\map{f_{\printnstepcnt{funccnt}}}{\R}{\R_0^+}$, kde $f_{\printcnt{funccnt}}(x)=x^2$, je \emph{na}, ale není \emph{prosté}.
        \item Zobrazení $\map{f_{\printnstepcnt{funccnt}}}{\R}{\R_0^+}$, kde $f_{\printcnt{funccnt}}(x)=x^2+1$, není \emph{prosté}, ani \emph{na}.
        \item Zobrazení $\map{f_{\printnstepcnt{funccnt}}}{\R}{\R^+}$, kde $f_{\printcnt{funccnt}}(x)=e^x$, je \emph{bijekce}.
        \item\label{item:inverzni_zobrazeni} Zobrazení $\map{f_{\printnstepcnt{funccnt}}}{A^2}{A^2}$, kde $f_{\printcnt{funccnt}}\big((x,y)\big)=(y,x)$, je \emph{bijekce}.
        \item\label{item:identita} Zobrazení $\map{f_{\printnstepcnt{funccnt}}}{A}{A}$, kde $f_{\printcnt{funccnt}}(x)=x$, je \emph{bijekce}.
        \item \sloppy Zobrazení $\map{f_{\printnstepcnt{funccnt}}}{A}{\powset{A}}$, kde $A$ je libovolná množina a $f_{\printcnt{funccnt}}(a)=\set{X\in\powset{A}\admid a\in X}$, je \emph{bijekce}.
    \end{enumerate}
\end{example}
Psát v matematice $f\big((x_1,x_2,\ldots,x_n)\big)$ je nezvyklé (jak jsme provedli v \ref{item:inverzni_zobrazeni}). Nejspíše by dávalo větší smysl v takovém případě nepsat vnořené závorky. Proto si zaveďme následující úmluvu.
\begin{convention}
    \sloppy Zápis $f\big((x_1,x_2,\ldots,x_n)\big)$ budeme nahrazovat symbolem $f(x_1,x_2,\ldots,x_n)$ stejného výzanmu (tj. obraz uspořádané $n$-tice).
\end{convention}
Poslední bod \ref{item:identita} je dosti významným příkladem zobrazení, které si zaslouží vlastní definici (viz \ref{def:identita}).
\begin{definition}[Identita]\label{def:identita}
    Nechť $\map{f}{A}{A}$ je zobrazení takové, že $f(x)=x$. Pak $f$ nazýváme \emph{identitou}. Identické zobrazení z $A$ do $A$ značíme $1_A$.
\end{definition}
(Inspirováno \cite{Becvar2019}, str. 10.)\\
U zobrazení a jejich skládání můžeme pozorovat jisté závislosti. Jejich důkazy jsou triviální a plynou přímo z definice, ale přesto si je zde uvedeme.
\begin{lemma}[Vlastnosti skládání zobrazení]
    Nechť $\map{f}{X}{Y}$ a $\map{g}{Y}{Z}$ jsou zobrazení. Pak
    \begin{enumerate}[label=(\roman*)]
        \item jsou-li $f,g$ prostá zobrazení, je i $g\circ f$ prosté zobrazení.
        \item jsou-li $f,g$ zobrazení na, je i $g\circ f$ zobrazení na.
        \item jsou-li $f,g$ bijekce, je i $g\circ f$ bijekce.
    \end{enumerate}
\end{lemma}
\begin{proof}
    \textit{(i)}. Budiž dány prvky $x,y\in X$ takové, že $x\neq y$. Protože $f$ je prosté, pak $f(x)\neq f(y)$. Protože prvky $f(x),f(y)\in Y$ jsou různé a $g$ je prosté, pak $g(f(x))\neq g(f(y))$.\par
    \textit{(ii)}. Mějme prvek $x\in X$. Je-li $f$ na, pak existuje $y$, takové, že $f(x)=y\in Y$. Analogicky pro $g$, které je na, platí, že existuje prvek $z\in Z$, takový, že $f(y)=z$. Tzn. $g(f(x))=z$.\par
    \textit{(iii)}. Přímý důsledek \textit{(i)} a \textit{(ii)}.
\end{proof}
(Kapitola inspirována \cite{MatousekNesetril2009}, str. 39--43.)