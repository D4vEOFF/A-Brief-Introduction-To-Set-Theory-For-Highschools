%%% Tento soubor obsahuje definice různých užitečných maker a prostředí %%%
%%% Další makra připisujte sem, ať nepřekáží v ostatních souborech.     %%%

%%% Drobné úpravy stylu

% Tato makra přesvědčují mírně ošklivým trikem LaTeX, aby hlavičky kapitol
% sázel příčetněji a nevynechával nad nimi spoustu místa. Směle ignorujte.
%\makeatletter
%\def\@makechapterhead#1{
%  {\parindent \z@ \raggedright \normalfont
%   \Huge\bfseries \thechapter. #1
%   \par\nobreak
%   \vskip 20\p@
%}}
%\def\@makeschapterhead#1{
%  {\parindent \z@ \raggedright \normalfont
%   \Huge\bfseries #1
%   \par\nobreak
%   \vskip 20\p@
%}}
%\makeatother

\setlength{\parskip}{0.3em}

% Toto makro definuje kapitolu, která není očíslovaná, ale je uvedena v obsahu.
\def\chapwithtoc#1{
\chapter*{#1}
\addcontentsline{toc}{chapter}{#1}
}

% Trochu volnější nastavení dělení slov, než je default.
\lefthyphenmin=2
\righthyphenmin=2

% Zapne černé "slimáky" na koncích řádků, které přetekly, abychom si
% jich lépe všimli.
\overfullrule=1mm

%%% Makra pro definice, věty, tvrzení, příklady, ... (vyžaduje baliček amsthm)

\theoremstyle{plain}
\newtheorem{theorem}{Věta}[section]
\newtheorem{lemma}[theorem]{Lemma}
\newtheorem{assertion}[theorem]{Tvrzení}
\newtheorem*{assertion*}{Tvrzení}

\theoremstyle{definition}
\newtheorem{definition}[theorem]{Definice}
\newtheorem{corollary}[theorem]{Důsledek}
\newtheorem{example}[theorem]{Příklad}
\newtheorem{remark}[theorem]{Poznámka}
\newtheorem{convention}[theorem]{Úmluva}
\newtheorem{denoting}[theorem]{Značení}

%%% Prostředí pro důkazy

\renewenvironment{proof}{
  \par\medskip\noindent
  \textit{Důkaz}.
}{
\newline
\rightline{$\qedsymbol$}
}
\newenvironment{solution}{
  \par\medskip\noindent
  \textit{Řešení}.
}{
\newline
\rightline{$\qedsymbol$}
}

%%% Prostředí pro sazbu kódu, případně vstupu/výstupu počítačových
%%% programů. (Vyžaduje balíček fancyvrb -- fancy verbatim.)

\DefineVerbatimEnvironment{code}{Verbatim}{fontsize=\small, frame=single}

%%% Prostor reálných, resp. přirozených čísel
\newcommand{\R}{\mathbb{R}}
\newcommand{\N}{\mathbb{N}}
\newcommand{\Q}{\mathbb{Q}}
\newcommand{\Z}{\mathbb{Z}}

%%% Užitečné operátory pro statistiku a pravděpodobnost
\DeclareMathOperator{\pr}{\textsf{P}}
\DeclareMathOperator{\E}{\textsf{E}\,}
\DeclareMathOperator{\var}{\textrm{var}}
\DeclareMathOperator{\sd}{\textrm{sd}}

%%% Příkaz pro transpozici vektoru/matice
\newcommand{\T}[1]{#1^\top}

%%% Vychytávky pro matematiku
\newcommand{\goto}{\rightarrow}
\newcommand{\gotop}{\stackrel{P}{\longrightarrow}}
\newcommand{\maon}[1]{o(n^{#1})}
\newcommand{\abs}[1]{\left|{#1}\right|}
\newcommand{\dint}{\int_0^\tau\!\!\int_0^\tau}
\newcommand{\isqr}[1]{\frac{1}{\sqrt{#1}}}

\newcommand{\norm}[1]{|| #1 ||}
\newcommand{\dotprod}[2]{\left(#1|#2\right)}
\newcommand{\compl}[1]{#1^\complement}
\newcommand{\map}[3]{#1:#2 \rightarrow #3}
\newcommand{\powset}[1]{\mathcal{P}\left(#1\right)}
\newcommand{\solutions}[1]{\left[K=\left\{#1\right\}\right]}
\newcommand{\set}[1]{\left\{#1\right\}}
\newcommand{\admid}{\;\middle\vert\;}
\newcommand{\sizeof}[1]{\left|#1\right|}
\newcommand{\dx}[1][]{
   \ifthenelse{\equal{#1}{}}%
      {\ensuremath{\,\mathrm{d}x}}%
      {\ensuremath{\,\mathrm{d}#1}}%
} % diferenciál

%%% Vychytávky pro tabulky
\newcommand{\pulrad}[1]{\raisebox{1.5ex}[0pt]{#1}}
\newcommand{\mc}[1]{\multicolumn{1}{c}{#1}}

% Full Names
\newcommand{\name}[1]{\textsc{#1}}

% Logical operators
\renewcommand{\implies}{\Rightarrow}
\renewcommand{\iff}{\Leftrightarrow}

% Paths
\newcommand{\chapterpath}[1]{components/ch#1}
\newcommand{\sectionpath}[1]{components/ch#1/sections}
\newcommand{\literaturepath}{components/literature}

% TODO
\newcommand{\todo}[1]{\textcolor{red}{(\noindent TODO: #1.)}}

% Rightsided note
\newcommand{\rightnote}[1]{\hspace*{\fill} $\triangleleft$ \textit{#1}}