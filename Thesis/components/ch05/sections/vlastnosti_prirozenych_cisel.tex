\section{Vlastnosti přirozených čísel}\label{sec:vlastnosti_prirozenych_cisel}
Podívejme se nyní blíže na vlastnosti přirozených čísel, které plynou z definice v sekci \ref{sec:prirozena_cisla}. Je celkem pozoruhodné, že čistě pomocí množin se nám podařilo vybudovat objekt, který s čísly zdánlivě na první pohled ani nesouvisí. Jáž jsme si také ukazovali, jak lze axiomaticky zavést přirozená čísla pomocí Peanových axiomů (viz \ref{sec:peanovy_axiomy}), které nám dávaly poměrně jednoznačnou představu, co od takové množiny požadujeme. Skutečně, množina přirozených čísel definovaná jako
\begin{equation*}
    \set{\emptyset,\set{\emptyset},\set{\emptyset,\set{\emptyset}},\set{\emptyset,\set{\emptyset},\set{\emptyset,\set{\emptyset}}},\dots}
\end{equation*}
splňuje tyto axiomy\footnote{Mějme na paměti, že v rámci teorie množin \textbf{Peanovy axiomy} již nejsou axiomy v pravém slova smyslu, nýbrž věty (tvrzení, která lze odvodit z axiomů \ZF).}. O jejich platnosti se čtenář může dočíst v knize \cite{Goldrei2017}, str. 40--41 a str. 43--44.\par
Na konci sekce \ref{sec:prirozena_cisla} jsme poukázali na to, že zavedení přirozených čísel použitým způsobem má za důsledek, že každý z prvků obsahuje všechny své předchůdce:
\begin{equation*}
    0=\emptyset,\quad 1=\set{0},\quad 2=\set{0,1},\quad 3=\set{0,1,2},\quad 4=\set{0,1,2,3},\quad\dots
\end{equation*}
Je tak docela pěkně vidět, co v řeči množin znamená, když je nějaké přirozené číslo větší/menší než jiné.
\begin{definition}\label{def:nerovnosti}
    Nechť jsou dána $n,m\in\N_0$. Pak definujeme
    \begin{enumerate}[label=(\roman*)]
        \item $n<m\stackrel{\text{def.}}{\iff}n\in m$,
        \item $n\leq m\stackrel{\text{def.}}{\iff}n<m\lor n=m$,
        \item $m>n\stackrel{\text{def.}}{\iff}n<m$,
        \item $m\geq n\stackrel{\text{def.}}{\iff}n>m\lor n=m$.
    \end{enumerate}
\end{definition}
O relaci "$\leq$" jsme již v minulé sekci o uspořádáních \ref{subsec:relace_usporadani} ukázali, že na $\N$ se jedná o lineární uspořádání (konkrétně v příkladu \ref{ex:cast_a_lin_usporadani}). Zde jsme s ní však nepracovali ve smyslu definice \ref{def:nerovnosti}. Zkusme se tedy přesvědčit, zdali je takto definovaná relaci skutečně lineárním uspořádáním na $\N_0$ (pro připomenutí definice viz \ref{subsec:relace_usporadani}). Nejdříve si však zformulujme následující lemma, které později využijeme.
\begin{lemma}\label{lem:o_predchudcich}
    $\forall n\in\N_0: n\subseteq n^+$.
\end{lemma}
\begin{proof}
    Platnost tohoto tvrzení lze hned vidět z definice následníka
    \begin{equation*}
        n^+=n\cup\set{n}.
    \end{equation*}
    Následník libovolného $n\in\N_0$ obsahuje všechny jeho předchůdce.
\end{proof}
Tento princip můžeme zobecnit pro libovolnou dvojici $n,m\in\N_0$.
\begin{corollary}\label{cor:o_predchudcich}
    $\forall n,m\in\N_0,\,n\leq m: n\subseteq m$.
\end{corollary}
\begin{theorem}
    $(\N_0,\leq)$ je lineárně uspořádaná množina.
\end{theorem}
\begin{proof}
    Nejdříve se přesvědčíme, že relace "$\leq$" je skutečně uspořádáním, tj. ověříme, zda je reflexivní, antisymetrická a tranzitivní.
    \begin{itemize}
        \item \textbf{Reflexivita} je celkem zřejmá z definice \ref{def:nerovnosti}. Vezmeme-li libovolný prvek $n\in\N_0$, pak určitě $n\leq n$, neboť $n\notin n$, avšak triviálně platí $n=n$.
        \item \textbf{Antisymetrie}. Uvažujme $n,m\in\N_0$, přičemž $n\neq m$. Zde opět využijeme náš poznatek, že každý prvek obsahuje všechny své předchůdce, tzn. pokud platí $n\in m$, a tedy $n\leq m\land n\neq m$, pak nemůže platit $m\in n$, tj. $m\leq n$ a naopak.
        \item \textbf{Tranzitivita}. Mějme prvky $n,m\leq\N_0$, tak, že $n\leq m$. Ukážeme, že $n\in\ell$. Podle důsledku \ref{cor:o_predchudcich} tedy platí
        \begin{equation*}
            n\subseteq m\subseteq\ell\implies n\subseteq\ell.
        \end{equation*}
        Nyní mohou nastat dvě možnosti:
        \begin{enumerate}[label=(\alph*)]
            \item $n=\ell$. Pak z definice relace nerovnosti platí $n\leq\ell$.
            \item $n\neq\ell$. Z výše odvozeného, tj. $n\subseteq\ell$, plyne, že $n$ je nutně předchůdcem $\ell$ a platí $n\in\ell$, tzn. $n\leq\ell$.
        \end{enumerate}
    \end{itemize}
    Tedy $(\N_0,\leq)$ je skutečně uspořádaná množina.
\end{proof}