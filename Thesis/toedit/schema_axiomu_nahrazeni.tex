\section{Schéma axiomů nahrazení}\label{sec:schema_axiomu_nahrazeni}
\begin{align*}
    \forall u\,\forall v\,\forall v^\prime\,(\varphi(u,v) \land \varphi(u,v^\prime) \implies v=v^\prime)\implies\\ \implies \forall a\,\exists z\,\forall x\,\bigl(x\in z \iff \exists y\,(y\in a \land \varphi(y,x))\bigr),
\end{align*}
kde formule $\varphi(u,v)$ neobsahuje proměnné $v^\prime$ a $z$.\par
Tento axiom je pravděpodobně nejsložitější, co do jeho zápisu. Zaměřme se nyní pouze na předpoklad
\begin{equation*}
    \forall u\,\forall v\,\forall v^\prime\,(\varphi(u,v) \land \varphi(u,v^\prime) \implies v=v^\prime).
\end{equation*}
Ten udává, jakou vlastnost musí splňovat formule $\varphi(u,v)$. Tvrzení je takové, že pokud existují množiny $v,v^\prime$ takové, že platí $\varphi(u,v)$ i $\varphi(u,v^\prime)$, pak množiny $v$ a $v^\prime$ musí být stejné. Resp. předpoklad požaduje, aby pro každé $u$ platila formule $\varphi(u,v)$ pro nejvýše jeden prvek $v$. Ekvivalentně bychom toto mohli napsat jako
\begin{equation*}
    \forall u\,\exists! v: \varphi(u,v).
\end{equation*}
Toto by nám již mělo být povědomé. Podobně jsme definovali zobrazení v definici \ref{def:zobrazeni}. V tomto případě můžeme tak $\varphi$ chápat jako formuli udávající, zda obrazem prvku $u$ je prvek $v$.\par
Druhá část
\begin{equation*}
    \forall a\,\exists z\,\forall x\,\bigl(x\in z \iff \exists y\,(y\in a \land \varphi(y,x))\bigr)
\end{equation*}
nám zaručuje, že všechny prvky $v$, kterým odpovídá (v rámci formule $\varphi(u,v)$) nějaký prvek $u\in a$, tvoří množinu $z$. Stručně řečeno, \textbf{obrazem libovolné množiny při definovatelném zobrazení je opět množina}.\par
Tento axiom nebyl součástí původních Zermelových axiomů. Posléze se však ukázalo, že existují množiny, jejichž existence není zbývajícími axiomy implikovaná. Např.
\begin{equation*}
    m=\set{x,\powset{x},\powset{\powset{x}},\powset{\powset{\powset{x}}},\dots},
\end{equation*}
kde $x\neq\emptyset$. Z axiomu nekonečna zaručující existenci nekonečné množiny $z$ víme, že pokud $x$ je prvkem $z$, pak i $x\cup\set{x}$ je prvkem $z$. Není těžké si rozmyslet, že toto pro $m$ není splněno. Nicméně při vhodné volbě formule $\varphi$ lze definovat zobrazení prvků nějaké aktuálně nekonečné množiny postulované axiomem nekonečna na množiny $x,\powset{x},\powset{\powset{x}},\dots$ a podle axiomu nahrazení tak tyto obrazy
\begin{equation*}
    \set{x,\powset{x},\powset{\powset{x}},\powset{\powset{\powset{x}}},\dots}
\end{equation*}
tvoří opět množinu.