\chapter{Základní pojmy a značení}\label{chap:zakladni_pojmy_a_znaceni}

V této kapitole zavedeme některá základní značení a pojmy, které dále v tomto textu využijeme. Je dosti možné, že některé záležitosti již čtenář dobře zná nebo o nich slyšel. Proto některé sekce věnující se konkrétním pojmům jsou formulovány spíše jako připomenutí než zavádění něčeho nového. I přesto však považuji jejich zmínku za nezbytnou, neboť se jedná o základní stavební kameny, na nichž budeme stavět další teorii.

\input{\sectionpath{02}/vyrokova_logika.tex}
\input{\sectionpath{02}/kvantifikatory_a_predikatovy_pocet.tex}
% \input{\sectionpath{02}/mnoziny_a_cisla.tex}
% \input{\sectionpath{02}/dukazy.tex}
% \input{\sectionpath{02}/relace.tex}
% \input{\sectionpath{02}/zobrazeni.tex}
% \input{\sectionpath{02}/relace_podrobneji.tex}
% \input{\sectionpath{02}/specialne_o_usporadanych_mnozinach.tex}