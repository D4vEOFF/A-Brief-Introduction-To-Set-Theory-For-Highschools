\section{Axiom nekonečna}\label{sec:axiom_nekonecna}
Už jsme zde ukázkově zmínili nám asi jedny z nejznámějších množin jako jsou přirozená čísla $\N$ nebo reálná čísla $\R$. Ačkoliv je šestice již zmíněných axiomů poměrně silná a~umožňuje nám pracovat velkou škálou množin, přesto by bylo stále ambiciózní tvrdit např. o~přirozených, racionálních či reálných číslech jako o~množinách v~kontextu \ZF{}. Axiom dvojice nebo axiomy sumy nám zatím dávají možnost mluvit pouze o~konečných množinách, byť libovolně velkých. Mohli bychom ještě udat jako důvod, že množiny jsou pro nás jediným přípustným objektem (jak jsme zmiňovali na začátku kapitoly) a~prvky množin musí být opět množiny. Jak ale později uvidíme (viz \ref{chap:budovani_ciselnych_mnozin}), číselné obory jsou také množinami v~\ZF{}. Podívejme se na axiom nekonečna \ref{item:axiom_nekonecna}.
\medskip

\begin{equation*}
    \exists y\, (\emptyset\in y \land \forall x\,(x\in y\implies x\cup\set{x}\in y)
\end{equation*}
Existuje množina $y$, kde pro každý její prvek $x$ platí, že je prvkem i~$x\cup\set{x}$. Zkráceně tento axiom postuluje existenci \textbf{aktuálně} nekonečné množiny.\par
Zde opět narážíme na problematiku, kterou jsme diskutovali v~historické části, a~to sice \textbf{potenciální} vs \textbf{aktuální} nekonečno. Axiomy dvojice a~sumy nám zaručovaly existenci \textbf{potenciálně} nekonečných množin, zatímco axiom nekonečna nám zaručuje existenci \textbf{aktuálně} nekonečné množiny (bez udání způsobu, jak takovou sestavit z již existujících množin).\par