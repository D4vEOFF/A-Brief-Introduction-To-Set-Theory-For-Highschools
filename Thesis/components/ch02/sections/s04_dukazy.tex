\section{Důkazy}\label{sec:dukazy}
V matematice se lze setkat s celou řadou různých tvrzení. Od primitivních, jejichž platnost je zřejmá až po složitější, nad jejich platností je třeba se více zamyslet. Čtenář se nejspíše zatím spíše setkával s matematikou, která zahrnovala užívání jistých postupů. Např. zjednodušování algebraických výrazů, řešení soustav rovnic, ověřování trigonometrických identit, aj. Avšak hodně postupů v matematice je založeno na již známých výsledcích, o nichž bylo dokázáno, že jsou pravdivé. Pokud ovšem máme dokázat určité tvrzení, je třeba, aby bylo naše zdůvodnění jednoznačné a logicky správné. V této sekci se proto podíváme na důkazové techniky používané v matematice, které budeme dále v textu využívat.\par
Matematická tvrzení jsou často různě klasifikována v závislosti na jejich povaze. Základními typy jsou tyto. 
\begin{itemize}
    \item \emph{Axióm}. Tvrzení, které implicitně považujeme za pravdivé a nedokazujeme jej. S axiomatikou jsme se již částečně seznámili v historické předmluvě (viz \ref{subsec:tm_soucasnost}).
    \item \emph{Věta}. Matematické tvrzení, jehož pravdivost můžeme ověřit důkazem.
    \item \emph{Lemma}. Pomocné tvrzení, které běžně využíváme pro důkaz jiného (typicky složitějšího) tvrzení.
    \item \emph{Důsledek}. Tvrzení, které je přímým důsledkem jiného tvrzení.
\end{itemize}
Čistě formálně však mezi \textbf{větou}, \textbf{lemmatem} a \textbf{důsledkem} není žádný rozdíl.

\subsection{Důkaz přímý}\label{subsec:dukaz_primy}
Jedná se o asi nejjednodušší typ důkazu. Často jsou matematická tvrzení formulována jako implikace, tzn. "Jestliže platí $A$, pak platí $B$.". Konkrétně, např. "Je-li $x<0$, pak $x^2>0$.".\par
Myšlenka důkazu je taková, že začínáme od předpokladu $A$, z něhož dále odvozujeme dílčí tvrzení tak dlouho, až dojdeme k požadovanému závěru $B$. Symbolicky, pokud si označíme dílčí tvrzení v důkazu $X_1, X_2, \ldots, X_n$, pak vlastně dokazujeme výrokovou formuli
\begin{equation}\label{eq:primy_dukaz_formule}
    (A \implies X_1) \land (X_1 \implies X_2) \land \cdots \land (X_{n-1}\implies X_n) \land (X_n \implies B)\; .
\end{equation}
V tomto procesu dokazování se využívá tautologie \ref{item:tautologie_4} z věty \ref{thm:vyznamne_tautologie}
\begin{equation*}
    (A \implies B) \land (B \implies C) \iff (A \implies C)\; .
\end{equation*}
Z tohoto faktu vyplývá, že pokud je každá z dílčích implikací pravdivá, pak je nutně pravdivá i implikace $A \implies B$, kterou jsme chtěli dokázat\footnote{Výrokové proměnné lze v konkrétním případě nahradit příslušnými predikáty.}. Podívejme se na příklad podobný příklad z úvodu.

\begin{assertion}
    Nechť $x\in\R$. Je-li $x<0$, pak $x^2+1>0$.
\end{assertion}
\begin{proof}
    Předpokladem našeho tvrzení je $x\in\R \land x<0$. Víme, že pro každé reálné číslo $x$ platí, že $x^2\geq 0$. Z toho již plyne
    \begin{equation*}
        0 \leq x^2 < x^2+1 \implies x^2+1>0\; .
    \end{equation*}
\end{proof}
Posloupnost dokázaných implikací bychom mohli podle \eqref{eq:primy_dukaz_formule} zapsat nyní jako
\begin{equation*}
    (x\in\R \land x<0 \implies x^2\geq 0) \land (x^2\geq 0 \implies x^2+1>0)\; ,
\end{equation*}
a tedy jsme dokázali i implikaci v původním tvrzení $x<0 \implies x^2+1>0$ .

Všimněte si, že důkaz našeho tvrzení ve skutečnosti vůbec nezávisel na předpokladu, že $x<0$, neboť jak jsme správně konstatovali, tak platí
\begin{equation*}
    \forall x\in\R : x^2>0\; .
\end{equation*}
Tj. dokázané tvrzení lze zobecnit klidně zobecnit.
\begin{assertion*}
    Nechť $x\in\R$. Pak platí $x^2+1>0$ .
\end{assertion*}
\begin{remark}
    Též bychom mohli psát "Je-li $x\in\R$, pak $x^2+1>0$."; význam by byl stejný.
\end{remark}

\begin{assertion}
    Nechť $n\in\N$ je liché číslo. Pak $3n+7$ je sudé číslo.
\end{assertion}
\begin{proof}
    Začneme opět u předpokladu, že $n\in\N$ je liché číslo. To znamená, že
    \begin{equation*}
        \exists k\in\N : n=2k+1\; .
    \end{equation*}
    Po dosazení obdržíme
    \begin{equation*}
        3(2k+1)+7=6k+3+7=6k+10=2(3k+5)\; .
    \end{equation*}
    Protože $3k+5$ je přirozené číslo, pak $3n+7$ je dělitelné dvěma a je tedy sudé, což jsme chtěli dokázat.
\end{proof}
\begin{assertion}[AG nerovnost]
    Pro $a,b\in\R_0^+$ platí
    \begin{equation*}
        \sqrt{ab}\leq\dfrac{a+b}{2}\; .
    \end{equation*}
\end{assertion}
\begin{proof}
    Při důkazu tohoto tvrzení vyjdeme z jednoduchého pozorování:
    \begin{equation*}
        (\sqrt{a}+\sqrt{b})^2\geq 0\; .
    \end{equation*}
    Nyní stačí výraz upravit a dostaneme požadovanou nerovnost.
    \begin{equation*}
        (\sqrt{a}+\sqrt{b})^2 = a+2\sqrt{ab}+b\geq 0 \implies \sqrt{ab}\leq \dfrac{a+b}{2}\; .
    \end{equation*}
\end{proof}
\begin{assertion}
    Pro $\forall x,y\in\R$ platí
    \begin{equation*}
        x<y \implies x < \dfrac{x+y}{2} < y\; .
    \end{equation*}
\end{assertion}
\begin{proof}
    Zde je třeba si všimnout "dvojité" nerovnosti v dokazovaném tvrzení. To nám již napovídá, že ve skutečnosti musíme dokázat 2 dílčí tvrzení, konkrétně
    \begin{equation*}
        x < \dfrac{x+y}{2}\;\;\;\text{a}\;\;\;\dfrac{x+y}{2} < y\; .
    \end{equation*}
    Při důkazu obou částí vyjdeme opět z předpokladu. Tedy mějme libovolná čísla $x,y\in\R$ taková, že $x<y$. Pak jistě platí
    \begin{equation*}
        x+x<x+y \implies 2x<x+y \implies x<\dfrac{x+y}{2}\; .
    \end{equation*}
    Tím jsme dokázali první nerovnost. Platnost druhé dokážeme analogicky:
    \begin{equation*}
        x+y<y+y \implies x+y<2y \implies \dfrac{x+y}{2}<y\; .
    \end{equation*}
\end{proof}
(Převzato z \cite{ChartrandPolimeniZhang2014}, str. 79 a \cite{MatematickaLogikaUK2010}, sekce \emph{důkaz přímý}.)\\
Ne všechna tvrzení jsou v matematice nutně formulována jako implikace. Často se lze setkat s tvrzeními formulovanými jako ekvivalence, tj. $A \iff B$. Důkazy takových výroků jsou již trochu delší, neboť už nestačí pouze ukázat $A \implies B$. Vzpomeňme si však na tautologii, která nám dávala do souvislosti ekvivalenci s implikací (viz \ref{item:tautologie_2} ve větě \ref{thm:vyznamne_tautologie}):
\begin{equation*}
    (A \iff B) \iff (A \implies B) \land (B \implies A)\; .
\end{equation*}
Z toho je již vidět, jak u takových tvrzení při důkazu postupovat. Zkrátka dokážeme zvlášť $A \implies B$ a $A \impliedby B$.
\begin{assertion}
    Nechť $x,y\in\Z$. Pak $3 \mid xy$ právě tehdy, když $3 \mid x$ nebo $3 \mid y$\; .
\end{assertion}
\begin{proof}
    \textit{($\implies$)}. Začneme s předpokladem, že $3 \mid xy$. Víme, že pokud je číslo dělitelné třemi, pak jej lze zapsat jako $3k$, kde $k\in\Z$. Uvažujme následující případy:
    \begin{itemize}
        \item $3 \mid x \land 3 \mid y$. Tehdy tvrzení jistě platí.
        \item $3 \nmid x$. Ukážeme, že pak nutně musí platit $3 \mid y$. Pokud $x$ není dělitelné třemi, pak jej lze zapsat buď jako $3k+1$, nebo $3k+2$, kde $k\in\Z$. 
        \begin{equation*}
            xy=(3k+1)y\;\;\;\text{nebo}\;\;\;xy=(3k+2)y
        \end{equation*}
        Protože čísla $3k+1$ a $3k+2$ nejsou dělitelná třemi, pak je vidět, že musí platit $3 \mid y$.
        \item $3 \nmid y$. Zde je postup analogický. 
    \end{itemize}
    Tím máme dokázanou implikaci $3 \mid xy \implies 3 \mid x \lor 3 \mid y$.\\
    \textit{($\impliedby$)}. Nyní předpokládáme, že platí $3 \mid x \lor 3 \mid y$; chceme ukázat, že $3 \mid xy$ Bez újmy na obecnosti\footnote{Termín \emph{bez újmy na obecnosti} (někdy zkráceně \emph{BÚNO}) se v matematických textech používá v situacích, kdy může nastat více možností, avšak říkáme, že jejich důkazy jsou analogické.}, nechť je $x$ dělitelné třemi. Pak existuje $x=3k$, kde $k\in\Z$. Po dosazení dostaneme
    \begin{equation*}
        xy=(3k)y=3(ky) \implies 3 \mid xy\; .
    \end{equation*}
    Tedy dokázali jsme obě implikace a tím i původní tvrzení.
\end{proof}
V sekci \ref{sec:mnoziny_a_cisla} jsme si ukázali \emph{de Morganovy vzorce} pro množiny. Nyní si ukážeme důkaz jejich platnosti. Zároveň si tak ukážeme příklad "množinového" důkazu.
\begin{assertion}[de Morganovy vzorce]
    Nechť $A,X_1,X_2,\ldots,X_n$ jsou libovolné množiny. Pak platí
    \begin{enumerate}[label=(\roman*)]
        \item\label{item:de_morgan_sets_1} $\displaystyle A \setminus \left(\bigcup\limits_{i=1}^{n}{X_i}\right)=\bigcap\limits_{i=1}^{n}{(A \setminus X_i)}\; ,$
        \item\label{item:de_morgan_sets_2} $\displaystyle A \setminus \left(\bigcap\limits_{i=1}^{n}{X_i}\right)=\bigcup\limits_{i=1}^{n}{(A \setminus X_i)}\; .$
    \end{enumerate}
\end{assertion}
\begin{proof}
    Nejdříve si uvědomme, co vlastně říkají dané rovnosti. Vyjadřují, že výsledné množiny jsou stejné. Kdy toto platí? Právě když dané množiny obsahují stejné prvky, tj.
    \begin{equation*}
        A=B \iff \left(\forall x : x\in A \iff x\in B\right)\; .     
    \end{equation*}
    Myšlenka našeho důkazu bude tedy taková, že uvážíme libovolný prvek $x$ náležící pravé straně rovnosti a ukážeme, že rovněž náleží i straně druhé.\par
    Ukážeme si platnost \ref{item:de_morgan_sets_1}, avšak důkaz \ref{item:de_morgan_sets_2} je zcela analogický. Tedy budiž dáno $x\in A \setminus \left(\bigcup_{i=1}^{n}{X_i}\right)$. Z definice rozdílu množin (viz \ref{def:zakladni_mnozinove_operace}) tedy musí platit
    \begin{equation*}
        x \in A \setminus \left(\bigcup\limits_{i=1}^{n}{X_i}\right) \iff x\in A \land x\notin \bigcup\limits_{i=1}^{n}{X_i}\; .
    \end{equation*}
    Protože však prvek $x$ nenáleží sjednocení množin $X_1,\ldots,X_n$, pak nenáleží žádné z nich:
    \begin{equation*}
        x\notin \bigcup\limits_{i=1}^{n}{X_i} \iff \forall i\in\set{1,\ldots,n} : x\notin X_i\; .
    \end{equation*}
    Celkově tedy nyní víme, že
    \begin{equation}\label{eq:de_morgan_leva_strana}
        x\in A \land \forall i\in\set{1,\ldots,n} : x\notin X_i
    \end{equation}
    Nyní se podívejme na pravou stranu zkoumané rovnosti. Z \eqref{eq:de_morgan_leva_strana} již plyne
    \begin{align*}
        x\in A \land \left(\forall i\in\set{1,\ldots,n} : x\notin X_i\right) &\iff \forall i\in\set{1,\ldots,n} : x\in (A \setminus X_i)\\
        &\iff x\in\bigcap\limits_{i=1}^{n}{(A \setminus X_i)}\; ,
    \end{align*}
    a celkově tedy jsme dokázali implikaci
    \begin{equation*}
        x \in A \setminus \left(\bigcup\limits_{i=1}^{n}{X_i}\right) \implies x\in\bigcap\limits_{i=1}^{n}{(A \setminus X_i)}\; .
    \end{equation*}
    Prvek $x$ jsme však volili libovolně a náš argument výše je tak univerzální (platí pro všechny prvky). Zároveň si však všimněme, že všechna odvozená dílčí tvrzení jsou ekvivalencemi a tedy platí i opačná implikace
    \begin{equation*}
        x \in A \setminus \left(\bigcup\limits_{i=1}^{n}{X_i}\right) \impliedby x\in\bigcap\limits_{i=1}^{n}{(A \setminus X_i)}\; .
    \end{equation*}

    Z toho dostáváme platnost původní rovnosti v \ref{item:de_morgan_sets_1}. Tj.
    \begin{equation*}
        \left(\forall x : x \in A \setminus \left(\bigcup\limits_{i=1}^{n}{X_i}\right) \iff x\in\bigcap\limits_{i=1}^{n}{(A \setminus X_i)}\right) \iff A \setminus \left(\bigcup\limits_{i=1}^{n}{X_i}\right)=\bigcap\limits_{i=1}^{n}{(A \setminus X_i)}\; .
    \end{equation*}
\end{proof}
Obecně pokud je tvrzení takové, že platí pro všechny objekty z jisté množiny, pak bývá právě častým způsobem důkazu, že se platnost ukáže obecně pro kterýkoliv objekt z dané množiny nezávisle na jeho volbě.

\subsection{Důkaz nepřímý}\label{subsec:dukaz_neprimy}
Řada tvrzení v matematice však není až tak jednoduchá na dokázání přímo. Důkazy, které jsme si ukazovali, vždy začínaly od předpokladu a postupně jsme došli k požadovanému závěru. Lze ale postupovat i jinak. Opět se odkážeme na dříve zmíněné tautologie věty \ref{thm:vyznamne_tautologie}, konkrétně na \ref{item:tautologie_5}:
\begin{equation}\label{eq:dukaz_neprimy_tautologie}
    (A \implies B) \iff (\neg B \implies \neg A)\; .
\end{equation}
Implikace je ve skutečnosti ekvivalentní s tvrzením, že pokud neplatí závěr, pak neplatí ani předpoklad. Na této skutečnosti je založen \emph{důkaz nepřímý} (též \emph{důkaz obměnou}). Podívejme se na příklady užití.
\begin{assertion}
    Nechť $x\in\Z$ a $3 \nmid (x^2-1)$. Pak $3 \mid x$.
\end{assertion}
V tomto případě máme dvě možnosti. Buď začneme s předpokladem $3 \nmid (x^2-1)$ \linebreak a dokážeme, že $3 \mid x$ (tedy dokážeme tvrzení přímo), nebo naopak budeme předpokládat, že $3 \nmid x$ a dokážeme negaci původního předpokladu. Ačkoliv by se jistě našla možnost, jak tvrzení dokázat přímo, přesto se nejspíše zdá jednodušší začít s předpokladem, že $x$ není dělitelné třemi.
\begin{proof}
    Nechť $3 \nmid x$. Ukážeme, že platí $3 \mid (x^2-1)$. Podle předpokladu lze $x$ zapsat jako $3k+1$ nebo $3k+2$, kde $k\in\Z$. Bez újmy na obecnosti pišme $x=3k+1$. Pak
    \begin{equation*}
        x^2-1=(3k+1)^2-1=9k^2+6k+1-1=9k^2+6k=3(3k^2+2k)\implies 3 \mid (x^2-1)\; .
    \end{equation*}
    Tedy dokázali jsme, že
    \begin{equation*}
        3 \nmid x \implies 3 \mid (x^2-1)\; ,
    \end{equation*}
    což je však podle \ref{eq:dukaz_neprimy_tautologie} ekvivalentní s
    \begin{equation*}
        3 \nmid (x^2-1) \implies 3 \mid x
    \end{equation*}
    a původní tvrzení je tak dokázané.
\end{proof}
\begin{assertion}
    Nechť jsou dány množiny $A$ a $B$. Pak
    \begin{equation*}
        A \cup B=A \iff B \subseteq A\; .
    \end{equation*}
\end{assertion}
\begin{proof}
    \textit{($\implies$)}. Tuto implikaci dokážeme obměnou. Nechť jsou dány množiny $A$ a $B$ takové, že $B$ není podmnožinou $A$. Pak
    \begin{equation*}
        \exists x\in B : x\notin A\; .
    \end{equation*}
    Prvek $x$ se se tedy objeví i ve sjednocení $A \cup B$, tj.
    \begin{equation*}
        x\in A \cup B\; .
    \end{equation*}
    Ale protože $x\notin A$, pak $A \cup B \neq A$.\\
    \textit{($\impliedby$)}. Opačnou implikaci lze již dokázat přímo a využijeme zde poměrně hezkého triku, který se při dokazování podobných tvrzení využívá. Tvrdíme-li, že dvě množiny se rovnají, pak ovšem i platí, že jsou vzájemně podmnožinami té druhé. Symbolicky
    \begin{equation*}
        X = Y \iff (X \subseteq Y) \land (Y \subseteq X)\; .
    \end{equation*}
    V našem případě budeme chtít ukázat, že platí
    \begin{equation*}
        (A \subseteq A \cup B) \land (A \cup B \subseteq A)\; .
    \end{equation*}
    Platnost inkluze $A \subseteq A \cup B$ je vidět okamžitě (vyplývá z definice sjednocení), neboť pro libovolný prvek $x$ platí:
    \begin{equation*}
        x \in A \implies x \in A \cup B
    \end{equation*}
    a tedy skutečně $A \subseteq A \cup B$.\par
    Zbývá ukázat, že $A \cup B \subseteq A$. Vezměme libovolný prvek $x \in A \cup B$; ukážeme že $x\in A$. Nyní mohou nastat dvě možnosti:
    \begin{itemize}
        \item $x \in A$. Pak máme triviálně požadovaný výsledek.
        \item $x \in B$. Z předpokladu víme, že $B \subseteq A$, z čehož opět plyne $x\in A$.
    \end{itemize}
    Dokázali jsme tedy obě inkluze, tj. $A \subseteq A \cup B$ a $A \cup B \subseteq A$ a tedy platí
    \begin{equation*}
        A \cup B = A\; .
    \end{equation*}
\end{proof}
(Převzato z \cite{ChartrandPolimeniZhang2014}, str. 111.)
\subsection{Důkaz sporem}\label{subsec:dukaz_sporem}
Už jsme si představili dvě základní důkazové techniky. Nyní k nim přidáme metodu třetí -- \emph{důkaz sporem}.\par
Uvažme, že máme tvrzení ve tvaru implikace $A \implies B$, které chceme dokázat. Podle \ref{item:reductio_ad_absurdum} ve větě \ref{thm:vyznamne_tautologie} víme, že vždy platí
\begin{equation}\label{eq:dukaz_sporem_logika}
    (P \implies \neg P) \implies \neg P\; .
\end{equation}
Tato tautologie říká, že pokud z výroku $P$ lze odvodit jeho negaci $\neg P$, pak výrok $P$ neplatí.\par
Myšlenka důkazu sporem je tedy taková, že \textbf{budeme předpokládat platnost negace dokazovaného tvrzení $\neg (A \implies B)$ a dojdeme k závěru, který je v rozporu předpokladem}. Z toho pak podle \eqref{eq:dukaz_sporem_logika} plyne, že znegované tvrzení neplatí a podle \emph{zákona vyloučeného třetího} (viz \ref{item:zakon_vylouceneho_tretiho} ve větě \ref{thm:vyznamne_tautologie}) musí platit tvrzení opačné (což je původní tvrzení).\par
Ještě si vzpomeňme na tautologii
\begin{equation*}
    (A \implies B) \iff B \lor \neg A\; .
\end{equation*}
Pomocí ní můžeme psát
\begin{equation*}
    \neg (A \implies B) \equiv \neg (B \lor \neg A) \equiv A \land \neg B\; .
\end{equation*}
To ostatně dává i smysl. Implikace je nepravdivá pouze, když platí její předpoklad, ale neplatí její závěr.
\begin{assertion}
    Nechť jsou dána $a,b\in\Z$, kde $a$ je sudé a $b$ je liché. Pak\linebreak $4 \nmid (a^2+2b^2)$.
\end{assertion}
\begin{proof}
    Nejprve znegujeme dokazované tvrzení, tj.
    \begin{equation*}
        \neg \big((2 \mid a \land 2 \nmid b) \implies 4 \nmid (a^2+2b^2)\big) \iff (2 \mid a \land 2 \nmid b) \land 4 \mid (a^2+2b^2)\; .
    \end{equation*}
    Pro spor tedy předpokládejme, že je-li $a$ sudé a $b$ liché, pak výraz $a^2+2b^2$ je dělitelný čtyřmi. Tedy existují čísla $k,l\in\Z$ taková, že $a=2k$ a $b=2l-1$. Tedy
    \begin{equation*}
        a^2+2b^2=(2k)^2+2(2l-1)^2=4k^2+8l^2-8l+2=4(k^2+2l^2-2l)+2\; .
    \end{equation*}
    Výraz $4(k^2+2l^2-2l)$ je jistě dělitelný 4. Avšak protože platí $4 \mid a^2+2b^2$, pak musí také platit $4 \mid 2$. To očividně však neplatí. To znamená, že znegované tvrzení tedy neplatí a platí tvrzení původní, což jsme chtěli dokázat.
\end{proof}
(Převzato z \cite{ChartrandPolimeniZhang2014}, str. 126)
\begin{assertion}
    Prvočísel je nekonečně mnoho.
\end{assertion}
\begin{proof}
    Pro spor naopak uvažujme, že prvočísel je konečně mnoho; označme si je $p_1, p_2, \ldots, p_n$. Definujeme číslo $m$ následovně:
    \begin{equation*}
        m=p_1p_2\cdots p_n\; .
    \end{equation*}
    Nyní k číslu $m$ přičteme 1
    \begin{equation*}
        m+1=p_1p_2\cdots p_n+1\; .
    \end{equation*}
    Na závěr celou rovnost vydělíme kterýmkoliv z čísel $p_1, p_2, \ldots, p_n$; bez újmy na obecnosti zvolme $p_1$:
    \begin{equation*}
        \dfrac{m+1}{p_1}=\dfrac{p_1p_2\cdots p_n+1}{p_1}=p_2\cdots p_n+\dfrac{1}{p_1}\; .
    \end{equation*}
    Číslo $p_2\cdots p_n$ je jistě přirozené, avšak $1/p_1$ již přirozené není (nejmenší prvočíslo je 2). Je vidět, že nově vzniklé přirozené číslo $m+1$ není dělitelné žádným z prvočísel $p_1, p_2, \ldots, p_n$. To však znamená, že $m+1$ buď samo je prvočíslo, nebo je dělitelné prvočíslem, které není součástí posloupnosti $p_1, p_2, \ldots, p_n$. V obou případech však dostáváme spor a prvočísel tedy nemůže být konečně mnoho.
\end{proof}

\subsection{Důkaz matematickou indukcí}\label{subsec:dukaz_indukci}

\todo{Doplnit cvičení}