\section{Porovnávání podle počtu prvků}\label{sec:porovnavani_podle_poctu_prvku}
Nekonečné množiny mají tedy dost pozoruhodné vlastnosti. V souvislosti s předešlou úvahou by nás tak mohlo napadnout, že při plně obsazeném hotelu je hostů "stejně mnoho" jako pokojů. Ovšem z případu \ref{item:novy_host} jsme mohli vidět, že po přesunutí hostů do vedlejšího pokoje byly pak již obsazeny "pouze" pokoje $2,3,\dots$ a první tak zůstal volný. Pokud by toto byla výchozí situace, mohlo by se nám tak zdát, že hostů je méně, neboť v prvním pokoji žádný není. Avšak jediné, co se stalo je, že hosté změnili svůj, čímž se celkem přirozeně nemohl porušit jejich počet. Změnil se tedy počet hostů a pokojů, nebo jich je stejně mnoho?\par
Náš náhled se pochopitelně odráží od porovnávání velikostí konečných množin. U těch toto nečinní žádný problém, stačí spočítat jejich prvky. U nekonečných množin již prvky "spočítat" nemůžeme. Existuje však ještě jeden způsob, který ve skutečnosti vůbec nevyžaduje schopnost počítání.
\medskip

\noindent\textbf{O domorodém kmenu a slepicích}. \textit{Představme si, že domorodý kmen hledá nového náčelníka. Po několika vyřazovacích kolech zbyli poslední dva kandidáti. Ostatní členové kmenu rozhodli, že náčelníkem se stane ten z kandidátů, který vlastní více slepic. Tento kmen je však matematicky velmi primitivní a jeho příslušníci umí počítat pouze do pěti. Oba dva kandidáti však mají ostře více jak pět slepic. Kmen se tedy rozhodl postupovat takto: každý z kandidátů přinese vždy jednu svojí slepici a tímto způsobem pokračují, dokud jednomu z nich slepice nedojdou. Ten, kterému dříve dojdou slepice prohraje a druhý je tak náčelníkem.}
\medskip

Z tohoto (možná lehce humorného) příkladu je již nejspíše jasné, jak lze množiny porovnávat. Pokud mají dvě množiny stejný počet prvků, pak lze jednotlivé prvky "spárovat". V matematické řeči náčelníci pouze sestrojují zobrazení z jedné množiny slepic do druhé. Toto dává jistě smysl v kontextu konečných množin. Pokud mají množiny stejně prvků, pak mezi nimi existuje bijekce (viz obrázek \todo{doplnit obrázek}).