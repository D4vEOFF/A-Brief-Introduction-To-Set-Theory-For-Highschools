\chapter{Dodatky k axiomům teorie množin}\label{chap:dodatky_k_axiomum_tm}
\section{Schéma axiomů nahrazení}\label{sec:schema_axiomu_nahrazeni}
\begin{align*}
    \forall u\,\forall v\,\forall v^\prime\,(\varphi(u,v) \land \varphi(u,v^\prime) \implies v=v^\prime)\implies\\ \implies \forall a\,\exists z\,\forall x\,\bigl(x\in z \iff \exists y\,(y\in a \land \varphi(y,x))\bigr),
\end{align*}
kde formule $\varphi(u,v)$ neobsahuje proměnné $v^\prime$ a $z$.\par
Tento axiom je pravděpodobně nejsložitější, co do jeho zápisu. Zaměřme se nyní pouze na předpoklad
\begin{equation*}
    \forall u\,\forall v\,\forall v^\prime\,(\varphi(u,v) \land \varphi(u,v^\prime) \implies v=v^\prime).
\end{equation*}
Ten udává, jakou vlastnost musí splňovat formule $\varphi(u,v)$. Tvrzení je takové, že pokud existují množiny $v,v^\prime$ takové, že platí $\varphi(u,v)$ i $\varphi(u,v^\prime)$, pak množiny $v$ a $v^\prime$ musí být stejné. Resp. předpoklad požaduje, aby pro každé $u$ platila formule $\varphi(u,v)$ pro nejvýše jeden prvek $v$. Ekvivalentně bychom toto mohli napsat jako
\begin{equation*}
    \forall u\,\exists! v: \varphi(u,v).
\end{equation*}
Toto by nám již mělo být povědomé. Podobně jsme definovali zobrazení v definici \ref{def:zobrazeni}. V tomto případě můžeme tak $\varphi$ chápat jako formuli udávající, zda obrazem prvku $u$ je prvek $v$.\par
Druhá část
\begin{equation*}
    \forall a\,\exists z\,\forall x\,\bigl(x\in z \iff \exists y\,(y\in a \land \varphi(y,x))\bigr)
\end{equation*}
nám zaručuje, že všechny prvky $v$, kterým odpovídá (v rámci formule $\varphi(u,v)$) nějaký prvek $u\in a$, tvoří množinu $z$. Stručně řečeno, \textbf{obrazem libovolné množiny při definovatelném zobrazení je opět množina}.\par
Tento axiom nebyl součástí původních Zermelových axiomů. Posléze se však ukázalo, že existují množiny, jejichž existence není zbývajícími axiomy implikovaná. Např.
\begin{equation*}
    m=\set{x,\powset{x},\powset{\powset{x}},\powset{\powset{\powset{x}}},\dots},
\end{equation*}
kde $x\neq\emptyset$. Z axiomu nekonečna zaručující existenci nekonečné množiny $z$ víme, že pokud $x$ je prvkem $z$, pak i $x\cup\set{x}$ je prvkem $z$. Není těžké si rozmyslet, že toto pro $m$ není splněno. Nicméně při vhodné volbě formule $\varphi$ lze definovat zobrazení prvků nějaké aktuálně nekonečné množiny postulované axiomem nekonečna na množiny $x,\powset{x},\powset{\powset{x}},\dots$ a podle axiomu nahrazení tak tyto obrazy
\begin{equation*}
    \set{x,\powset{x},\powset{\powset{x}},\powset{\powset{\powset{x}}},\dots}
\end{equation*}
tvoří opět množinu.

\section{Axiom fundovanosti}\label{sec:axiom_fundovanosti}
\begin{equation*}
    \forall a\,\Bigl(a\neq\emptyset \implies \exists x:\bigl(x\in a \land x\cap a=\emptyset\bigr)\Bigr)
\end{equation*}
Tento axiom slouží svým způsobem jako omezení množin, které lze uvažovat. Tvrzení je takové, že každá neprázdná množina musí obsahovat alespoň jeden prvek, který je s ní \emph{disjunktní} (tj. má s ní prázdný průnik). Tím zamezujeme existenci některých typů množin, jako třeba množiny obsahující samy sebe, tj. $a\in a$. Jmenovitě např.
\begin{equation*}
    a=\set{a},\;b=\set{b,\emptyset}\;\text{a jiné.}
\end{equation*}
Lze se snadno přesvědčit, že při existenci takových množin by axiom fundovanosti byl porušen. Pokud bychom připustili např. existenci množiny $x^\prime$, pro kterou by platilo, že $x^\prime\in x^\prime$, pak podle axiomu dvojice \ref{item:axiom_dvojice} je též množinou i $u=\set{x^\prime}$. Podle axiomu fundovanosti musí $u$ obsahovat prvek $x$, takový, že $x\cap x^\prime=\emptyset$. Protože však pouze $x^\prime$ je prvkem $u$, pak musí nutně platit (protože $x^\prime\neq\emptyset$), že $x^\prime\cap u=\emptyset$. To ale neplatí!
\begin{equation*}
    x^\prime\cap\set{x^\prime}=x^\prime,
\end{equation*}
neboť $x^\prime\in x^\prime$. Tzn. $u$ tedy \textbf{nesplňuje} axiom fundovanosti a není tak množinou v \ZF.\par
Dalšími důsledky axiomu fundovanosti je vyloučení cyklů v relaci "býti prvkem", tj. např.
\begin{equation*}
    x_1\in x_2\in x_3\in x_1.
\end{equation*}
Trochu obecněji lze nahlédnout, že nikdy tak nemůže nastat situace, kdy bychom našli nekonečný řetězec "do sebe zanořených" množin
\begin{equation*}
    \dots \in x_n\in \dots\in x_2\in x_1\in x_0.
\end{equation*}
\medskip

Axiom fundovanosti tedy slouží jako obecná charakteristika všech myslitelných množin v \ZF. Oproti všem ostatním je tedy trochu jiného charakteru, neboť doposud zmíněné axiomy byly spíše "konstrukční". Jejich postupnou aplikací jsme byli schopni sestrojit z menších množin množiny větší. Lze ukázat, že axiom fundovanosti je ekvivalentní s tvrzením, že všechny množiny v \ZF lze generovat z prázdné množiny opakovanou aplikací axiomu potence a sumy.