\section{Vlastnosti přirozených čísel}\label{sec:vlastnosti_prirozenych_cisel}
Podívejme se nyní blíže na vlastnosti přirozených čísel, které plynou z von~Neumannovy definice v sekci \ref{sec:prirozena_cisla}. Je celkem pozoruhodné, že čistě pomocí množin se nám podařilo vybudovat objekt, který s čísly zdánlivě na první pohled ani nesouvisí. Jáž jsme si také ukazovali, jak lze axiomaticky zavést přirozená čísla pomocí Peanových axiomů (viz \ref{sec:peanovy_axiomy}), které nám dávaly poměrně jednoznačnou představu, co od takové množiny požadujeme. Skutečně, množina přirozených čísel definovaná jako
\begin{equation*}
    \set{\emptyset,\set{\emptyset},\set{\emptyset,\set{\emptyset}},\set{\emptyset,\set{\emptyset},\set{\emptyset,\set{\emptyset}}},\dots}
\end{equation*}
splňuje tyto axiomy\footnote{Mějme na paměti, že v rámci teorie množin \textbf{Peanovy axiomy} již nejsou axiomy v pravém slova smyslu, nýbrž věty (tvrzení, která lze odvodit z axiomů ZF).}. O jejich platnosti se čtenář může dočíst v knize \cite{Goldrei2017}, str. 40--41 a str. 43--44.\par
Na konci sekce \ref{sec:prirozena_cisla} jsme poukázali na to, že zavedení přirozených čísel použitým způsobem má za důsledek, že každý z prvků obsahuje všechny své předchůdce:
\begin{equation*}
    0=\emptyset,\qquad 1=\set{0},\qquad 2=\set{0,1},\qquad 3=\set{0,1,2},\qquad 4=\set{0,1,2,3},\qquad\dots
\end{equation*}
Je tak docela pěkně vidět, co v řeči množin znamená, když je nějaké přirozené číslo větší/menší než jiné.
\begin{definition}
    Nechť jsou dána $n,m\in\N_0$. Pak definujeme
    \begin{enumerate}[label=(\roman*)]
        \item $n<m\stackrel{\text{def.}}{\iff}n\in m$,
        \item $n\leq m\stackrel{\text{def.}}{\iff}n<m\lor n=m$,
        \item $m>n\stackrel{\text{def.}}{\iff}n<m$,
        \item $m\geq n\stackrel{\text{def.}}{\iff}n>m\lor n=m$.
    \end{enumerate}
\end{definition}
O relaci "$\leq$" jsme již v minulé sekci o uspořádáních \ref{subsec:relace_usporadani} ukázali, že na $\N$ se jedná o lineární uspořádání (konkrétně v příkladu \ref{ex:cast_a_lin_usporadani}). 