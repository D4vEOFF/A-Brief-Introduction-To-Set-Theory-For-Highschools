\section{Peanovy axiomy}\label{sec:peanovy_axiomy}
Jedním ze způsobů, jak popsat přirozená čísla, je zavedením určitých axiomů pro ně. Tímto se zabýval matematik, logik a filozof \name{Giuseppe~Peano} (1858--1932), který zavedl soustavu axiomů, dnes nazývanou \emph{Peanovy axiomy}, která vystihuje jejich vlastnosti. Později uvidíme, tak tyto vlastnosti splňují \textbf{právě} přirozená čísla s nulou $\N_0$\footnote{Ve skutečnosti takových množin existuje více. Lze však ukázat, že pro všechny existuje bijekce na $\N_0$, která zachovává všechny vztahy mezi jejich odpovídajícími prvky (přesněji tzv. \emph{izomorfismus}). Všechny množiny splňující Peanovy axiomy mají tak "shodnou strukturu". Důkaz tohoto faktu zde však vynecháme.}.
\medskip

\noindent\textbf{Peanovy axiomy pro přirozená čísla}:\\
$X$ je množina obsahující (speciální) prvek $0_X\in X$ a $\map{s}{X}{X}$ zobrazení takové, že platí:
\begin{enumerate}[label=({P}\arabic*)]
    \item\label{item:peanuv_axiom_1} zobrazení $s$ je prosté, tj. $\forall x,y\in X\,x\neq y: s(x)\neq s(y)$,
    \item\label{item:peanuv_axiom_2} $\forall x\in X: s(x)\neq 0_X$ a
    \item\label{item:peanuv_axiom_3} $\forall A\subseteq X\,\bigl(0_X\in X \land (x\in A\implies s(x)\in A)\implies A=X\bigr)$.
\end{enumerate}
Idea zobrazení $s$ v kontextu Peanových axiomů je taková, že každému $x$ je přiřazen jeho \emph{následník}\footnote{Tento termín si formálně definujeme v sekci \todo{doplnit odkaz} pomocí množin.} $s(x)$.
\begin{itemize}
    \item První axiom \ref{item:peanuv_axiom_1} tak říká, že pokud máme dva různé prvky $x,y$, pak nikdy nemohou mít stejného následníka.
    \item Druhý axiom \ref{item:peanuv_axiom_2} se týká speciálně prvku $0_X$ a zaručuje, že $0_X$ není následníkem žádného prvku.
    \item Třetí axiom \ref{item:peanuv_axiom_3} postuluje, že pokud libovolná podmnožina $A$ obsahuje prvek $0_X$ a pro každé její $x$ obsahuje i jeho následníka $s(x)$, pak nutně $A$ je nutně rovna celé množině $X$. Tento axiom se též nazývá \emph{princip matematické indukce} a často jej využíváme v důkazech (viz podsekce \ref{sec:dukaz_indukci} v příloze).
\end{itemize}
Z kontextu lze vidět, že prvek $0_X$ zde zastává roli čísla nula. Skutečně, pokud bychom si označili postulovanou množinu $X$ jako $\N_0$ a její prvky $1,2,3,\dots$, tj. $\N_0=\set{0,1,2,\dots}$, kde $0_X$ interpretujeme právě symbolem $0$ a funkci $s$ definovali $s(n)=n+1$, pak lze vidět, že $\N_0$ splňuje axiomy \ref{item:peanuv_axiom_1}, \ref{item:peanuv_axiom_2} a \ref{item:peanuv_axiom_3}.\par
Nutnost prvních dvou axiomů si můžeme poměrně lehce představit. První požadavek na prostost zobrazení $s$ je celkem přirozený. Např. číslo $7$ je následníkem čísla pouze čísla $6$. Nikdy tak nemůže vzniknout situace jako na obrázku \ref{fig:peanovy_axiomy_1}.
\begin{figure}[H]
	\centering
	\includegraphics[scale=\normalipe]{ch04_peanovy_axiomy_1.pdf}
    \caption{Každý prvek je následníkem právě jednoho prvku.}
    \label{fig:peanovy_axiomy_1}
\end{figure}
Podobně prvek $0$ není následníkem žádného prvku (viz obrázek \ref{fig:peanovy_axiomy_2}).
\begin{figure}[H]
	\centering
	\includegraphics[scale=\normalipe]{ch04_peanovy_axiomy_2.pdf}
    \caption{Žádný prvek není následníkem 0.}
    \label{fig:peanovy_axiomy_2}
\end{figure}
Není těžké si též uvědomit, že přirozená čísla bez nuly $\N$ též splňují \ref{item:peanuv_axiom_1}, \ref{item:peanuv_axiom_2} a \ref{item:peanuv_axiom_3}, stačí za prvek $0_X$ vzít číslo $1$.