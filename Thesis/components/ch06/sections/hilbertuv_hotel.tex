\section{Hilbertův hotel}\label{sec:hilbertuv_hotel}
Jak jsme si již uvedli v historickém úvodu (konkrétně v podsekci \ref{subsec:cantor}), matematik \name{Georg~Cantor} byl jedním z prvních, kteří se zabývali konceptem nekonečna v uvedeném smyslu, aniž by se ohlížel na předsudky, které plynuly o nekonečnu z naší intuice. Jeden takový případ jsme si již ukázali u pozorování \name{Galilea~Galilei} (viz podsekce \ref{subsec:galileo}), který však ale naopak dospěl k závěru, že porovnáváním velikostí nekonečných množin nemá smysl se zabývat. Cantorovy úvahy o nekonečnu byly tak z počátku velmi kontroverzní a odmítané, avšak dnes jsou již drtivou většinou matematické komunity uznávané. Ve své době však vznikaly různé "paradoxy nekonečna"\footnote{Uvozovky jsou zde uvedeny právě z důvodu, že dnes již tyto záležitosti za paradoxy nepovažujeme.}, které ilustrovaly, jak neintuitivní může být práce s nekonečnem. Jedním z nejznámějších paradoxů tohoto typu je tzv. \emph{Hilbertův hotel}, který je pojmenován po německém matematikovi \mbox{\name{Davidu~Hilbertovi}}\footnote{D. Hilbert si v historii matematiky připsal mnohé zásluhy. Jako první uspokojivě vybudoval axiomatickou eukleidovskou geometrii a také zformuloval 23 nejdůležitějších problémů matematiky (tehdy zatím nevyřešených), které prezentoval na 2. mezinárodním matematickém kongresu v Paříži v roce 1900, jímž nasměroval úsilí matematické komunity na další století. Některé z těchto problémů jsou dodnes otevřené a není na ně známá jednoznačná odpověď.} (1862--1943), jenž přišel s tímto myšlenkovým experimentem.
\medskip

\noindent\textbf{Zadání problému}. \textit{Představme si, že v jednom neznámém městě stojí hotel, kde pracuje matematiky znalý recepční. Hotel má jednu zvláštní vlastnosti, a to sice, že má nekonečně mnoho pokojů. Všechny pokoje v hotelu jsou jednolůžkové postupně očíslované přirozenými čísly $1,2,3,\dots$ a hotel je plně obsazen. Recepční se postupně potýká s následujícími situacemi.}
\begin{enumerate}[label=\textit{(\roman*)}]
    \item Do hotelu přijde nový zákazník. Běžný recepční by nejspíše zákazníka poslal pryč s tím, že hotel je plný. Avšak tento recepční si ale uvědomí, že i přesto může nového zákazníka snadno ubytovat. Protože hotel je nekonečný, vymyká se principům platným u konečných množin. Není tak žádný problém každého z hostů požádat, aby se přesunul do vedlejšího pokoje. To znamená, že host v pokoji č. 1 se přesune do pokoje č. 2, host v pokoji č. 2 do pokoje č. 3, atd. Situaci lze sledovat na obrázku \ref{fig:hilbertuv_hotel_novy_host}.
    \begin{figure}[H]
        \centering
        \includegraphics[scale=\normalipe]{ch06_hilbertuv_hotel_novy_host.pdf}
        \caption{Situace před a po přesunutí hostů.}
        \label{fig:hilbertuv_hotel_novy_host}
    \end{figure}
\end{enumerate}