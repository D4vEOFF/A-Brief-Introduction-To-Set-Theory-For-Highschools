\section{Kvantifikátory a predikátový počet}\label{sec:kvantifikatory_a_pred_pocet}

Logické spojky nám jistě poskytují nástroj pro vyjádření celé řady různých výroků. Vyjádřit např. výrok "Anička má zelené vlasy ($Z$) a modré oči ($M$)." tak pro nás není problém. Symbolicky bychom mohli napsat např.
\begin{equation*}
    Z_{\text{Anička}} \land M_{\text{Anička}}\; .
\end{equation*}
Co kdybychom toto chtěli prohlásit místo o jednom člověku např. o všech obyvatelích Prahy? Užitím čistě logických spojek, jak jsme prováděli doposud, bychom museli napsat např.
\begin{equation*}
    (Z_{\text{Anička}} \land M_{\text{Anička}}) \land (Z_{\text{Eva}} \land M_{\text{Eva}}) \land (Z_{\text{Tomáš}} \land M_{\text{Tomáš}}) \land \cdots \land (Z_{\text{Jiří}} \land M_{\text{Jiří}})\; .
\end{equation*}
To je sice správné, ale poměrně těžkopádné vyjádření tak jednoduchého výroku. Jistě bychom neřekli "Anička má zelené vlasy a modré oči a zároveň Eva má zelené vlasy a modré oči a zároveň \dots". Čtenář nejspíše tuší, že existuje jednodušší způsob vyjádření takového výroku, resp. výrokové formule. K tomu v logice slouží právě takzvané \emph{kvantifikátory}.\par
V praxi bychom zkrátka řekli: "Všichni obyvatelé Prahy mají zelené vlasy a modré oči." (takové tvrzení je jistě výrok). Tímto způsobem formulujeme podobné výroky i v logice. Zkrátka prohlásíme, že pro každého obyvatele Prahy $x$ platí
\begin{equation*}
    M_x \land Z_x\; .
\end{equation*}
K formálnímu zápisu podobných tvrzení využíváme tzv. \emph{univerzální kvantifikátor}, který zapisujeme pomocí $\forall$.\par
Nyní se ještě zamysleme, co nám toto říká z pohledu logiky. Tvrzení je takové, že je-li $x$ obyvatelem Prahy, pak má zelené vlasy a modré oči. V řeči logických spojek toto není nic jiného, než implikace. Označíme-li výrok "$x$ je obyvatelem Prahy." jako $P_x$, pak bychom mohli napsat
\begin{equation}\label{eq:univ_kvatifikator_priklad}
    \forall x \big(P_x \implies Z_x \land M_x\big)\; .
\end{equation}
Takový výrok bychom četli: "\textbf{Pro všechna $x$ platí}, že pokud platí $P_x$, pak platí $Z_x$ a zároveň $M_x$". Ve výrazu \eqref{eq:univ_kvatifikator_priklad} můžeme uvažovat libovolná $x$ (nemusí se ani jednat o lidi), avšak pouze u $x$, která splňují předpoklad $P_x$ tvrdíme, že splňují i závěr $Z_x \land M_x$. Proto jsme ve výrazu nepoužili konjunkci, tj.
\begin{equation*}
    \forall x \big(P_x \land Z_x \land M_x\big)\; ,
\end{equation*}
neboť bychom vzali např. obyvatele Brna, pak by byl již výrok nepravdivý (kvůli $P_x$).

Druhým typem kvantifikátoru je tzv. \emph{existenční kvantifikátor}. Uvažme, že bychom chtěli naopak říci, že ze všech obyvatel Prahy má alespoň jeden zelené vlasy a modré oči. S využitím čistě logických spojek by pak znamenalo, že jedna z dílčích formulí je pravdivá
\begin{equation*}
    (Z_{\text{Anička}} \land M_{\text{Anička}}) \lor (Z_{\text{Eva}} \land M_{\text{Eva}}) \lor (Z_{\text{Tomáš}} \land M_{\text{Tomáš}}) \lor \cdots \lor (Z_{\text{Jiří}} \land M_{\text{Jiří}})\; .
\end{equation*}
I zde však máme kratší alternativu, a to s využitím symbolu $\exists$ pro existenční kvantifikátor. Opět se však nejdřív podívejme na naše tvrzení. To říká, že existuje $x$ takové, že $x$ je obyvatelem Prahy a zároveň má zelené vlasy a modré oči. Zde si tedy naopak vystačíme pouze s konjunkcí:
\begin{equation}\label{eq:exist_kvatifikator_priklad}
    \exists x \big(P_x \land Z_x \land M_x\big)\; .
\end{equation}
Přirozeně se zde nabízí otázka, proč jen nenahradit univerzální kvantifikátor ve výrazu \eqref{eq:univ_kvatifikator_priklad} za existenční. Pokud bychom napsali
\begin{equation}\label{eq:exist_kvatifikator_priklad_implikace}
    \exists x \big(P_x \implies Z_x \land M_x\big)\; ,
\end{equation}
pak by tvrzení již neplatilo pouze na obyvatele Prahy (v případě nesplněného předpokladu je implikace pravdivá), ale třeba pro obyvatele Brna by toto tvrzení také byla pravda. Pokud by však v Praze neexistoval občan se zelenými vlasy a modrýma očima, pak by výraz \eqref{eq:exist_kvatifikator_priklad} by nepravdivý, ale výraz \eqref{eq:exist_kvatifikator_priklad_implikace} by pravdivý již byl.\par

\todo{Negace formulí s kvantifikátory}

\todo{doplnit cvičení}