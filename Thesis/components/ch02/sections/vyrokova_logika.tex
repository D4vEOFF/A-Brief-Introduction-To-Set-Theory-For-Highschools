\section{Výroková logika}\label{sec:vyrokova_logika}

Tato část je čtenáři pravděpodobně již zčásti známa ze střední školy. Řadu vět (matematických i~nematematických) lze matematicky chápat jako \emph{výrok}, tj. tvrzení, o~kterém lze jednoznačně prohlásit, zdali je, či není pravdivé. Výrokům přiřazujeme tzv. \emph{pravdivostní hodnotu}, která je buď 1 pro \emph{pravdivý} výrok, nebo 0 pro \emph{nepravdivý} výrok.\par
Za výroky lze považovat např. tvrzení:
\begin{itemize}
    \item "Prší.",
    \item "Prší a~svítí slunce.",
    \item "Nebude-li pršet, nezmoknem.",
    \item "Když bude pršet, zmokneme."
\end{itemize}
a mnohé jiné (u každého z nich jsme schopni jednoznačně určit jeho pravdivostní hodnotu). K formálnímu zápisu tvrzení v~matematice vyžíváme tzv. \emph{logické spojky} a~\emph{kvantifikátory}.

\subsection{Logické spojky}\label{subsec:logicke_spojky}
Mezi logické spojky řadíme \emph{negaci} $\neg$, \emph{konjunkci} $\land$, \emph{disjunkci} $\lor$, \emph{implikaci} $\implies$ a~\emph{ekvivalenci} $\iff$. Připomeňme si stručně jejich významy.
\begin{convention}[Abeceda pro výrokové proměnné]
    \label{conv:abeceda_vyrokovych_promennych}
    Pro označení \emph{výroků} nebo též \emph{výrokových proměnných} (tj. proměnných, které mohou nabývat pravdivostních hodnot 0, nebo 1) budeme používat velká písmena latinské abecedy $A,B,\dots,Y,Z$, případně opatřená indexy.
\end{convention}
Uvažujme libovolné výroky $A$ a~$B$.
\begin{itemize}
    \item Negace $\neg A$ má opačnou pravdivostní hodnotu než $A$.
    \item Konjunkce $A \land B$ je pravdivá právě tehdy, když je pravdivý výrok $A$ \textbf{a současně} je pravdivý výrok $B$. Tedy má-li $A$ nebo $B$ pravdivostní hodnotu 0, pak i~$A \land B$ má pravdivostní hodnotu 0. Čteme \emph{"$A$ a~(zároveň) $B$"}.
    \item Disjunkce $A \lor B$ je pravdivá, pokud alespoň jeden z výroků $A$ a~$B$ je pravdivý. Výrok $A \lor B$ je tedy nepravdivý pouze pokud jsou současně nepravdivé výroky $A$ i~$B$. Čteme \emph{"$A$ nebo $B$"}
    \item U implikace se často mluví o~výroku $A$ jako o~\emph{předpokladu} a~o~$B$ jako o~\emph{závěru}. Výrok $A \implies B$ pak říká, že pokud platí výrok $A$, \textbf{pak nutně platí} i~výrok $B$. Čteme \emph{"jestliže $A$, pak $B$", "z $A$ vyplývá $B$" či "$A$ implikuje $B$"}. Zde se hodí upozornit na to, že mezi předpokladem a~závěrem nemusí být nutně souvislost.
    \item Ekvivalence $A \iff B$ je pravdivá, pokud jsou výroky $A$ a~$B$ \textbf{současně pravdivé} nebo \textbf{současně nepravdivé}. Čteme \emph{"$A$ právě tehdy, když $B$"}.
\end{itemize}
Výroky uvedené výše obsahující dané logické spojky lze přehledně zapsat do tabulky pravdivostních hodnot (viz tabulka \ref{tab:logicke_spojky}).
\begin{table}[H]
    \centering
    \begin{tabular}{|cc|ccccc|}
    \hline
    $A$ & $B$ & $\neg A$ & $A \land B$ & $A \lor B$ & $A \implies B$ & $A \iff B$ \\ \hline
    1   & 1   & 0        & 1           & 1          & 1              & 1          \\
    1   & 0   & 0        & 0           & 1          & 0              & 0          \\
    0   & 1   & 1        & 0           & 1          & 1              & 0          \\
    0   & 0   & 1        & 0           & 0          & 1              & 1          \\ \hline
    \end{tabular}
    \caption{Tabulka pravdivostních hodnot pro základní logické spojky}
    \label{tab:logicke_spojky}
\end{table}
Vraťme se nyní ještě k implikaci. Ve skutečnosti tato logická spojka je pravděpodobně tou nejsložitější na pochopení ze všech čtyř zmíněných, neboť při neobezřetnosti je často (a to i~v~běžné mluvě) zaměňována za ekvivalenci. Uvažme tvrzení "Jestliže nebudeš jíst, nedostaneš zmrzlinu.". Synáček by v~takovou chvíli očekával, že když naopak oběd sní, tak zmrzlinu dostane, avšak ze striktně matematického hlediska mu ji tatínek i~tak dát nemusí a~přesto by nelhal. Je důležité si uvědomit, že v~případě nesplnění předpokladu nám implikace o~závěru nic neříká.\par

\subsection{Výrokové formule}
Pokud se ohlédneme za výroky, které jsme zatím uvažovali, vždy se jednalo o~výroky \emph{složené}. Vezmeme-li např. výrok "Číslo 2 je sudé, nebo liché.", tak jej lze rozdělit na dva "jednodušší" výroky, tj. "Číslo 2 je sudé." a~"Číslo 2 je liché.", přičemž dané výroky jsou spojeny disjunkcí $\lor$. Tyto výroky však již žádné logické spojky neobsahuje a~tedy je nelze dále "rozložit".
\begin{convention}[Abeceda pro výrokové formule]
    Pro označení výrokových formulí budeme používat malá písmena řecké abecedy, tj. $\alpha,\beta,\gamma,\dots$, případně opatřená indexy.
\end{convention}
Pro výroky zavádíme následující terminologii.
\begin{itemize}
    \item \emph{Výrokovou formulí} nebo též \emph{logickou formulí} nazveme výrok obsahující libovolný počet výrokových proměnných a~logických spojek.
    \item Speciálně, pokud výrok neobsahuje žádnou logickou spojku, nazýváme jej nazýváme \emph{atomickým}, resp. \emph{atomickou formulí}.
\end{itemize}
Tento popis však nelze považovat za definici, neboť je zde pochopitelně řada nepřesností. Např. $A\neg$ nebo $AB\implies\land C$ určitě nejsou korektní výrokové formule. Podrobnější informace k~tomuto se čtenář může dočíst v~příloze \ref{chap:dodatky_k_logice}.
\begin{remark}
    Občas budeme v~této sekci zkráceně psát pouze \emph{formule}. Později se zmíníme i~o~tzv. \emph{predikátových formulích}, nicméně z kontextu vždy bude zřejmé, v~jakém smyslu daný termín používáme.
\end{remark}
\begin{convention}["Rovnost" výrokových formulí]
    \label{conv:rovnost_logickych_formuli}
    Uvažujme, že máme libovolné výrokové formule $\varphi$ a~$\psi$. Pokud $\varphi$ a~$\psi$ jsou stejné výrokové formule, pak budeme psát $\varphi\sim\psi$.
\end{convention}
Řekneme-li, že výrokové formule "jsou stejné", pak se formule shodují ve svém zápisu. Máme-li např. výrokové formule
\begin{align*}
    \varphi_1&\sim (\neg A) \land \bigl(B \lor C\bigr),\\
    \varphi_2&\sim \bigl((\neg A) \land B\bigr) \lor \bigl((\neg A) \land C\bigr)\;\quad\text{a}\\
    \varphi_3&\sim \bigl((\neg A) \land B\bigr) \lor \bigl((\neg A) \land C\bigr),
\end{align*}
pak můžeme psát, že $\varphi_2\sim\varphi_3$, ale nikoliv $\varphi_1\sim\varphi_2$, byť $\varphi_1$ a~$\varphi_2$ mají shodnou tabulku pravdivostních hodnot.\par
Nyní se nabízí otázka ohledně pořadí logických operací. Mohli bychom si pomoci používáním závorek, existuje však o~něco příjemnější přístup.  Pro zjednodušení zápisu dalších výrokových formulí se proto budeme držet následující úmluvy \ref{conv:poradi_operaci}.
\begin{convention}[Pořadí logických operací]\label{conv:poradi_operaci}
    Budeme dodržovat následující pořadí logických operací:
    \begin{enumerate}[label=(\arabic*)]
        \item Negace $\neg$ má přednost před všemi ostatními logickými spojkami.
        \item Konjunkce a~disjunkce $\land,\lor$ jsou rovnocenné a~mají přednost před implikací a~ekvivalencí $\implies,\iff$, které jsou sobě rovnocenné.
    \end{enumerate}
\end{convention}
\begin{example}
    Zjednodušení některých formulí při aplikaci zavedeného pořadí logických operací v~úmluvě \ref{conv:poradi_operaci}.
    \begin{enumerate}[label=(\roman*)]
        \item $(\neg A) \land (\neg B)\quad\rightsquigarrow\quad \neg A \land \neg B$,
        \item $\neg(\neg A)\quad\rightsquigarrow\quad \neg\neg A$,
        \item $\Bigl(\bigl(A \land B\bigr) \lor \bigl(\neg C\bigr)\Bigr) \implies \bigl(\neg A\bigr) \land \bigl(\neg C\bigr)\\ \rightsquigarrow\quad(A \land B) \lor \neg C \implies \neg A \land \neg C$.
    \end{enumerate}
\end{example}

Nyní si připomeňme asi nejznámější postup pro vyhodnocování logických formulí, a~to sice \emph{tabulkovou metodu}. Její myšlenkou bylo rozdělit danou výrokovou formuli postupně na dílčí formule a~takto postupovat i~u~daných dílčích formulí. Tímto způsobem nakonec dojdeme k~samotným atomickým formulím, kde zkoumáme všechny možné kombinace jejich pravdivostních hodnot (resp. kombinace pravdivostních hodnot jejich výrokových proměnných).\par
Před ukázkou na příkladech si ještě zavedeme jedno značení, které budeme potřebovat.
\begin{definition}[Logická ekvivalence výrokových formulí]
    Mějme výrokové formule $\varphi$ a~$\psi$. Řekneme, že $\varphi$ a~$\psi$ jsou \emph{logicky ekvivalentní}, což zapisujeme jako $\varphi\equiv\psi$, pokud je formule $\varphi \iff \psi$ pro všechny pravdivostní hodnoty výrokových proměnných obsažených ve $\varphi$ a~$\psi$ pravdivá.
\end{definition}
Pokud tedy budeme mít např. formule
\begin{align*}
    \varphi &\sim \neg (A \land B),\\
    \psi &\sim \neg A \lor \neg B,
\end{align*}
pak můžeme psát $\varphi\equiv\psi$, neboť jak se lze přesvědčit, formule $\neg (A \land B) \iff \neg A \lor \neg B$ je vždy pravdivá.\par
Jaký je rozdíl mezi $\equiv$ a~$\iff$? Formálně vzato bychom mohli mezi formule jednoduše vkládat ekvivalenci, avšak pak by se nám tato logická spojka mohla plést s ekvivalencemi, které jsou součástí $\varphi$ a~$\psi$.
\begin{example}\label{ex:vyrokova_formule_1}
    Mějme formuli
    \begin{equation*}
        \varphi\sim A \land \neg B \iff A \lor B.
    \end{equation*}
    Pro jaké pravdivostní hodnoty výroků $A$ a~$B$ je formule $\varphi$ pravdivá?
    \begin{solution}
        Postupujme způsobem popsaným výše, tj. rozdělme nejdříve danou formuli na dílčí formule. V tomto případě dílčími formulemi $\varphi$ jsou
        \begin{equation*}
            \varphi_1\sim A \land \neg B \quad \text{a} \quad \varphi_2\sim A \lor B,
        \end{equation*}
        které jsou spojeny ekvivalencí $\iff$, tj.
        \begin{equation*}
            \varphi\sim \varphi_1 \iff \varphi_2.
        \end{equation*}
        Formule $\varphi_1$ obsahuje atomický výrok $A$ a~formuli $\neg B$ spojené konjunkcí $\land$.\linebreak Označme tedy ještě
        \begin{equation*}
            \varphi_3\sim \neg B
        \end{equation*}
        $\varphi_3$ již obsahuje pouze atomický výrok $B$ v~negaci $\neg$.\par
        Podívejme se nyní na dílčí formuli $\varphi_2$. Ta obsahuje atomické výroky $A$ a~$B$ spojené disjunkcí $\lor$. Zapišme nyní vše zmíněné po řadě do tabulky pravdivostních hodnot (viz tabulka \ref{tab:ex_vyrokova_formule_1}).\par
        \begin{table}[H]
            \centering
            \begin{tabular}{|cc|cccc|}
            \hline
            $A$ & $B$ & $\varphi_3\sim \neg B$ & $\varphi_1\sim A \land \neg B$ & $\varphi_2\sim A \lor B$ & $\varphi\sim A \land \neg B \iff A \lor B$ \\ \hline
            1   & 1   & 0               & 0                          & 1                    & 0                                          \\
            1   & 0   & 1               & 1                          & 1                    & 1                                          \\
            0   & 1   & 0               & 0                          & 1                    & 0                                          \\
            0   & 0   & 1               & 0                          & 0                    & 1                                          \\ \hline
            \end{tabular}
            \caption{Tabulka pravdivostních hodnot pro $\varphi_1$, $\varphi_2$, $\varphi_3$ a~$\varphi$}
            \label{tab:ex_vyrokova_formule_1}
        \end{table}
        Z tabulky \ref{tab:ex_vyrokova_formule_1} můžeme již vidět, že formule $\varphi$ je pravdivá pro $A\equiv 1$ a~$B\equiv 0$ nebo pro $A\equiv 0$ a~$B\equiv 0$.
    \end{solution}
\end{example}

Tento středoškolský postup je zcela jistě vždy funkční. Avšak ne vždy je moudré jej ihned aplikovat. Zkusme se podívat ještě na jeden příklad výrokové formule.
\begin{example}\label{ex:vyrokova_formule_2}
    Mějme logickou formuli
    \begin{equation*}
        \psi\sim (A \land \neg A \implies B) \lor \bigl((A \iff B) \land (C \lor \neg C)\bigr).
    \end{equation*}
    Pro jaké pravdivostní hodnoty výroků $A,B,C$ je formule $\psi$ pravdivá?
    \begin{solution}
        V tuto chvíli bychom aplikací metody použití v~příkladu \ref{ex:vyrokova_formule_1} museli vyšetřit pravdivostní hodnotu formule $\psi$ pro celkem $2^3=8$ různých kombinací pravdivostních hodnot $A,B,C$. Jistě bychom takto též došli k řešení, nicméně práci si můžeme značně ulehčit. (Prosím čtenáře, aby se zde pozorněji zaměřil na formuli $\psi$ v~zadání.)\par
        Ve skutečnosti jsou některé dílčí formule zjednodušitelné. Zaměřme se pro začátek na formuli
        \begin{equation*}
            A \land \neg A.
        \end{equation*}
        Může tato formule být někdy pravdivá? Jistě, že nemůže. Libovolný výrok buď \textbf{platí, a~nebo platí jeho negace}, což nikdy nemůže nastat současně. Taková formule má pak vždy pravdivostní hodnotu 0 bez ohledu na pravdivostní hodnotu $A$. Tedy
        \begin{equation*}
            A \land \neg A\equiv 0.
        \end{equation*}
        Z výše uvedeného také ovšem plyne, že formule
        \begin{equation*}
            C \lor \neg C\equiv 1,
        \end{equation*}
        neboť opět platí buď $C$, nebo jeho negace $\neg C$.\par
        Vyšetřovanou formuli $\psi$ tedy můžeme zjednodušit
        \begin{align*}
            \psi&\sim \bigl((\overbrace{A \land \neg A}^{\equiv 0}) \implies B\bigr) \lor \bigl((A \iff B) \land (\overbrace{C \lor \neg C}^{\equiv 1})\bigr)\equiv  \\ &\equiv (0 \implies B) \lor \bigl((A \iff B) \land 1\bigr).
        \end{align*}
        Tento krok nám však umožňuje provést další úpravy. Podívejme se blíže na formuli
        \begin{equation*}
            (A \iff B) \land 1.
        \end{equation*}
        Výsledek této konjunkce vždy bude záviset na pouze na pravdivostní hodnotě $A \iff B$, tzn. konjunkce je zde nadbytečná a~můžeme psát
        \begin{equation*}
            (A \iff B) \land 1 \equiv  A \iff B.
        \end{equation*}
        Čeho si lze dále všimnout je, že výrok
        \begin{equation*}
            0 \implies B
        \end{equation*}
        je také vždy pravdivý (viz tabulka \ref{tab:logicke_spojky}). Celkově se tedy výroková formule $\psi$ zjednoduší takto
        \begin{align*}
            \psi&\sim (\overbrace{0 \implies B}^{\equiv 1}) \lor \bigl((A \iff B) \land 1\bigr)\equiv  \\ &\equiv 1 \lor (A \iff B).
        \end{align*}
        Disjunkce je však pravdivá právě tehdy, když je alespoň jeden z výroků pravdivý, což zde platí. Z tohoto dostáváme výsledek, že
        \begin{equation*}
            \psi\equiv 1.
        \end{equation*}
        Tedy bez ohledu na to, jaké pravdivostní hodnoty budou mít výroky $A,B,C$, bude formule $\psi$ vždy pravdivá. Pokud bychom přeci jen přistoupili na použití tabulkové metody, které jsme se zpočátku vyhnuli, můžeme se skutečně přesvědčit, že náš závěr je správný (viz tabulky \ref{tab:ex_vyrokova_formule_2_cast_1} a~\ref{tab:ex_vyrokova_formule_2_cast_2}).
        \begin{table}[h]
            \centering
            \begin{tabular}{|ccc|cccccc|}
            \hline
            $A$ & $B$ & $C$ & $\neg A$ & $\neg C$ & $A \land \neg A$ & $C \lor \neg C$ & $A \iff B$ & $(A \land \neg A) \implies B$ \\ \hline
            1   & 1   & 1   & 0        & 0        & 0                & 1                 & 1          & 1                             \\
            1   & 1   & 0   & 0        & 1        & 0                & 1                 & 1          & 1                             \\
            1   & 0   & 1   & 0        & 0        & 0                & 1                 & 0          & 1                             \\
            1   & 0   & 0   & 0        & 1        & 0                & 1                 & 0          & 1                             \\
            0   & 1   & 1   & 1        & 0        & 0                & 1                 & 0          & 1                             \\
            0   & 1   & 0   & 1        & 1        & 0                & 1                 & 0          & 1                             \\
            0   & 0   & 1   & 1        & 0        & 0                & 1                 & 1          & 1                             \\
            0   & 0   & 0   & 1        & 1        & 0                & 1                 & 1          & 1                             \\ \hline
            \end{tabular}
            \caption{Tabulka pravdivostních hodnot podformulí formule $\psi$ (1. část).}
            \label{tab:ex_vyrokova_formule_2_cast_1}
        \end{table}
        \begin{table}[h]
            \centering
            \begin{tabular}{|ccc|cc|}
            \hline
            $A$ & $B$ & $C$ & $(A \iff B) \land (C \lor \neg C)$ & $\psi$ \\ \hline
            1   & 1   & 1   & 1                                  & 1                                                                                       \\
            1   & 1   & 0   & 1                                  & 1                                                                                       \\
            1   & 0   & 1   & 0                                  & 1                                                                                       \\
            1   & 0   & 0   & 0                                  & 1                                                                                       \\
            0   & 1   & 1   & 0                                  & 1                                                                                       \\
            0   & 1   & 0   & 0                                  & 1                                                                                       \\
            0   & 0   & 1   & 1                                  & 1                                                                                       \\
            0   & 0   & 0   & 1                                  & 1                                                                                       \\ \hline
            \end{tabular}
            \caption{Tabulka pravdivostních hodnot podformulí formule $\psi$ (2. část).}
            \label{tab:ex_vyrokova_formule_2_cast_2}
        \end{table}
    \end{solution}
\end{example}
Tento typ formulí je poměrně významný, a~proto pro ně zavádíme speciální pojmenování v~definici \ref{def:tautologie}.
\begin{definition}[Tautologie]\label{def:tautologie}
    Výrokovou formuli $\varphi$ nazveme \emph{tautologií}, pokud $\varphi\equiv 1$. To znamená, že formule $\varphi$ je pravdivá pro všechny pravdivostní hodnoty výrokových proměnných.
\end{definition}
Některé tautologie jsme využili již pri řešení příkladu \ref{ex:vyrokova_formule_2}. Uveďme si zde ještě několik dalších významných příkladů.
\begin{theorem}[Významné tautologie]\label{thm:vyznamne_tautologie}
    Následující výrokové formule jsou tautologie:
    \begin{enumerate}[label=(\roman*)]
        \item\label{item:tautologie_1} $\neg (A \iff \neg A)$
        \item\label{item:zakon_vylouceneho_tretiho} $A \lor \neg A$ \rightnote{zákon vyloučeného třetího}
        \item\label{item:zakon_identity} $A \iff A$ \rightnote{zákon identity}
        \item\label{item:zakon_dvoji_negace} $\neg\neg A \iff A$ \rightnote{zákon Dvojí negace}
        \item\label{item:de_morgan_1} $\neg (A \land B) \iff \neg A \lor \neg B$ \rightnote{De Morganovo pravidlo}
        \item\label{item:de_morgan_2} $\neg (A \lor B) \iff \neg A \land \neg B$ \rightnote{De Morganovo pravidlo}
        \item\label{item:modus_ponens} $\bigl(A \land (A \implies B)\bigr) \implies B)$ \rightnote{pravidlo Modus ponens\footnote{Česky \emph{pravidlo vynětí}}}
        \item\label{item:modus_tollens_1} $\bigl((A \implies B) \land \neg B\bigr) \implies \neg A$ \rightnote{pravidlo Modus tollens\footnote{Česky \emph{popírání důsledku}}}
        \item\label{item:reductio_ad_absurdum} $(A \implies \neg A) \implies \neg A$\rightnote{reductio ad absurdum\footnote{Česky \emph{důkaz sporem}}}
        \item\label{item:tautologie_5} $(A \implies B) \iff (\neg B \implies \neg A)$
        \item\label{item:tautologie_2} $(A \iff B) \iff (A \implies B) \land (B \implies A)$
        \item\label{item:tautologie_3} $(A \implies B) \iff B \lor \neg A$
        \item\label{item:tautologie_4} $(A \implies B) \land (B \implies C) \implies (A \implies C)$
        \item\label{item:tautologie_6} $A \land (B \lor C) \iff (A \land B) \lor (A \land C)$
        \item\label{item:tautologie_7} $A \lor (B \land C) \iff (A \lor B) \land (A \lor C)$
    \end{enumerate}
\end{theorem}

Čtenář si pravdivosti těchto výroků může ověřit prostým sestavením tabulek pravdivostních hodnot daných logických formulí. Tautologie \ref{item:modus_ponens}, \ref{item:tautologie_5} a~\ref{item:tautologie_2} se hodně využívají při dokazování tvrzení (viz příloha \ref{chap:dukazy}).